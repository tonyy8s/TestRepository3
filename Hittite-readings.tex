\documentclass[10pt]{article}
\title{Readings in Hittite}  
\author{Anthony D. Yates}
\date{} 

%this is a test comment
%here's another
%test comment 3

%%-------						----------%%
%------------ BEGIN PREAMBLE -----------------------------------------------------------------------
%%-------						----------%%

%------------ SPACING & MARGINS -------------------------------------------------------------------

\usepackage{setspace}
\usepackage[left=1in,right=1in,bottom=.6in,top=.8in]{geometry}
%\setspace{2}
\usepackage[hang,flushmargin]{footmisc} 
%\usepackage[landscape]{geometry}
%\setlength{\footnotesep}{\baselineskip}
%\renewcommand{\footnotelayout}{\doublespacing}
\usepackage{paracol}

%------------ LINGUISTIC & MATHEMATICAL CHARACTERS ------------------------------------

\usepackage{amssymb}
\usepackage{tipa}

%------------- DRAWING / FIGURES / COLOR ---------------------------------------------------------

\usepackage{tikz}
\usetikzlibrary{matrix,decorations.pathreplacing}
\usepackage{arydshln}
\usepackage{color}

%------------  PAGE HEADER & FOOTER ---------------------------------------------------------------

\usepackage{fancyhdr}
\pagestyle{fancy}
\fancyfoot{}
\fancyhead[R]{\scriptsize A.D. Yates {\textvertline} {\bf \thepage}}
\fancyhead[L]{\scriptsize Readings in Hittite}
\renewcommand{\headrulewidth}{0.25pt}
\renewcommand{\footrulewidth}{0pt}

%------------ CUSTOM SECTION HEADERS -------------------------------------------------------------

\usepackage{titlesec}
\setcounter{secnumdepth}{5}
\titleformat{\section}{\normalfont\large\bfseries}{\S \thesection}{1em}{}
\titleformat{\subsection}[runin]{\normalfont\normalsize\bfseries}{\S \thesubsection}{1em}{}
\titleformat{\subsubsection}[runin]{\normalfont\normalsize\bfseries}{\S \thesubsubsection}{1em}{}
\titleformat{\paragraph}[runin]{\normalfont\normalsize\bfseries}{\S \theparagraph}{1em}{}
\titleformat{\subparagraph}[runin]{\normalfont\normalsize\bfseries}{\S \thesubparagraph}{1em}{}

%------------  FONTS ---------------------------------------------------------------------------------------------

\usepackage[polutonikogreek,english]{babel}
\usepackage{teubner}

%------------ EXAMPLES & ENUMERATION --------------------------------------------------------------

\usepackage{comment}
\specialcomment{presentation}{\begingroup\large\color{gray}\bigskip\noindent}{\endgroup}

\excludecomment{presentation}						%%comment out for presentation version!

\specialcomment{notes}{\begingroup \small \begin{center}\rule{4cm}{0.2pt} \textbf{Notes:} \rule{4cm}{0.2pt}\end{center}}{\begin{center}\rule{4cm}{0.2pt}\end{center}\endgroup}

\usepackage{hyperref}
\usepackage{enumerate}
\usepackage{gb4e}

%--------- Special Commands & Symbols ------------------------------------------------------

\newcommand{\supersc}[1]{$^{\textrm{\scriptsize{#1}}}$}  	% for superscripts
\newcommand{\subsc}[1]{$_{\textrm{\scriptsize{#1}}}$}	% for subscripts
\newcommand{\bit}[1]{\textbf{\textit{#1}}}				% for bold/italic glossing
\newcommand{\p}[1]{{\tiny[{#1}]}}					% for grammatical glossing
%\newcommand{\comment}[1]{\large\color{gray}{\bigskip #1}}


%--------- Indo-European Special Characters --------------------------------------------------

	%---------- PIE ------------------------------------------------------------

		%-------------- Special consonants ----------------------------

\newcommand{\labk}{k{\supersc{w}}}							%---	for labial *k
\newcommand{\labg}{g{\supersc{w}}}						%---	for labial *g
\newcommand{\labgh}{g\supersc{wh}}						%---	for labial *gh
\newcommand{\palk}{\textroundcap{k}}						%---	for palatal *k
\newcommand{\palg}{\textroundcap{g}}						%---	for palatal *g
\newcommand{\palgh}{\textroundcap{g}\supersc{h}}				%---	for palatal *gh
\newcommand{\gh}{g\supersc{h}}							%---	for *dh
\renewcommand{\dh}{d\supersc{h}}							%---	for *dh
\newcommand{\bh}{b\supersc{h}}							%---	for *bh
\newcommand{\hi}{h\subsc{1}}								%---	for *h1
\newcommand{\hii}{h\subsc{2}}								%---	for *h2
\newcommand{\hiii}{h\subsc{3}}								%---	for *h3

		%------------- Special vowels ------------------------------------

\newcommand{\J}{\={\j}}
\newcommand{\I}{\={\i}}									%---	for long *i
\newcommand{\II}{\textacutemacron{\i}}						%---	for accented long *i
\renewcommand{\r}{\textsubring{r}}							%---	for syllabic *r
\newcommand{\R}{\ensuremath{\acute{\textsubring{r}}}}			%---	for accented syllabic *r
\renewcommand{\l}{\textsubring{l}}							%---	for syllabic *l
\renewcommand{\L}{\ensuremath{\acute{\textsubring{l}}}}			%---	for accented syllabic *l

		%------------ Phonological symbols ----------------------------

\newcommand{\ep}{\subsc{\textschwa}}						%---	for "schwa indogermanicum"
\newcommand{\zero}{{\footnotesize $\varnothing$}}				%---	for the "zero" sign				
\newcommand{\pr}{\'{ }}									%---	for floating acute

		%------------ General length, accents, etc. -------------------
		
\newcommand{\brlg}[1]{$\breve{\bar{\textrm{{#1}}}}$}
\newcommand{\abrlg}[1]{$\acute{\breve{\bar{\textrm{{#1}}}}}$}
\newcommand{\abr}[1]{$\acute{\breve{\textrm{{#1}}}}$}
\newcommand{\alg}[1]{$\acute{\bar{\textrm{{#1}}}}$}
\newcommand{\ac}[1]{$\acute{\textrm{{#1}}}$}	


	%------------ Hittite ---------------------------------------------------------

		%----------- Logograms --------------------------------------------
		
\newcommand{\hith}{\textsubwedge{h}}
\newcommand{\Hith}{\textsubwedge{H}}
\newcommand{\sh}{\v{s}}
\newcommand{\hpl}{\supersc{{\Hith}I.A}}
\newcommand{\man}{\supersc{L\'U}}
\newcommand{\men}{\supersc{L\'U.ME\v{S}}}
\newcommand{\mpl}{\supersc{ME\v{S}}}
\newcommand{\wood}{\supersc{GI\v{S}}}
\newcommand{\meat}{\supersc{UZU}}
\newcommand{\cloth}{\supersc{T\'UG}}
\newcommand{\stone}{\supersc{NA\subsc{4}}}
\newcommand{\wool}{\supersc{S\'IG}}
\newcommand{\field}{\supersc{A.\v{S}\`A}}
\newcommand{\city}{\supersc{URU}}
\newcommand{\bread}{\supersc{NINDA}}
\newcommand{\clay}{\supersc{DUG}}
\newcommand{\troops}{ER\'IN\supersc{ME\v{S}}}
\newcommand{\chariotry}{AN\v{S}E.KUR.RA\supersc{ME\v{S}}}
\newcommand{\vessel}{\supersc{DUG}}

		%----------- Other Hittite Characters -----------------------------
		
\newcommand{\hiverb}{\textit{{\hith}i--}verb}
\newcommand{\miverb}{\textit{mi--}verb}

	%----------- Vedic ------------------------------------------------------------

\newcommand{\ns}{\textdotbreve{m}}						%for 
\newcommand{\rr}{\ensuremath{\bar{\textsubring{r}}}}			%for long syllabic -r-
\newcommand{\RR}{\ensuremath{\acute{\bar{\textsubring{r}}}}}		%for long accented syllabic -r-
\renewcommand{\.}[1]{\textsubdot{#1}}


%------------ Bibliography ---------------------------------------------------------

\usepackage{makeidx}
\usepackage{natbib}
\bibpunct[:]{(}{)}{;}{a}{}{,}
\defcitealias{EWA}{\textit{EWA}}
\defcitealias{NIL}{\textit{NIL}}
\defcitealias{aigr1.1}{\textit{AiGr.}1}
\defcitealias{aigr2.1} {\textit{AiGr.}2.1}
\defcitealias{aigr2.2}{\textit{AiGr.}2.2}
\defcitealias{aigr3}{\textit{AiGr.}3}
\defcitealias{GrHL}{\textit{GrHL}}
\defcitealias{AHP}{\textit{AHP}}
\defcitealias{EDH}{\textit{EDH}}
\defcitealias{LIV2}{\textit{LIV}$^2$}

%--------------- Temporary Special Commands -----------------------------

%\newcommand{\ienai}{\greektext b\~h d" >i'enai}
%\newcommand{\imenai}{\greektext b\~h d" \As{i}menai}
%\newcommand{\imen}{\greektext b\~h d" \As{i}men}
%\newcommand{\ithi}{\greektext b'ask" \As{i}ji}

%%-------						----------%%
%------------------ END PREAMBLE ---------------------------------------------
%%-------						----------%%

\begin{document}
\maketitle
\thispagestyle{empty}

%------------------ START WORKING! -------------------------------------------


\section{The Anitta Text (CTH 1.A)}

\begin{description}

\item[Publication:] KUB 36.98 + 26.71 + 22.5 + 50.1 + 3.22
\item[Edition:] \citet{neu1974anitta}
\item[Background:] \citet[291-93]{hoffner1980historians}: 

``The historical inscription of Anitta, son of Pithana, king of Ku\v{s}\v{s}ar, was first edited in transliteration by E. Forrer in 1922. Since then its three exemplars have been published in cuneiform copies, and it has been translated and studied many times. Both Anitta and his father Pithana are mentioned in documents from the Old Assyrian trading colony at K\"ultepe. It is thought that Anitta's rule coincided with the period represented by k\=arum Kani\v{s} I b, which is contemporary with the reigns of
\v{S}am\v{s}i-Adad I and I\v{s}me-Dagan of Assyria (middle chronology, c. 1814-1742) -- first half of the 18th century. The Hittite composition which concerns Anitta was long believed to be a late copy from the 13th century. In 1951 Otten showed that the orthography and grammar of the text were Old Hittite, and in the 1960s and 1970s it has become clear that the sign forms of copy A are typical Old Hittite (reign of Hattusili I or Mursili I). Since Pithana and Anitta, though clearly Anatolian natives and not Assyrians, were never mentioned as ancestors by the Old Hittite kings, it was assumed that they were not Hittites. The text, it was supposed, must have been composed in another language and was subsequently translated into Hittite. The prime candidate for this other language was Hattic, the language of many of the non-Hittite Anatolian rulers of the period. In 1974 E. Neu produced a new edition of the text, utilizing the latest insights into the Old Hittite language. Neu pointed out that translations into Hittite from Hattic are always marked by a certain awkwardness (\textit{Holprigkeit}) , which betrays them as translations. The Anitta text, on the other hand, shows none of this translational syntax, but appears to be a fresh composition in Old Hittite.''
[HH goes on to conclude that this conclusion is supported by similarities in historiographical and narrative technique, but that it diverges sufficently from later compositions in such a way that it must be a precursor of this mode: ``the historiographic technique required centuries of practice before it could become the articulated science which flourished under Mur\v{s}ili.'' -ADY]

\smallskip 

cf. \citet[35-36]{bryce2005kingdom}: ``{\ldots}the so-called Anitta inscription, a text preserved in fragmentary form in three copies,74 allegedly from an original carved on a stela set up in the gate of the king's city. Although once thought to have been written in Old Assyrian, the original text was probably written in `Hittite' (Nesite). However, the earliest surviving version of it is a copy (in Hittite) apparently made during the Hittite Old Kingdom some 150 years or more after the original [so maybe the earliest copy c. 1600? -ADY]. The inscription deals with the conquests in central Anatolia of two kings who were apparently members of a ruling dynasty based originally in a city called Kussara---Pithana and his son Anitta, the author of the text{\ldots}We have no firm grounds for assuming that Pithana sought to identify himself as the champion of an Indo-European ethnic group, or that the conflicts in which he and his son Anitta engaged reflect a struggle for political and military supremacy between two ethnic groups, Hattian and Indo-European {\ldots}

\smallskip

\item[1 :] \bit{Q\'IB\'I=MA} \p{pres. impv. act.} `speak!'; Hittite proclamations do not normally begin this way; possibly a relic of a period before standardization of scribal practices

\item[2 :] \bit{nepi\v{s}z=a\v{s}=(\v{s})ta} \p{n. abl.} archaic ablative ending confined to consonant stems, with later ending [\textit{-za}] extracted from \textit{a-}stems (cf. \citetalias{GrHL} \S3.14 n.29); however, ablative makes no sense here; (CM:) unclear whether Hittite actually can do adnominal ablative; this is a long-standing problem \bit{\=a\v{s}\v{s}u\v{s}} \p{m. nom. s.; u-st.} `kind; dear'

\item[4 :] \textbf{x x x x} in break at line-end, perhaps restore `became hostile', but sense is better with `jealous' (\textit{vel sim.})

\item[5 :] \bit{pangarit} \p{instr.; r-st. (?)} `great number'; only in instr. `en masse; with a great force'; cf. \p{denom.} \textit{pangariya-} `become numerous' (secondary denominative to unattested \textit{*-ro-} adjective)

\item[6 :] \bit{nakkit} \p{n. instr.; subst. adj.} `heavy; important, serious; powerful'; 
\p{subst.} `with power / force' (only instr.)

\bit[8 :] \bit{\`U} `and; but'; here clearly `but', which shows that Akkadogram collapses Hitt. \textit{=(m)a} and \textit{=(y)a}

\item[8 :] \bit{takki\v{s}ta} \p{3s. pret. act.} `put together; devise; undertake; mingle'; on the semantics, cf. Kl \textit{s.v.}: ``The semantic interpretation of this verb is quite difficult. We find, for instance, \textit{id\=alu tak\v{s}}- `[to do ill (to someone \p{acc.})]', \textit{tak\v{s}ul tak\v{s}}- �to conclude a peace-treaty', KASKAL-\textit{\v{s}a tak\v{s}}- �to undertake a campaign', \'E\textit{-er tak\v{s}}- `to allot a house (to someone)', GE\v{S}TIN \textit{wetenit tak\v{s}}- �to mingle wine with water''';  the first of these idiomatic expressions apparent here

\item[9 :] \textbf{[\hspace{.5cm}]} (CM:) possibly restore \textit{s=u\v{s}} \bit{annu\v{s} attu\v{s}} \p{acc. pl.} `mothers (and) fathers'; note the regular asyndetic combination of these lexical items; \textit{contra} \citet[36]{bryce2005kingdom}, ``[t]his statement'' is \emph{not} ``unique in cuneiform literature;'' rather, it is a regular trope in both Hittite and more generally across Anatolia

\item[10 :] \bit{atta\v{s}=ma\v{s}} \p{c. gen. s.; a-st.} `my father'; while nom./gen. distinction is formally ambiguous, the clitic pronoun must be genitive (cf. =mi\v{s} \p{nom.}), which is obj. of postpos. \textit{appan} `after, behind' \textbf{saniya} \p{adj. c. d-l. s.; i-st.} `same'

\item[11 :] \textbf{\supersc{D}UTU}\bit{-az} \p{c. abl.} `Sun(-god)'; more likely `sun' as celestial object here, hence `to/from the east'

\item[12 :] \bit{{\hith}ullanzan {\hith}ullanun} \p{c. acc. s.; a-st.} `battle; fight' \p{1s. pret. act.} `do battle; strike down (in battle)' (cognate accusative); contextually, perhaps `put down a revolt', but `revolt' generally is (CM:) unjustified

\item[17 :] \bit{{\hith}antai\v{s}i} \p{c. d-l. s.; s-st. (?)} `warmth; heat'; effectively `heat of day' vs. \textit{i\v{s}panti} `night' in ln. 18; rare anim. \textit{*s}-stem; noun rather than adj. unexpected here

\item[20 :] \textbf{[ x x ]} possibly restore \textit{\v{s}=u\v{s}} \bit{{\hith}appar\=enun} \p{1s. pret. act.} `give/hand over; allot'; the sense `sell' is secondary

\item[21 :] \bit{ \=appa} \p{adv.} `in turn'

\item[24 :] \bit{\=appa{\ldots}a\v{s}\=a\v{s}i} \p{3s. pres. act.} `re-settle'; but {\city}\textit{N\=e}[ - remains problematic under this interpretation \textbf{[ x x ]} Neu reads \textit{l\=e} in the break (CM \textit{approbante})

\item[25 :] \bit{utniandan} \p{n. gen. pl.} `land'; archaic genitive ending \textit{-an} $<$ \textit{*-om, *-oHom} here preserved; here an \textit{-nt}-stem variant beside regular ablauting \textit{utn\=e}

\item[26 :] \textbf{UR.MA{\Hith}}\bit{-i\v{s}} \p{c. nom. s.} `lion' (= Luw. \textit{walwa/i-}); \bit{utn\=e} \p{n. acc. pl.} `land'; likely object of missing verb in break, since `to X a land like a lion' is an Akkadian topos

\item[33 :] \bit{tuppiaz} \p{n. abl.} `tablet'; acc. to CM, `(I copied) from a tablet\subsc{abl.}', with verb missing in the break; \textit{n.b.} in OH instr. reading of abl. is impossible \textbf{K\'A.GAL-\textit{YA}} \p{d-l.} `gate; tower, fortress'

\item[34 :] \bit{URRAM \v{S}ERAM} \p{adv.} `later; in the future' \textbf{{\hith}ulliy\=ezzi} \p{3s. pres. act.} `dispute, contravene' (\textit{contra} Neu `zerschlagen' or HH `alter, deface'; T. \textit{s.v.} cites as deverbative, but probably denom. to unnattested nominal \textit{*hulle/i-} `fight'; seemingly interchangeable in this text with \textit{{\hith}ulla/i-}, above in ln. 12 and more importantly, in ln. 35 immediately below

\item[36 :] \bit{d\=an namma} \p{adv.} `and furthermore'; following CM, rather than Neu's `a second time'

\item[37 :] \bit{\v{s}ardiya\v{s}=\v{s}ann=a} \p{c. gen. pl.} `helper; ally'; particle chain almost certainly contains \textit{=\v{s}an} the 3s. poss. pron. clitic rather than local particle \textit{=\v{s}an} (cf. \citetalias{GrHL} \S28.84) + geminating \textit{=a} `and';  surface [\textit{\v{s}\v{s}}] likely results here from a low level assimilation of underlying \textit{/-ns-/-}; for this rule, cf. \citet{kimball1999hhp} \S7.8.5.1: ``Prevocalic /ns/ [in sandhi -ADY] was normally realized as [ss] after a short vowel that was historically unaccented in historical terms (\textit{sic}) and presumably remained unaccented: with sentence particle \textit{-\v{s}\v{s}an}, e.g. \textit{na-as-sa-an} StBoT 25, 26 IV 5, 8, 12, 13 = / nu-an-san/.''

\item[38 :] \bit{arunaz} \p{c. abl. s.} `sea'; \textit{anda} in this position means `in the direction of'; the `Zalpa' here referenced is specifically `Zalpa-by-the-sea'

\item[39 :] \textbf{\supersc{D}}\bit{\v{S}iu\v{s}=(\v{s})ummin} \p{c. acc. s} ` idol / statue of our god Siu'; this passage confirmed reading of 1pl. poss. pron. clitic \textit{summi--} `our'

\item[43 :] \bit{{\hith}u\v{s}uwantan} \p{ptcpl. c. acc. s.} `living, alive'; archaic zero-grade form beside synchronically regular \textit{{\hith}ui\v{s}want-} to \textit{{\hith}ui\v{s}}- `be alive'

\item[44 :] \textbf{URU}\bit{{\Hith}attu\v{s}a-}\textbf{x x} CM here restores: \supersc{URU}\textit{{\Hith}attu\v{s}a[\v{s} per\=an id\=alu ta]kki\v{s}ta} `H- previously did evil, (but) I left it alone.'

\item[45 :] \bit{t\=al{\hith\hith}un} \p{3s. pret. act.} `let; leave (alone)' 

\item[46 :] \bit{ki\v{s}tanziattat} \p{3s. pret. mp.} `become hungry'; denom. \textit{kistant-} \p{c.} `hunger' $\leftarrow$ \textit{ka\v{s}t-} \p{c.} `id.'; with EN, understand `the state' as referent of subj. clitic \textit{=a\v{s}}

\item[48 :] \textbf{Z\`A.A{\Hith}.LI}\bit{-an} `cress; (a type of) weed' \textbf{aniy\=enun} \p{3s. pret. act.} `make, do; create, engender'; idiomatically here `sow' \textbf{\textit{{\hith}azziy\=ettu}}  pres. act. impv. `strike (down)'

\item[54 :] \bit{{\hith}uittiyati} \p{3s. pret. mp.} `lead'; trans. mp. in OS with \textit{--ti} ending is significant \bit{=an} \p{encl. c. acc. s.} with EN, the referent seems to be ER\'IN{\mpl}-\textit{\v{S}U} `its troops' in previous clause

\item[55 :] \bit{wetenun} \p{1s. pret. act.} `build; add' {\bf URU}\bit{-iyan} \p{c. gen. pl.} `city; fortification' = \textit{{\hith}appiriyan} with archaic gen. pl. ending, hence `behind the fortifications'

\item[56 :] \bit{ABNI} \p{Akk. 1s. pret. act.} `I build; I added' = Hitt. \textit{wetenun}

\item[56 :] \bit{\v{S}iuna\v{s}=(\v{s})ummin} `of our god Siu'; the encl. poss. must be a NH scribal error for correct \textit{*\v{s}umma\v{s}}, with the same singleton spelling of \textit{-\v{s}-} as at 39, above; and note a similar clitic spelling error at 57, below; NH scribes no longer fully command poss. pron. clitics

\item[57 :] \textbf{\supersc{D}}\bit{\v{S}iuna=\v{s}ummi\v{s}} `of our god Siu'; NH scribal error for correct \textit{*\v{s}umma\v{s}}; cf. 56, above

\item[58 :] \textbf{KASKAL}\bit{-az} \p{c. acc. s.} `campaign, expedition; journey'; here EN argues for underlying \textit{la{\hith\hith}a-} `campaign rather than \textit{palsa/i-} `way, path' \textbf{{\hith}ali\v{s}\v{s}iyanun} \p{1s. pret. act.} ` \textbf{\textit{apedanda}} \p{instr.} `that' \bit{{\hith}ali\v{s}\v{s}iyanun} \p{1s. pret. act.} `enclose, surround; supply, equip, provide with' 

\item[59 :] \bit{m\=alta{\hith}un} \p{1s. pret. act.} `recite; vow, pledge; sacrifice'; \textit{{\hith}i-}verb with nice cognates Arm. \textit{malt'em}, Lith. \textit{meld\v{z}i\`u}  `pray'; OSax. \textit{meldon} `tell', `pray' $<$ \textit{*meldh-}, which even AK \textit{s.v.} reconstructs with normal root-internal \textit{*o/{\zero}} ablaut \bit{({\hith}uwar-)} \p{1s. pret. act.} contra Neu, better to restore [\textit{na{\hith}{\hith}un}] than [\textit{ta{\hith}{\hith}un}], hence from \textit{{\hith}uwarn--} `(go) hunt' than from \textit{{\hith}uwart--} `curse'; the former also fits better with the directly following description of hunting exploits

\item[60 :] \textbf{\v{S}A{\Hith} GI\v{S}\textit{\supersc{\textsubdot{S}\'I}}} `pigs of the reed'

\item[61 :] \textbf{AZ{\hpl}} `bear; wild animal (?)' \textbf{\textit{L\=U{\ldots}L\=U{\ldots}ULU{\ldots}}} `whether they are {\ldots} or {\ldots} or {\ldots}' \textbf{UG.TUR} `leopard' \textbf{D\`ARA.MA\v{S}} `(lit.) `deer-goat' = `ibex; wild/horned goat' \textbf{D\`ARA} `deer'

\item[64 :] \bit{w\=etandanni=\v{s}\v{s}i=ma} \p{c. d-l. s.} `year('s interval)'; here, effectively `in the next year'; nom. s. \textit{*w\=ettantatar} \p{r/n-st.}, with the long vowel of the base also apparent in the derivative (per CM); the ptcl. \textit{=\v{s}\v{s}i} is inexplicable, and likely to be an error

\item[66 :] \bit{dali\v{s}} \p{3s. pret. act.} `let; leave'  (st. \textit{dala-}); arguably an example of QSV \index{quasi-serial verb construction} in OH with \textit{w\=et}

\item[69 :] \bit{lukkit} \p{3s. pret. act.} `light up; kindle, ignite' ; cf. MP \textit{luk(k)-} `become light'

\item[70 :] \bit{{\hith}ulale\v{s}\v{s}ar} \p{n. nom./acc. s.} `extent; enclosure, perimeter'; CM reads instead URU\textit{-riya\v{s}} `perimeter of city', followed by a list of defenders

\item[71 :] \bit{\textsubdot{S}\'IMTI} EN suggests `colored; marked', interpreting the Akkadogram with the following logograms, hence `colored{\ldots}silver and gold'; this reading is purely contextual, though, and elsewhere problematic

\item[72 :] \bit{{\hith}uittitti} \p{3s.pret.mp.} `drag, draw; \p{mp.} draw oneself back; withdraw'

\item[74 :] \bit{{\hith}enkumu\v{s}} \p{???} `gift, present'; this spelling is wholly mysterious; CM tentatively suggests emendation/restoration \textit{{\hith}enku[nit uwet]} `he came with gifts'

\item[75 :] \textbf{PA.GAM} `sceptre, staff'

\item[78 :] \bit{tunnaki\v{s}na} \p{n. all. s.; r/n-st.} `inner chamber; bedroom' = \'E.\v{S}\`A \textbf{\textit{paizzi}} `goes'; switch into historical present, which is continued in the main clause

\item[79 :] \bit{p\=eram=mit} `(lit.) my before; before me'; enclitic shows original OH situation, where it agrees with the neuter nominal; phonologically, note low level assimilation /-\textit{n=m}-/ $\rightarrow$ [-m=m-] and the fluctuation already apparent in OH between \textit{p\=eran} and \textit{peran} (cf. AK \textit{s.v. peran} and observe the interesting \textit{per\=an=mit} found in the duplicate, one of the few exx. which suggest that cliticization entails a shift of accent to the immediately preceding syllable similar to Cl. Lat. \citep[cf.][106-107]{AHP}

\end{description}

\section{Fragments mentioning Anum-{\Hith}irbi and the city of Zalpa (CTH 2.2)}

\begin{description}

\item[Publication :] KUB 36.99
\item[Edition :] None? 

\item[Ro 6 :] \textbf{\textit{katti=\v{s}\v{s}i}} `with him'; if rightly restored, a nice OH archaism

\item[Vo 5 :] \textbf{\textit{tuttut}} \p{3s. pret. act.} `treat mercificully'; st. \textit{duwaddu-} ; curiously, according to T \textit{s.v.}, \textit{duddu-} should be later stem

\item[Vo 6 :] \textbf{\textit{kuitki}} looks like a postposed relative clause, though syntactically odd: `he took [his goods]\subsc{i}, [whichever (ones)]\subsc{i} he found'; does OH regularly do this? and later?

\end{description}


\section{The Story of Zalpa (CTH 3.A)}

\begin{description}

\item[Publication :] 
\item[Edition :] \citet{otten1973zalpa} (= StBoT 17).  \\ 
\citet{holland2007zalpa}
\item[Background :] \citet{holland2007zalpa}: \\

``The \textit{Tale of Zalpa} (CTH 3.1) is a fragmentarily preserved Hittite text that, in the form in which it bas been transmitted to us, falls into two disjunct parts. First is the story of thirty brothers born at the same time to the queen of Kane\v{s}. These brothers were placed in baskets and set adrift on a river which carried them to the sea in the land of Zalpa. There the gods took them up and reared them. When they grew up they went back to Kanes in search of their mother and found her, but at the same time they met and presumably married their thirty sisters, who during their absence also bad been born to their mother at one time. Only the youngest brother recognizes his sisters and objects to the potential incest. The text breaks here. With the resumption of the text the scene of the action is Zalpa. After a god's blessing the text continues with a series of apparently historical events---the outbreak and cessation of hostilities, the assassination of a princess, tabarna's reprisals against Zalpa, dynastic struggles, and the public rejection of demands from the king of Hattusa---which culminate in the destruction of the city Zalpa{\ldots}\citet[1]{otten1973zalpa} characterizes KBo 22.2 as a tablet written in old ductus, and consequently datable to the 16th or the 15th century BCE. Otten further states that the sign forms are `graziler' than those on the Zukra\v{s}i tablet, the sole Hittite tablet in old ductus with a securely datable archaeological provenience. Since elsewhere Otten makes a consistent distinction between `alter Duktus' and `typisch alter Duktus', it is clear that in Otten's view this tablet does not belong to the oldest group of Hittite tablets{\ldots}[Ultimately, after much debate and controversy -ADY] while KBo 22.2 does not appear to be a consistently written Old Hittite manuscript, as Otten thought, on the other hand it is not a consistent Middle Hittite manuscript either, but rather has a mixture of early and somewhat later sign forms. The paleographic evidence provided by these sign shapes thus supports a later dating of KBo 22.2 than is ordinarily accepted.''

\smallskip

\item[A Obv.]

\smallskip

\item[1 :] \textbf{\city}\bit{kani\v{s}} `of Kanes/Nesa'; the equivalence of Nesa and Kanes was established on the basis of this text; cf. {\city}\textit{N\=e\v{s}a} in ln. 7

\item[2 :] \bit{k\I=wa kuit} a rhetorical question; on IE parallels for this structure, see \citet{hackstein2004kuit} \bit{walkuan} \p{n. acc. s.} `bad/evil omen; monstrosity, freak'; semantic range similar to Lat. \textit{monstrum} etc.; etym. to be taken with the `wolf' word \textit{*w\textsubring{l}{\labk}o-} but with different morphology (cf. Luw. \textit{walwa/i-} `lion') \bit{tuppus} \p{c. acc. pl.; a-st.} `basket; chest; pot'; to be distinguished from more frequent, formally overlapping \textit{tuppi-} `tablet'  \textbf{sakanda} \p{n. instr.} `oil'; on the archaic, unproductive instrumental ending in \textit{-t/d(a)}, see \citetalias{GrHL} \S3.35; the recognition of this ending by \citet{hoffner1995oil} has allowed us to dispense with the older reading (e.g. Otten) `shit', which caused obvious problems for interpretation \bit{\v{S}I-MA} \p{pron. fem. nom. s.} `she'; grammatically correct Akkadian forms are not the norm in Hittite, but are pervasive in this text
 
\item[3 :] \bit{andan} \p{adv.} `within, in(side)' \bit{zik\=et} \p{3s. pret. act.} iter. \textit{d\=ai} `place', here in its distributive function; later formally renewed as \textit{ziki\v{s}k-} \bit{\v{su}-} \p{conj.} `and'; OH-only clausal conjunction, typically `referring to sequential actions' (cf. \citetalias{GrHL} \S29.12) and restricted to the preterite (in complementary distribution with \textit{ta-}, which is used with pres.-fut. and impv.; cf. \citetalias{GrHL} \S29.3)

\item[4 :] \textbf{A.AB.BA} \p{c. d-l. s.} `sea' = \textit{aruna-} \textbf{\city}\bit{zalpuwa} 
\p{adj. (?) c. all. s.} possibly in its original function here as possessive adj. formed to the toponym Zalpa

\item[6 :] \bit{i\v{s}tarna p\=air} `(years) passed within / in between / meanwhile'

\item[7 :] \bit{apa\v{s}ila} \p{c. nom. s.} `she herself'; only in the nom., strengthened by a particle \textit{-ila} (so also, \textit{zikila, ukila}, cf. Lat. \textit{eg\=o-met, n\=os-met}; see \citetalias{GrHL} \S5.3) \bit{ianzi} \p{3pl. pres. act.} `go'; Otten's attempt to derive from \textit{iya-} `do, make' (``machen [sich auf den Weg]'') cannot stand, nor can it be taken from \textit{iya-} `go along; march', since (i.) it is MP only and (ii.) it never takes a goal; this form must be an archaism, the single clear instance of an uncompounded indicative form of `go' ($<$ \textit{*{\hi}ei-}); elsewhere in Hitt., possibly only a slightly remade \textit{(y)ezzi} `he goes' in a NH copy, though significantly, the uncompounded root is regularly found in Luwian (hence PA)

\item[8 :] \textbf{AN\v{S}E}-\bit{in} \p{c. acc. s.} `donkey; ass'; here probably collectively, viz. `a herd of\ldots' \bit{nannianzi} \p{3pl. pres. act.} `lead; drive; march'  ; iter. to \textit{nai-} in \textit{-anna-}, \underline{not} reduplicated! \bit{tar\v{s}ikanzi} \p{3pl. pres. act.} `speak'; iter. to \textit{tar-} `speak', which recall is suppletive plural to \textit{t\=e-} `speak', i.e. (cf. \citetalias{GrHL} \S12.48); archaic spelling here, later \textit{tar-a\v{s}-kV-}: \\

\begin{tabular}{|l|l|l|l|l|} \hline
{} & \textbf{Present} & {} & \textbf{Preterite} & {} \\ \hline
\textbf{1} & \textit{t\=emi} & \textit{tarweni} & \textit{t\=enun} & {---} \\
\textbf{2} & \textit{te\v{s}i, t\=e\v{s}i} & \textit{tarteni, t\=eteni} & \textit{te\v{s}} & {---} \\
\textbf{3} & \textit{tezzi} & \textit{taranzi} & \textit{t\=et} & \textit{terer} \\ \hline
\end{tabular} \\

%Possibly connect to Lith. \textit{ta\~rti} `id.' and (fairly marginal; redup. aor. in HHerm., a few other forms, pres. only Hsch.) Gk. {\greektext tetor'hsw} `ring, resound; pronounce shrilly, shriek'
\item[9 :] \bit{k\=ani} \p{adv.} `here' \bit{tunnakki\v{s}} \p{n. acc. s.; s/n-st.} `inner chamber' \bit{inutten} \p{2pl. pret. ind. (?) act.} `make warm'; formally, pres. impv. or pret. act; in either case, problematic for interpretation \bit{arkatta} \p{3s. pres. mp. (?)} `mount; copulate; cover'; the meaning of this word is disputed; CM (following CW) favor `copulate'

\item[10 :] \bit{kuwapit} `where / whither?' \p{interr. adv.}; see \citetalias{GrHL} \S27.14 \bit{arumen (!)} \p{1pl. pret. act.} the emendation to \textit{aumen} `we saw' here (as well as in ln. 11, below), proposed by \citet[185]{eichner1974chronik}, has long been accepted; though the sense of lnn. 9-10 are rather unclear, their function in the narrative is not: they serve as a somewhat clunky plot device allowing the sons the opportunity to talk about their miraculous birth

\item[12 :] \bit{nu=uz} the alternate spelling \textit{nu-uz-za} is found ony in OH \textbf{I}\bit{-\v{S}U} \p{adv.} `all at once; suddenly' \bit{h\=a\v{s}ta} \p{3s. pret. act.} `give birth (+ \textit{za})' ; note that already in OH, {\hith}i-verbs with stem-final \textit{-\v{s}} already show a 3\supersc{rd} sing. pret. in \textit{-ta} (cf. \citetalias{GrHL} \S11.6 n.11), which \textit{may} be the source of spread of \textit{-\v{s}ta} in this function (cf. \citetalias{GrHL}13.20)

\item[13 :] \bit{merir} \p{3s. pret. act.} `disappear; become lost'; MP inflection is restricted to late texts; CM accepts idea that meaning `die' associated with this root is an innovation of late PIE \bit{karti=\v{s}mi} \p{n. d-l. s.; c-st.} `their heart'; the 3pl. poss. encl. \textit{=\v{s}mi} is formally identical to 2pl. in all cases; \bit{\`U} \p{conj.} `and; but'; the Akkadogram can also function as Hitt. \textit{=(m)a}

\item[14 :] \bit{UMMA=NI=\v{s}an} \p{c. acc. s.} `our mother' = Hitt. \textit{anna=\v{s}umma/in=\v{s}an}, another example of perfectly grammatical Akkadina; the Akkadogram must be distinguished from \textit{UMMA} = \textit{ki\v{s}\v{s}an} `thus, the following' found throughout this passage; \textit{NI} is the Akk. 1pl. poss. suffix, for which we'll recall the paradigm (cf. \citet[78]{vandenhout2011elements}: \\

\begin{tabular}{|ll|ll|} \hline
-C-\textit{YA} / -V-\textit{I} & `mine' & \textit{-NI} & `our' \\
\textit{-KA} & `your' (s.) & \textit{-KUNU} & `your' (pl.) \\
\textit{-\v{S}U / -ZU} & `his' & \textit{-\v{S}UNU} & `their' (m.) \\
\textit{-\v{S}A} & `her' & \textit{-\v{S}INA} & `their' (f.) \\ \hline
\end{tabular}

\textbf{wemiyawen} \p{1pl. pret. act.} `find (+ \textit{=san})'; nice tense/aspect contrast here between preterite and marked imperfective \textit{\v{s}an{\hith}i\v{s}kiweni}

\item[16 :] \bit{kar\=atan} \p{c. acc. s.} `entrails, innards; inner being, character'; see AK \textit{s.v.} for a (pretty reasonable seeming) discussion of semantics, including in this passage: ``On the basis of [this] context, it has been claimed that \textit{kar\=at-} should mean ``\"Ausseres, H\"ulle'' \citep[thus][139]{rieken1999stammbildung}, but this seems unnecessary to me: compare Puhvel's translation ``the gods installed another character in them, and [\underline{better}: `so that / with the result that' -ADY] their mother does not recognize them''.'

\item[17 :] \bit{\v{S}A} `her'; a very rare correct usage of the fem. Akk. poss., typically always \textit{\v{S}U} in Hittite texts

\item[18 :] \bit{{\hith}antezziya\v{s}} \p{adj. c. nom. pl.(?)} `first, foremost'; the nom. pl. ending \textit{-e\v{s}} is expected; the use of \textit{-a\v{s}} in OH is very unlikely, leading some to argue that here we have a retention of the original PIE \textit{*-o-}stem nom. pl. generally lost in Hittite; that such an archaism is preserved in a complex \textit{*-tyo-} suffix seems doubtful though \bit{neku\v{s}=(\v{s})mu\v{s}} \p{c. acc. pl.} `their sisters'; orthographic single writing of -\v{s}- at clitic boundary \bit{appizziya\v{s}=a=\v{s}\v{s}an} `but the last', with contrastive non-geminating \textit{=a / ma}; the particle \textit{=\v{s}an} always appears with geminate \textit{=\v{s}\v{s}} after vowels, as is the case with all \textit{s-}initial enclitic pronouns and particles \citep[cf.][ \S11.4.5, with ref. to \citet{melchert1994ahp} for possible motivation]{kimball1999hhp}

\item[19 :] in the break before this clause, Otten understands `he said', but three problems: (i) absence of \textit{=wa}; (ii.) the use of \textit{=\v{s}an} then wholly unmotivated with \textit{verbum dicen{\I}}; (iii.) probably not enough space on tablet \bit{da\v{s}k\=eweni} \p{1pl. pres. act.} `take'; possibly \textit{-\v{s}ke-} in distributive function here (cf. \citetalias{GrHL} \S24.16) \bit{\v{s}aliktumari} \p{2pl. pres. mp.} `touch; violate; have illicit sexual intercourse; reach'; clearly, here, the incest taboo is at stake

\item[20 :] \bit{\=ara} \p{adv.} `(it is) [not] right/proper'; never occurs without \textit{natta} `NEG'

\smallskip

\item[A Rev.]

\smallskip

\item[5{\pr} :] \bit{{\hith}engani} \p{n. d-l. s.; n-st.} `death, doom; disease, plague'

\item[6{\pr} :] \bit{e=a} apparently interpreted by GH as a parenthetical question ``Are they not?'' (\textit{dubit\=o}) with no further discussion

\item[7{\pr} :] \bit{\=ar\v{s}a} \p{3s. pret. act} `arrive, reach'; more typically \textit{(a-)ar-a\v{s}}, but alternate spelling here nicely points to phonetic [\'a:rs]

\item[8{\pr} :] \bit{i\v{s}parza\v{s}ta} \p{3s. pret. act} `escape; flee'

\item[9{\pr} :] \bit{{\hith}usuwantan} \p{ptcpl. c. acc. s.} `living; alive'; following CW, an archaic zero-grade form here, elsewhere replaced by full-grade \textit{*{\hii}wes-went-} $>$ \textit{{\hith}ui\v{s}want-} with pre-coronal raising of \textit{*e} to [i] (cf. \textit{witt-} `year' $<$ \textit{*wet-s})

\item[10{\pr} :] \bit{ara{\hith}zanda wetet} \p{3s. pret. act} `build' + \p{adv.} `(all) around; outside of' = `besiege'; but this collocation is not the normal means to express this notion, rather either \textit{anda wa{\hith}nu-} or \textit{{\hith}atke\v{s}\v{s}nu-}, both common in NH

\item[11{\pr} :] \textbf{MU II\supersc{KAM}} `2 years'; the III found in dupl. can be taken as evidence for a visual rather than oral copy \bit{kattan} \p{adv.} `next to; beside'; effectively, `there; in that place'

\item[12{\pr} :] \bit{\v{s}=u\v{s}} \p{encl. pron. c. acc. pl.} `them'; an abrupt change of subjects (paralleled in B rev., where \textit{nu} instead of \textit{\v{s}u}) without any sort of adversative conjunction (i.e. \textit{=(m)a})

\item[13{\pr} :] \bit{aruwanzi} \p{inf.} `bow to; worship, revere; pray to'; note haplology, repaired in NH dupl.

\item[15{\pr} :] \textbf{LUGAL}\bit{-u\v{s}=(\v{s})mi\v{s}} \p{c. nom. s.} `your (pl.) king'; once again, single spelling -\v{s}- of [s=s] at clitic boundary  \bit{katti=\v{s}mi} `with you; beside you', viz. in addition to the citizens, also the troops

\item[16{\pr} :] \textbf{TIL.LA} `complete'

\smallskip

\item[B Obv.]

\smallskip

\item[3{\pr} :] \textbf{NINDA.KUR\subsc{4}.RA} `thick bread'

\item[4{\pr} :] \bit{i\v{s}\v{s}a=\v{s}\v{s}a} \p{c. all. s.; s-st.} `his/her mouth'; the \underline{only} ablauting \textit{s-stem} in Hitt. (nom. s. \textit{ai\v{s}}); the allative poss. clitic is fairly rare (cf. \citetalias{GrHL} \S6.4 n.19)

\item[5{\pr} :] \bit{\v{s}=an} \p{c. (?) acc. s.}; the use of c. acc. encl. \textit{=an} to refer back to \textit{memal} \p{n.} is problematic; we expect rather \textit{=at} \bit{i\v{s}ta{\hith}ta} \p{3s. pret. act.} `swallow, gulp; taste, savor' 

\item[6{\pr} :] \bit{m{\I}yaru} \p{3s. pres. impv. mp.} `grow; thrive, prosper' ; a likely OH example of QSV (the `phraseological' construction) \index{quasi-serial verb construction} (GH \textit{approbante}), though clitic behavior cannot confirm uniclausality, so simple asyndeton is technically possible, but the verb of motion is otherwise semantically nonsensical; the presence the `phraseological' construction in OH is contrary to claims of \citet{vandenhout2010phraseological}, who is followed by \citet{koller2013restructuring} among others; note that CHD tentatively restores [ \textit{nu} \ldots in the break following \textit{Zalpuwa\v{s}} against \citet{holland2007zalpa}

\item[7{\pr} :] \bit{appiziyan} \p{adv.} `afterward'; formally, neut. acc. s. of \textit{appeziya-} `last', substantivized in adverbial function

\item[10{\pr} :] \textbf{{\man}\v{S}\`A.TAM} `chamberlain; steward' 

\item[11{\pr} :] \bit{U\v{S}MIT} \p{3s. pret. act.} `(has) killed'

\item[12{\pr} :] \bit{TU\v{S}MET} \p{2s. pret. act.} `you (have) killed'; another example which shows that the scribe knew Akkadian quite well, though some minor mistakes (cf. \textit{-E-} vocalism in verbal ending here with \textit{-I-} in preceding ln. 11{\pr}

\item[14{\pr} :] \bit{{\hith}ulanzanni=pat} \p{n. d-l. s.} `battle; fight'; this formation may be \textit{hapax} (\textit{*{\hith}ulanz\=adar}), typically \textit{*nt}-stem \textit{{\hith}ulanzant-} \textbf{BA.\'U\v{S}} \p{3s. pres. act.} `is dead; dies'

\item[15{\pr} :] \bit{e=\v{s}ta} \p{clitic pron. c. nom. pl.} `they' + local ptcl. {(a)\v{s}ta} (see \citetalias{GrHL} \S28.108-14) \textbf{I} \bit{\v{S}U\v{S}I} `1 \textit{\v{S}}-measure'

\item[16{\pr} :] \bit{uddana\v{s}=\v{s}a\v{s}} \textbf{EN}\bit{-a\v{s}} `the lord of his (!?) affair'; the ritual client is regularly designated by \textit{uddana\v{s}} EN\textit{-a\v{s}}, but the presence of the poss. clitic is unprecedented; recall, though, that such titles are Akk. calques; it is possible, then, that this scribe who clearly knew Akk. was a bit too literal in rendering it, or that the expression had not yet been fixed at this stage in Hitt.

\item[17{\pr} :] \bit{a\v{s}a\v{s}ta} `settle, colonize; cause to settle' ; here in the caus. sense, it seems; interesting that their is a caus. \textit{a\v{s}e\v{s}anu-} to this often MP, though trans. verb; distribution of meanings?

\item[18{\pr} :] \bit{I\v{S}M\=UMA} `they heard'; striking Akk. usage here, elsewhere only fixed 3\supersc{rd} s. \textit{I\v{S}ME}

\item[23{\pr} :] \bit{watarna{\hith\hith}i\v{s}} \p{3s. pret. act} `order, command; notify, inform' \bit{kuit} \p{neut. acc. s.} `whatever' ; in initial-position, begins an indeterminate relative clause (cf. \citetalias{GrHL} \S30.59) \bit{{\hith}anti} \p{adv.} `separately'

\item[25{\pr} :] \bit{tupal\=an} \p{c. acc. s.; a-st.} `scribe (?)'; the meaning supposes a derivative of \textit{tuppi-} `tablet' with the regular \textit{-(a)la-} `profession' suffix; but note that \textit{tuppi-} is consistently spelled with geminate \textit{-pp-}

\item[28{\pr} :] \bit{\=epten} \p{2pl. pret. act.} `take, seize' + \textit{kurur} = `become hostile', functionally equivalent to \textit{ki\v{s}at}  

\item[29{\pr} :] \bit{s\=unizzi} \p{3s. pres. act.} `fills'; regularly a \textit{{\hith}i-verb}, but with a \textit{mi}-verb stem which normally takes the shape \textit{\v{s}unniya-}, though here lacking the \textit{-ya-} suffix \bit{ha\v{s}\v{s}a hanza\v{s}\v{s}a} \p{c. all. s.} `progeny of progeny; all future generation'; a fixed expression with both elements always inflected in same case; for attestation and analysis, see \citet{melchert1976hassa}; cf. P. \textit{s.v. {\hith}a\v{s}\v{s}a-}, who interprets the morphological-opaque second element as a ``hittitized Luwoid genitival adjective''

\item[30{\pr} :] \textbf{G\'IR}{\bit -anza} \p{n. erg. s.} `knife'; with CM, more likely ergative than instrumental \citep[\textit{pace}][]{otten1973zalpa, holland2007zalpa}

\item[32{\pr} :] \textbf{{\wood}TUKUL G\'ID.DA} `(lit.) long weapon; mace'
\item[33{\pr} :] \bit{p\=au} \p{3s. pres. impv. act.} `give'

\smallskip

\item[B Rev.]

\smallskip

\item[16{\pr} :] \bit{talit} \p{1s. (?) pres. impv. act.} `take'; Otten's interpretation as \textit{3rd s.} of \textit{dala-} `leave, let' makes no sense in clause with 1st person subject, nor can we accept the suggestion of \citet{holland2007zalpa} as instrumental in parallel with following clause; possible here a very rare 1s. impv. form to \textit{d\=a-} `take' (cf. \textit{\=e\v{s}lut, \=e\v{s}lit} `Let me be\ldots!'); this idea already in AK \textit{s.v. da-} \textit{ug} \bit{{\wood}intaluzzit} \p{c./n. instr.} `shovel'

\item[18{\pr} :] \bit{IKKIR} \p{3s. pret. act.} `he became hostile'

%\item[22{\pr} :] Another very odd construction with similar elements to those in \textit{A rev. 6{\pr}}, with bizarre fronting of `soldiers', in this case preceding the conjunction \textit{\v{s}u}, which is immediately followed by \textit{kuit}, though this doesn't seem like an indefinite relative clause and so we should ask why we don't need an accented element before \textit{kuit}{\ldots}\underline{ask CM}

\item[23{\pr} :] \bit{\=ara\v{s}} \p{3s. pret. act.} `arrive'; more typical spelling \textit{a-ar-a\v{s}} here (cf. n. \textit{ad} A rev. 7{\pr})

\item[28{\pr} :] \bit{tabarna\v{s}} read -\textit{an} \p{acc. s.}; discussion of emendation in GH \textit{ad loc.}

\item[29{\pr} :] \bit{\v{s}={\ldots}} Otten \citep[followed by][]{holland2007zalpa} emends to \textit{\v{s}e a-kir} = \textit{\v{s}=e akir} `(and) they died', assuming copyist error (\textit{rect\=e} cf. A rev. 13{\pr}); this line extremely problematic

\item[32{\pr} :] \bit{ELQE} `he took, seized'

\end{description}

\section{The Ritual for the King and the Queen (CTH 416)}

\begin{description}

\item[Publication:] KBo 30.33 + 17.1 + 25.3 +
\item[Edition:] \citet{otten1969king} (=StBoT 8) \\
\citet{neu1980ritual} (=StBoT 25) \\
See also \citet{neu1983glossary} (=StBoT 26)
\item[Background:] [add]

%%%%%

\item[\underline{Nr.2 Vs. I}]
%%%

\item[1 :] \bit{kalulupi=\v{s}mi} \p{c. d-l. s.} `digit (i.e. finger \textit{or} toe)' (\v{S}U.SI) 

\item[6{\pr} :] \bit{i\v{s}\v{s}\=a=\v{s}ma} \p{n. all. s.} `mouth'; plene writing here marks accentual shift in ablauting paradigm

\item[10{\pr} :] \bit{\v{s}{\I}nann=a} \p{c. acc. s.; a-st.} `figure, form; image' \bit{wiln\=as} \p{n. gen. s.} `clay'; nom. \textit{wilan}, later \textit{wilana-} \p{c.}

%%%

\item[\underline{Vs. I}]
%%%

\item[2{\pr} :] \bit{allappahhanzi} \p{3pl.pres. act.} `spit on; vomit on'; `spit onto' with acc. obj., since at Vs I. ln. 16{\pr}, the verb \textit{la{\hith}ui} `pours' describes the king spitting water from his mouth into a vessel

\item[3{\pr} :] \bit{{\hith}uyanzi} \p{3pl. pres. act.} `run'; according to CM, \underline{not} as per Otten a rare example of acc. of direction (with LUGAL\textit{-un} SAL.LUGAL\textit{-ann=a}), rather trans. `they run (around)' \textbf{III}\bit{-ki\v{s}=a=sma\v{s}} \p{adv.} `thrice'; the multiplicative suffix is optionally \textit{-ki\v{s}/-i\v{s}}, for which nicely cf. Gk. {\greektext poll'aki / poll'akis} (cf. \citetalias{GrHL} \S9.54); other multiplicative suffixes are \textit{-anki} and Akk. \textit{-\v{S}U}

\item[5{\pr} :] \textbf{SAL.LUGAL}\bit{-a\v{s}\v{s}=a=an} Ott. flags the singular clitic pron. \textit{=an}, which must have  multiple referents `cow' and `figure'; CM observes variable agreement patterns throughout this ritual---here, agreement with the nearer

\item[6{\pr} :] \bit{partaunit=u\v{s}} \p{n. instr.} `wing, feather'; in this isolated case, a (probably deliberately suppressed) example of clitic doubling in OH; though without right-extraposition, the clitic precedes its nominal referent, which is added seemingly as a means of disambiguation---note that it has been three times since the king and queen have been mentioned together in the discourse  \bit{a\v{s}a\v{s}kizzi} \p{3s. pres. act.} `seat'; \textit{*-ske-} semantics are difficult; stands beside a normal reduplicated present \textit{a\v{s}\=a\v{s}-} `seat'

\item[7{\pr} :] \bit{{\hith}urtiyallan} \p{c. acc. s.} `vessel, pitcher'; item of realia only in this ritual

\item[8{\pr} :] \textbf{AN.BAR}\bit{-a\v{s} n\=epi\v{s}} `sky of iron'; the nature of this (\textit{hapax}) item is obscure 
\bit{kitta} \p{3s. pres. mp.} `lies; is placed'; recall that this verb also functions as suppletive passive to \textit{dai-} `place'

\item[9{\pr} :] \bit{tarma\v{s}=\v{s}an} \p{c. nom. s.; a-st.} `peg, nail'; the segmentation is ambiguous, though, and in OH this could equally be \textit{tarma\v{s}\v{s}=a=(a)n}, with the OH local particle \textit{=an} `in' frequent in this ritual (where it was first identified); I wonder whether it is precisely such ambiguous contexts which lead to the demise of \textit{=an} 

\item[11{\pr} :] \bit{ka\v{s}ata} \p{ptcl.} treated as an extended variant of \textit{ka\v{s}a} by \citet{hoffner1968review}, who argues that its use with preterite forces a present-perfect intepretation `have (just) X-ed' (cf. \citetalias{GrHL} \S24.27-30); for  \bit{l\=alu\v{s}} \p{c. acc. pl.} `tongue; (malicious) gossip'

\item[12{\pr} :] \bit{irma(n)=\v{s}ma\v{s}} \p{c. acc. pl.} `sickness, illness'; clitic boundary assimilation of \textit{*n=s} $>$ [-\v{s}=\v{s}-], with simplified spelling of geminate; \bit{kardi=\v{s}mi=ya=at=kan} \p{c. d-l. s.} `heart'; this can only mean `from their heart'; thus already in OH it was possible to use the dat.-loc. to mark separation in place of expected ablative

\item[13{\pr} :] \bit{{\hith}ar\v{s}ani=\v{s}mi} \p{n. d-l. s.; r/n-st.} `head'

\item[15{\pr} :] \textbf{LUGAL}\bit{-i} \textbf{SAL.LUGAL}\bit{-i=a} the discourse motivation for right extraposition here is hard to understand; generally, though, right extraposition is permissible in native OH, but not with `clitic doubling' \bit{\=arri} \p{3s. pres. act.} `wash' ; to be distinguished from \textit{ar-} `arrive' by geminate \textit{-rr-}

\item[17{\pr} :] \bit{n=at} the use of \textit{t=at} in the identical context in immediately preceding ln. 16{\pr} exemplifies the difficulty in distinguishing functionally between these two conjunctions

\item[18{\pr} :] \bit{i\v{s}\v{s}az=(\v{s})met} \p{n. abl.} `mouth'; the single affricate is allowed to stand orthographically for the sequence [{\texttslig}=s]

\item[19{\pr} :] \bit{kal\=ulupi=\v{s}mit} \p{c. instr.} `finger'; the pron. clitic can only be abl.-instr. or neut. acc., erroneously added to d-l. s.; \textit{recte} copyist at 3 I 13{\pr} \textit{ka-lul-lu-pi-iz-mi-da-a\v{s}-ta}, here with affricate \textit{-z-} spelling [t=s]; still, the instr. seems bizarre in this context: `he takes the things fastened with their fingers'  ({*}`with their fingers') \bit{i\v{s}karanta} \p{ptcpl. n. acc. pl.; nt-st.} `prick; stick fasten' (\textit{iskar}- )

\item[20{\pr} :] \bit{n=e=n} \p{encl. n. acc. pl.} pron. has as referent subst. neut. acc. ptcpl.; it is followed by the first attestation in this text of the very rare OH-only local particle \textit{=an} `within' ($<$ PIE \textit{*en}), on which cf. \citetalias{GrHL} \S28.96ff.; the initial vowel is elided after vowels

\item[21{\pr} :] \bit{ap\=u\v{s}} \p{c. acc. pl.} `those'; common gender here is unexpected; the referent must be in the first place `tongues' (common grammatical gender, inanimate natural gender) but probably also `(things) fastened' (neuter gramm., inanimate natural); the neut. acc. clitic pron. \textit{=e} shows a more typical agreement pattern, with reversion to neuter for mixed grammatical, but inanimate natural gender group \bit{{\hith}antezumni} \p{d-l. s.} `forecourt; entrance hall'

\item[22{\pr} :] \bit{paiwani} \p{1pl. pres. act.} `go'; the endings \textit{-wani} and \textit{-tani} are found already in OH beside \textit{-weni} and \textit{-teni}, the latter to be understood as the unaccented reflex of the former ($<$\textit{*-{\pr}weni, -{\pr}teni})

\item[23{\pr} :] \bit{{\hith}arkanzi=ma=an} another clear attestation of \textit{=an}; I don't understand how the plural subj. goes with singular verb though \bit{{\hith}ars\=arr=a} \p{n. acc. pl.} `head'; geminating \textit{=a} here introduces a `both{\ldots}and' structure; interestingly, though, the clitic occurs in this position rather than before the adnominal genitive \textit{antu{\hith}\v{s}a\v{s}}, as is typical; it seems that `heads of a man' (or `human heads') is treated as a unitary prosodic group \index{prosodic domain} \textbf{{\wood}\v{S}UKUR{\hpl}}\bit{ -ya} \p{acc. pl.} `spear, lance'

\item[24{\pr} :] \bit{i\v{s}{\hith}a\v{s}kanta} \p{adj. n. nom. pl.} `bloody'; the morphology is problematic; looks like a ptcpl. to a \textit{-ske-} verb, but the base must ultimately be nominal \textit{es{\hith}ar-} `blood'; unclear how these two facts can be reconciled

\item[25{\pr} :] \textbf{T\'UG{\hpl}}\bit{ -u\v{s}} \p{c. acc. pl.} `clothing; robe' \bit{putaliyante\v{s}} \p{ptcpl. c. nom. pl.} `fasten; engird, wrap'; denom. to \supersc{GADA}\textit{putalli-} `(valuable cloth); sash'; the ritual participants here are wearing belts into which their robes can be tucked

\item[26{\pr} :] \bit{{\hith}alina\v{s}} \p{gen. s.} `clay' \bit{tarlipit} \p{instr.} T. \textit{s.v.} `red liquid (?)'; realia only in this ritual, where it is poured; T. assumes `red', in view of ln. 27{\pr} \textit{tarweni=ma=at e\v{s}{\hith}ar} `we call it blood' \bit{\v{s}uwamu\v{s}} \p{adj. c. acc. pl.} `full'; on this ending in \textit{u-}stem adj., cf. \citetalias{GrHL} \S4.43; one of only a few cases where it does not occur with genitive of contents

\item[27{\pr} :] \bit{petumini} \p{1pl. pres. act.} `bring' ; for the dissimilatory \textit{-m-} in \textit{1pl.} of this verb, see \citetalias{GrHL} \S1.126 (cf. \S1.86)

\item[28{\pr} :] \textbf{LUGAL}\bit{-i} \p{c. d-l. s.} `king'; dative nominal (in addition to clitic) can signal possession

\item[29{\pr} :] \bit{p\=ai} \p{3s. pres. act.} `give'; exhibits permissible gapping of the nominal obj. in second clause

\item[30 :] \bit{\=e\v{s}zi} \p{3s. pres. act.} `sit'; the presence of \textit{=\v{s}an} rules out homophonous \textit{\=e\v{s}zi} `is'

\item[31{\pr} :] \textbf{ZABAR} `bronze' \bit{apatt=a=(a)n} yet another attestation of \textit{=an}; this clause to be understood as a kind of parenthetical

\item[32{\pr} :] \textbf{ER\'IN{\mpl}}\bit{-n=a=(a)n} \p{c. acc. s.} `troop'; another \textit{=an}; cf. ln. 33{\pr}, below

\item[33{\pr} :] \bit{p\=era\v{s}=\v{s}et} `above him'; note assimilation of \textit{*/-n=s-/} $>$ [{\it-s=s-}]; the referent of \textit{=\v{s}et} is \textit{kui\v{s}}; the \textit{-e-} vocalism in in attached pronominal clitic is unexpected with \textit{p\=eran}, which does not (CM, \textit{pace aliis}) go back to an old root noun (cf. Vs. II ln. 30 \textit{s\=er=\v{s}met} and n. \textit{ad loc.}); significantly, note \textit{=\v{s}it} in dupl. \textbf{\wood}\bit{zup\=ari} \p{n. acc. s.; i-st.} `torch' \bit{appananda} \p{adv.} `behind'

\item[35{\pr} :] \bit{\v{s}\=aw\=atara\v{s}} \p{c. nom. s.} `hornist'; possibly still a free-standing genitive `(the man) of the horn' (cf. nom. s. \textit{\v{s}\=aw\=atar} \p{n.; r-st.} `horn'), since there is no evidence for hypostasis (e.g. acc. \textit{*saw\=ataran}); the double plene writing is likely an error, note \textit{s\=awatar-} in ln. 37{\pr} below

\item[38{\pr} :] \bit{t=}\textbf{[}\bit{a\v{s}ta} cf. \textit{n=a\v{s}ta} in ln. 20{\pr} above; again, \textit{nu} and \textit{ta} seem functionally indistinguishable 

%%%

\item[\underline{Vs. II}]

%%%

\item[2 :] \bit{{\hith}ariemi} \p{1s. pres. act.} `bury'

\item[7 :] \bit{{\hith}appini=ma=n} \p{d-l. s.} `(open) flame'; another ptcl. \textit{=an}

\item[9 :] \bit{tuwaattu} \p{?} `mercy! take pity!'; a frozen expression of unclear origin; regularly found in ritual contexts with vocatives addressing the gods, hence almost certainly some kind of supplication; it also frequently appears as \textit{duddu}, which can be taken as evidence for syncope; some (e.g. AK) interpret \textit{hapax} \textit{tuttut} as a continuant of the (uncommon) 2\supersc{nd} s. impv. ending \textit{-t} ($<$ \textit{*-dhi}) of the \textit{mi-}conjugation; this ending generally confined to \textit{\=it} `go!', \textit{uwatet} `bring!' and causatives in \textit{-nu-} (cf. \citetalias{GrHL} \S11.6 n.19)

\item[11 :] \bit{papr\=atar=\v{s}met} \p{n. acc. s.; r/n-st.} `impurity'; apposition here, i.e. `I have taken your impurity, viz. the bloody, terrible tongues of iron'; alternatively, simple conjunction \bit{{\hith}atugau\v{s}} \p{adj. c. acc. pl.} `terrible; fearful'

\item[12 :] \bit{\v{s}ume\v{s}\v{s}=(a)=u\v{s}} \p{pron. 2pl. nom.} `you'; clitic pronoun \textit{=us} is presumably object of impv. lost in the break

\item[15 :] \bit{nuwa} \p{adv.} `still' \bit{aniyemi} \p{1s. pres. act.} `ritually treat'

\item[16 :] \bit{pai=mu} \p{2s. impv. act.} `give'; note that the imperative has a short vowel \textit{-a-} in OH in contrast with \textit{3s. ind.} \textit{p\=ai} `gives', which is phonologically expected contra \citet{kimball1998origin, kimball1999hhp} (an idea originally of W. Cowgill, rejected by CM and Kimball; now CM recants, but not yet in print); this distinction later lost, impv. begins to appear with long vowel; following \citet{bauer2011hittite}, CM attributes this apparent case of imperative fronting \index{imperative fronting} to Hattic influence

\item[17 :] \bit{kuit} \p{n. acc. s.} `what'; clause-initial \textit{kuit} produces indefinite interpretation `whatever'

\item[23 :] \textbf{NINDA.KUR\subsc{4}.RA}\bit{-us} \p{c. acc. pl.} `thick loaf/bread' \bit{EM.\textsubdot{S}\'U.TIM} `sour' \bit{{\hith}upparan} \p{c. acc. s.} `bowl, cup' \bit{ye[m](i)} \p{1s. pres. act.} `do, make'; the spelling indicates [ye-], the OH reflex which emerges via analogy from weak (??), since initial yod generally lost in PA ; cf. \textit{iyami} restored from dupl. in lnn. 20, 22; the spelling \textit{iya-} generalized in NH is only orthographic, not evidence for phonological change

\item[25 :] \bit{mi\v{s}tiliya} `?'; meaning unknown, only as specifier of `time'

\item[28 :] \bit{{\hith}alma\v{s}\v{s}uitti} \p{c. d-l. s.; t-st.} `dais'; a famous Hattic loanword

\item[30 :] \bit{\v{s}\=er=s\v{s}met} `above them'; for CM, only \textit{s\=er} (and not other adpositions., \textit{p\=eran} etc.) reflects a relexicalized old root noun; it alone takes the split possessive construction, and shows consistent \textit{-e-} vocalism in attached pronominal clitic (cf. Vs. I ln 33{\pr} and n. \textit{ad loc.})

\item[31 :] \bit{wa{\hith}nuzi} \p{3s. pres. act.} `cause to turn/revolve; wave' (viz. `away the evils'); recall that spelling of infix with -\textit{\'u}- is grounds for AK positing \textit{*-eu-} $>$ Hitt. [u]

\item[36 :] \bit{wal{\hith}anzi} \p{3pl. pres. act.} `they strike'; viz. `they strike the spears together; they clang them'

\item[39 :] \bit{ap\=e=a} \p{neut. acc. pl.} `those'; if the restoration is correct (though note Otten's doubts in n.2), there are two oddities; first, we would expect topicalization of the pronoun with the particle, but it appears to the right of \textit{{\hith}alma\v{s}\v{s}uitt=a=z {\hith}a\v{s}\v{s}[az]}; second, we would expect the particle to be realized as \textit{=ma} or \textit{=ya} after the vowel; it seems very likely that this is an error of some kind, since the the co-presence of \textit{=a} and \textit{=ma}/\textit{=ya} does not make sense (??)

\item[42 :] \bit{kitkar} \p{adv.} `at the head'; for \citet{nussbaum1986hh}, a univerbation of `head' and a formant to the deictic stem \textit{*{\palk}e}- \bit{{\hith}ilamni} \p{c. d-l. s.} `gatehouse'

\item[43 :] \bit{\v{s}a\v{s}uwanzi} \p{inf.} `sleep'

\item[46 :] \textbf{III}\bit{-\v{s}u} \p{adv.} `thrice'; right-extra position is not expected here, but seems to be the normal treatment for this scribe, cf. ln. 50 below

\item[49 :] \bit{=\v{s}\v{s}an} \p{ptcl.}; the local particle appears after the conjoined group X Y-\textit{ya}; some treat this as a (rare, but not unprecedented) example of a clause-internal \textit{=\v{s}an} attaching to a dat.-loc. s. nominal, whence comparison with Lat. \textit{vobiscum}; however, the antiquity of this formation in Latin is now seriously doubted, so an explanation by PIE inheritance should be doubted; this problem may be more elegantly understood (CM \textit{approbante}) as a case of prosodic domain formation \index{prosodic domain} conjoined pair, with consequent blocking of `Wackernagel's Law'

\item[54 :] \textbf{\supersc{D}UTU}\bit{-i} \p{c. d-l. s.} `Sun-god', with postposition \textit{mena{\hith\hith}anda} in the sense `facing'

\item[56 :] \bit{X-riyala\v{s}=(\v{s})mi\v{s}} `?'; though we have nearly the entire word, unclear what to restore here; CM suggests \textit{[au]riyala\v{s}} `messenger', hence a nominal sentence: `The eagle is (your) messenger.'; however, there does not appear to be enough room on the tablet to fit this sign \bit{kullupi} \p{n. acc. s. ?} `sickle' \textbf{{\stone}AR\`A}\bit{-an} \p{c. acc. s.} `millstone'; paired with \textit{kullupi}, an example of male and female imagery

\item[57 :] \bit{sum\=a\v{s}} \p{pron. 2pl. dat.} `you' \bit{ukt\=uri} \p{adv.} `always, forever; continuously' \bit{{\hith}ar\v{s}in} \p{adj. c. acc. s.} `(+ \supersc{NINDA}) leavened (loaf)'

%%%
\item[\underline{Rs. III}]
%%%

\item[1 :] \bit{ma{\hith\hith}anda} \p{conj.} `(just) as'; OH only, a univerbation of \textit{man} + \textit{{\hith}anda} occasionally attested without assimilation; the second element usually taken as \textit{*{\hii}ent-} `forehead', but unclear how the semantics work

\item[2 :] \bit{ukt\=uri} \p{n. nom. pl. / adv.} `eternal'; could adverbial or show agreement with nearer (`heaven and `earth', both neuter); the neut. pl. would otherwise be unexpected in referring to a group that includes  entities which are both grammatically and naturally animate; the dupl. in 3 II 15{\pr} actually has - ] \textit{re-e\v{s}}, i.e. the c. nom. pl. ending

\item[3 :] \bit{namma} \p{adv.} `then; again'; typically in this position `again', but more likely here `(and) then'

\item[4 :] \bit{\=appanda=ma=\v{s}\v{s}e} \p{adv.} `behind (it)', viz. the bird as it flies away \bit{natta=an uk} `It was not \textit{I} who released it\ldots'; the unmarked word order would have the negator immediately before the verb

\bit{tarna\v{s}} \p{3s. pret. act.} `release'; some (incl. CW) have taken the use of the singular here as a special property of the king and the queen as the unit (cf. Otten trans. `K\"onigspaar'); this is very unlikely in view of the parallel passage at Rs. III ln. 17-18 with plural agreement; for CM, much more probable that this is one of several examples in this ritual of agreement with the nearer

\item[9 :] \bit{tarmaemi} \p{1s. pres. act.} `transfix; fasten'; denom. \textit{tarma-} \p{c.} `nail, pin; bolt'

\item[11 :] \bit{\=erna\v{s}=\v{s}met} \p{n. acc. s.; n-st.} `illness'; low-level assimilation at clitic boundary of /\textit{n=s}/ $>$ [s=s]

\item[13 :] \bit{p\=eta} \p{2s. (!) impv. act.} `bring'; the subject must be {\man}K\'UR-\textit{na\v{s}} and object pron. clitic \textit{=at}, so a 2\supersc{nd} s. impv. is nonsensical; on the basis of Rs. IV ln. 4 {\men}K\'UR{\ldots}petantu in an identical context, it is tempting to emend the text to a 3\supersc{rd} s. impv. \textit{-tu}

\item[14 :] \bit{par\v{s}{\hith}a} \p{1s. pret. act.} `break'; OH shows only MP forms of this verb (3s. \textit{par\v{s}iya(ri)}; 3pl. \textit{par\v{s}anda}; the \textit{-i-} in the 3rd. s. stem is unexplained; in accordance with general pattern, the trans. MP verb is later `activated'; thus in MH/NH, we find an active \textit{mi-}verb \textit{par\v{s}iya-}; for \citet[70 n.138]{jasanoff2003verb}, this stem is a renewal of an old MP almost exactly parallel to Ved. \textit{\'a\'saya[t]} `lay'

\item[20 :] \bit{papr\=atar} \p{n. acc. s.} `impurity; squalidness'

\item[21 :] \bit{zuw\=aluwal} `?'; realia only in this ritual  \bit{z\=eante\v{s}} \p{ptcpl. c. nom. pl.} `cook' $<$ zeya- (MP) \textbf{S\'IG ZA.G`IN}\bit{-it} \p{instr.} `wool' + `blue''

\item[22 :] \bit{{\hith}ul\=aliyami} \p{1s. pres. act.} `wrap up/around'; denom. {\cloth}\textit{{\hith}ulali}- \p{n.} `wrapping; bandage'

\item[23 :] \bit{kunkumati}\textbf{\supersc{SAR}} \p{n. acc. s.} `(a plant)'; \supersc{SAR} is a logogram which typically follows, like \supersc{MU\v{S}EN}

\item[24 :] \textbf{NUMUN}\bit{-an} \p{n. nom. s.} `seed; progeny' = \textit{warwalan-} \bit{pattarr=a} \p{n. acc. s.; r/n-st.} `(braided) basket'; in the absence of determinative {\wood}, indistinguishable from homophonous \textit{pattar} `feather'

\item[25 :] \textbf{\wood}\bit{zupari} \p{n. nom. s.} `torch'; typically spelled with [-pp-]

\item[26 :] \bit{garauni\v{s}i} \p{n. d-l. s.; r/n-st.} `horn'; nom. s. \textit{karawar} \bit{muriyale\v{s}} \p{c. nom. pl.; i-st.} `clusters/bunches of grapes'; a connection with Gk. {\greektext m\=ur'ios} `countless; 10,000', though see objections of P. 

\item[28 :] \bit{i\v{s}garandan} \p{ptcpl. acc. s.} `pierce; stick, fasten'  (\textit{i\v{s}kar}-  `prick; transfix')

\item[34{\pr} :] \bit{muriyala\v{s}} `grape(-shaped) bread'; meaning on the basis of {\bread}\textit{muriyala-} elsewhere

\item[42{\pr} :] \bit{t=u\v{s}=(a)\v{s}ta}, with normal orthographically omission of -\v{s}- in cluster

\item[45{\pr} :] \bit{tum\=eni} \p{1pl. pres. act.} `take'; nice archaism here, with plene suffix; already in OH (OS) subject to replacement by analogic \textit{d\=aweni} (e.g. StBoT 25 \#137 ii 16)

\item[45{\pr} :] \bit{p\=eraz=(\v{s})mit} `above them' for \textit{/p\=eran=s\v{s}met/}; CM suggests a kind of orthographic hypercorrection for expected [\textit{p\=era\v{s}=\v{s}met}] with /\textit{n=s}/ $>$ [\textit{s-s}] assimilation  \bit{ur\=ani} \p{3s. pres. act.} `burn \p{intrans.}' $<$ \textit{ur-} (MP); remote dissimilation from \textit{*ur\=ari}, cf. \citetalias{GrHL} \S1.128

\item[47{\pr} :] \bit{i\v{s}{\hith}anda} \p{n. instr.; r/n-st.} `blood'; note archaic instrumental ending \textit{-t(a)}

\item[48{\pr} :] \bit{par\v{s}wani} \p{1pl. pres. act.} `chase'; this verb must be \textit{par\v{s}-} `chase' rather than `break', since the latter is always MP in OH (cf. Rs. III ln. 14 and n. \textit{ad loc.})

%%%
\item[\underline{Rs. IV}]
%%%

\item[6 :] \bit{pi\v{s}n\=a\v{s}} \p{c. nom. s.} `man'; a nominal sentence: `There is a man.'; Otten's suggestion to read instead the logogram KA\v{S} `beer' is unlikely since phonetic complements of this size are extremely rare, and the plene writing of -\textit{n\=a}- would be unprecedented

\item[8 :] \bit{tarlipa\v{s}\v{s}an} \p{adj. c. acc. s.} `characterized by/of \textit{t-}'; CM suggests a possessive adj. in \textit{-a\v{s}\v{s}a-}, viz. `a \textit{tarlipiy-}an (cup of blood)'; \bit{umeni} \p{1pl. pres. act.} `see; behold'; archaic ablaut preserved, with zero-grade of root + e-grade suffix

\item[9 :] \bit{\v{s}agai\v{s}} \p{c. nom. s.} `sign; omen'

\item[13 :] \bit{pai\v{s}ga{\hith}at} \p{3s. pret. mp.} `go'; note that \textit{pai-} belongs to the small class of verbs which show mediopasive inflection in their \textit{*-ske-} derivatives; according to CM, this pattern is found in unaccusative verbs \textit{kinun=a} \p{adv.} `(but) now'; temporal adverb seems to require a pres.-perf. reading of verb

\item[14 :] \bit{ain wain} \p{c./n. acc. s.} `oh (and) woe' \bit{pittuliu\v{s}} \p{c. acc. pl.} `anxiety'; the etym. of Eng. \textit{anxiety} matches the Hittite sense as well, viz. physical constriction

\item[15 :] \bit{g\=apinan} \p{c. acc. s.} `thread, string, cord'

\item[19 :] \bit{araummi} `?'; of utterly unclear meaning and morphological structure  \bit{i\v{s}hiyanda} \p{ptcpl. n. nom. pl.} `bind'

\item[22 :] \textbf{GAD}\bit{-an} `linen, cloth' \bit{pe\v{s}\v{s}iyemi} `throw; shove'; univerbation of \textit{*pe-} preverb + \textit{*{\hi}s-ye-}; cf. Skt. \textit{\'asyati} `hurls' \bit{\v{s}u=u\v{s}} \p{conj.} `and'; this is the \textit{only} example of \textit{\v{s}u} + pres. in OH, which is otherwise in perfect complementary distribution with \textit{ta} (\textit{\v{s}u} + pret., \textit{ta} + pres.); cf. \citetalias{GrHL} \S29.3

\item[23 :] \bit{i\v{s}pantuzzi=ya} \p{n. acc. s.} `libation; offering'

\item[24 :] \textbf{{\man}\'U.{\Hith}\'UB}\bit{-za} `deaf (man)'

\item[25 :] \bit{t=u\v{s}=(\v{s})ta} \p{pron. encl. c. acc. pl.} `them'; common gender with mixed common ({\bread}\textit{{\hith}ar\v{s}ae\v{s}} + neut. (\textit{i\v{s}pantuzzi}) antecedents

\item[27 :] \bit{\v{S}A QATI=\v{S}UNU} `of their hands'; right-extraposed gen. of specification with `fingers'; effectively clarifies whether it \textit{kalulupa-} is `finger' or `toe' \bit{{\hith}a{\hith\hith}al} \p{n. acc. s.} `bush'

\item[28 :] \textbf{{\wood}}\bit{{\hith}arpa=ma} \p{neut. nom. pl.} `wood piles'; \textit{nominativus pendens}, viz. `As for the wood piles\ldots'; \textbf{G\`IR}\bit{-\v{s}i} \p{c. d-l. s.} `foot'; observe gapping in conjoined clause \textbf{I}\bit{-anta} `1'; the individuating suffix \textit{-ant-} counts collective sets

\item[29 :] \bit{d\=a} \p{2s. impv. s.} `take'; an example of imperative fronting \index{imperative fronting}, again according to CM because of Hattic influence

\item[32 :] \textbf{Z\'IZ{\hpl}}\bit{-s=a} \p{gen. s.} `spelt' \bit{{\hith}ar\v{s}\=arr=a} \p{neut. nom. pl.} `head'; geminating \textit{=a} conjoins \textit{{\hith}ulalian kuita} and the phrase `heads of grain and spelt'; Z\'IZ{\hpl}\textit{-\v{s}=a} must also show geminating \textit{=a}, which forms conjoined genitives with \textit{{\hith}alkiya\v{s}} \bit{apatt=a} \p{neut. acc. s.} `that'; unexpectedly, this must a case of agreement with \textit{kuit}, i.e. the farther

\item[33 :] \bit{m\=arka{\hith\hith}i} \p{3s. pres. act.} `divide, separate'

\item[35 :] \bit{i\v{s}tappuli=\v{set}} \p{n. nom. s.} `stopper' \bit{\v{s}uliya\v{s}} \p{c. gen. s.; ai-st.} `lead'; gen. of material

\item[36 :] \bit{{\hith}armi} \p{1s. pres. act.} `have; hold'; a pres.-prog. interpretation here, given the pretense that the ritual practitioner has been secretly concealing two bird, which he then releases, affrighting the king and queen

\item[37 :] \bit{alki\v{s}t\=an} \p{c. acc. s.} `branch'; a rare example of the acc. of direction\bit{waritanzi} \p{3pl. pres. act.} `shout; cry out'

\end{description}

\section{The Edict of Telepinu}

\begin{description}

\item[Publication:] [add]
\item[Edition:] \citet{hoffmann1984telepinu}
\item[Background:] According to the text, Telepinu was the first Hittite king to set down laws for succession. For a long time, this idea was taken to reflect historical reality; now, the claim is viewed much more cynically: it seems extremely unlikely that Telepinu was the inventor of primogeniture. Rather, the text is in large propagandistic, attributing to Telepinu singly this accomplishment, and thereby reinforcing his claim---and in turn, that of his descendants---to inherited power.

\smallskip

\item[2 :] \bit{=apa} \p{ptcl.} typically denotes physical contact between two object, or else marker of terminative aspect (cf. \citetalias{GrHL} \S28.100-7); in this latter function, CM suggests an  etymological connection with Lat. \textit{ob-}
 
\item[3 :] \bit{gaena\v{s}=\v{s}e\v{s}} \p{c. nom. pl.} `in-laws; relatives by marriage'; the encl. poss. ending in \textit{-e\v{s}} guarantees nom. pl.; however, we would expect then \supersc{X}\textit{gaene\v{s}}; one possibly variously suggested is that it is an archaic reflex of the correct inherited \textit{*o-}stem nom. pl. ending \textit{*-\=os} \bit{{\hith}a\v{s}\v{s}anna\v{s}=\v{s}a\v{s}} \p{n. gen. s.; r/n-st.} `birth; family'; the expression in this line parallels Eng. \textit{kith and kin} ($<$ \textit{*gen{\hiii}- \& *gen{\hi}}), with Hittite replacement of latter by \textit{{\hith}a\v{s}\v{s}}- `beget'; syntactically interesting, with geminating \textit{=a} marking only `kith' (IH interprets `in-laws; relatives by marriage'), where LU.ME\v{S} must be interpreted as a determinative, while it must stand for a noun in `kin' with dependent genitive

\item[4 :] \bit{taruppante\v{s}} \p{ptcpl. c. nom. pl.} `unify; assemble, gather'; likely not yet a periphrastic perfect in OH, which structure becomes productive later in the language

\item[5 :] \bit{la{\hith\hith}a=ma} \p{c. all. s.} `campaign; expedition, journey'; this sentence exemplifies the OH pattern, where \textit{=ma} is cliticized to the contrasted element; later reanalyzed as marking contrast/topic-switch on entire clause, whence regular \textit{m\=an=ma} etc.

\item[6 :] \bit{kuttanit} \p{n. instr.; r/n-st.} `part of upper body below head; strength'; for CM, the trapezius, hence its metaphorical usage `strength'; \bit{tara{\hith\hith}an} \p{sup.} `overcome; conquer'; in this line and following, read rather \textit{tar-u{\hith}-{\hith}a-an}, where the sequence \textit{-u{\hith}-{\hith}a-} spells the labialized pharyngeal fricative [h\supersc{w}]

\item[7 :] \bit{tarranut} \p{3s. pret. act.} `make strong; (+ \textit{ar{\hith}a}) make weak; unman'; caus. to \textit{tarra-} `be able' (MP); this use of \textit{ar{\hith}a}---effectively, \textit{un-}---is quite rare, occurring only with a handful of verbs

\item[8 :] \bit{=u\v{s}} \p{pron. c. acc. pl.} `them'; no common gender antecedent, either \textit{constructio ad sensum} or case-attraction with \textit{ir{\hith}a-}\p[c.], but the referent is certainly `lands' \bit{yet} \p{3s. pret. act.} `make'; lit. `made them into the borders of the sea' \bit{=ma} strictly speaking, \textit{=ma} is post-consonantal and so should not appear here (rather \textit{=a}); likely an error introduced by NH scribe \bit{wizzi} \p{3s. pres. act.} `come'; for CM, the shift into the present tense is for historical backgrounding

\item[11 :] \bit{maniya{\hith\hith}e\v{s}kir} \p{3pl. pret. act.} `hand over; allot; supervise, govern'

\item[12 :] \bit{tittiyante\v{s}} \p{ptcpl. c. nom. pl.} `place onto, apply; assign'; a redupl. form of \textit{*dhe{\hi}-} `place', clearly not (as sometimes interpreted) to \textit{*dhe{\hi}-} `suck'; note the overt dative object (viz. `assigned to') in parallel lnn. 19-20

\item[14 :] \bit{{\hith}a\v{s}\v{s}ana\v{s}=\v{s}i\v{s}=\v{s}i\v{s}=a} \p{c. gen. s.} `family; kin'; the gen. poss. clitic is properly \textit{=\v{s}a\v{s}}

\item[17 :] [\bit{ar{\hith}a}] has been omitted here; cf. parallel ln. 7

\item[21 :] \bit{mar\v{s}e\v{s}\v{s}ir} \p{3pl. pret. act.} `become false; disloyal'; fient. to \textit{marsa-} `false'; note the variant in dupl. C \textit{mar\v{s}\=er}, the denominative `stative' in \textit{-\=e-} which shows this formal category could also have fientive function, as noted already by \citet{watkins1971stative}

\item[22 :] \bit{karipuwan} \p{sup.} `devour, consume'; a rare non-\textit{*ske-} form in supine; could be taken as weak evidence for inherent atelic aspect of the verbal root \bit{i\v{s}{\hith}a\v{s}=a=\v{s}ma\v{s}=\v{s}an} \p{c. d-l. pl.} `their lords'; the clitic order here is aberrant, we would expect \supersc{X}\textit{i\v{s}{\hith}a\v{s}=(\v{s})ma\v{s}=a=\v{s}\v{s}an} \bit{ta\v{s}ta\v{s}e\v{s}kiwan} \p{sup.} `whisper; plot, conspire'; an onomatopoeic formation, with \textit{*-ske-} reinforcing iterative semantics

\item[23 :] \bit{\=e\v{s}\v{s}uwan} \p{sup.} `do, make'; iter. to \textit{iya-} `do', with fluctuation in initial vocalism; idiomatically, `commit bloodshed; shed blood'; cf ln. 34, below

\item[28 :] \textbf{NAM.RA{\mpl}} `deportee; settler' \textbf{\city}\bit{{\Hith}alpa\v{s}} `Aleppo'

\item[29 :] \textbf{{\city}KA.DINGIR.RA} `Babylon'; lit. `the gate of the gods'

\item[31 :] \bit{p\=e} \p{adv.} `to(ward)'; only free-standing in \textit{p\=e {\hith}ark-} `bring to; hand over to' \textbf{{\man}S\`ILA.\v{S}U.DU\subsc{8}} `cupbearer' = {\man}SAGI

\item[32 :] \textbf{NIN} `sister' = \textit{nega-}

\item[33 :] \bit{ule\v{s}ta} `enwrap, enfold; meld'; the sense of \textit{sara}) is unclear; the same verb appears in the Telepinu myth, describing the god's self-concealment

\item[35 :] \textbf{DINGIR{\mpl}}\bit{-an} \p{c. nom. (?). pl.} `god'; since an archaic genitive in \textit{=an} does not fit the sense, this could be the bare Akkadogram with the acc. pron. clitic \textit{=an}; however, the tablet is probably too broken to tell for certain; note that CM's hand copy of this section differs substantially from \citet{hoffmann1984telepinu}

\item[40 :] \bit{k\I iyanun kuit} a rhetorical question (`What (on earth) have I done?');

\item[43 :] \textbf{KA\subsc{5}.A{\hpl}}\bit{-u\v{s}} \p{c. acc. pl.} `fox' \bit{{\hith}a{\hith\hith}alla\v{s}} \p{d-l. pl.} `bush, shrub' + \textit{par{\hith}andu\v{s}} = `chased into the bushes' \bit{werir} `call; shout'; possibly `called X Y', but normally \textit{{\hith}alzai- + =za} in this construction

\item[45 :] \bit{we{\hith}attat} \p{3s. pret. mp.} `turn'; tablet too broken to determine the sense

\item[56 :] \bit{wiyat} \p{3s. pret. act.} `send'

\item[59 :] \bit{\v{s}an{\hith}ta} \p{3s. pret. act.} `(+\textit{appan}) inquire/ask after'

\item[60 :] \bit{{\hith}alukan} \p{c. acc. s.} `message, news, information'

\item[63 :] \textbf{\supersc{m}}\bit{{\Hith}ant{\I}li\v{s}\v{s}=a} \p{c. nom. s.} `Hantili';  the secure restoration on basis of NH dupl. consistently uses incorrect geminating \textit{=a} here and in what follows (e.g. ln. 66, 69) \textbf{{\man}\v{S}U.GI} `old man; elder'

\item[66 :] \bit{\v{s}an{\hith}ir} \p{3pl. pret. act.} `seek; (+ \textit{e\v{s}{\hith}ar}) seek revenge (against the killer) for the blood (of the victim)'; cf. \textit{e\v{s}{\hith}ar iya-} `kill'

\item[67 :] \textbf{{\man}K\'UR{\mpl}}-\bit{\v{S}U} `enemies'; the plural {\mpl} to be attributed to scribal error; dupl. B + C have correct singular {\man}K\'UR-\textit{\v{S}U}

\smallskip

\item[Vs. II]

\smallskip

\item[3 :] \textbf{ER\'IN{\mpl}}\bit{-u\v{s}} \p{c. acc. (?) pl.; t-st.} the acc. pl. ending must be an error of some sort, since it must be taken as subject of intrans. \textit{pai-} `go', and ER\'IN is a \textit{t-}stem

\item[4 :] \bit{w\=e\v{s}kanta} \p{3pl. pres. mp.} `come'; iter. \textit{uwa-} `come'

\item[6 :] \bit{piy\=et} \p{3s. pret. act.} `send (to/along)'

\item[7 :] \bit{=za} \p{ptcl.} the use of the refl. particle with \textit{kuenta} is unexpected

\item[8 :] \textbf{{\man}KA\v{S}\subsc{4}.E} `runner'; KA\v{S}\subsc{4} = \textit{{\hith}uwai-/huya-} `run'

\item[9 :] \bit{=[a]z} interesting writing of reflexive particle 

\item[10 :] \bit{{\hith}antezziyan} \p{adj. c. acc. s.} `first, foremost; eldest'; cf. \textit{apezziyan} `last; youngest' (e.g. Zalpa) \textbf{NIN}\bit{-ZU} `his sister'; phonological reflex of \textit{-\v{S}U} after coronals (acc. to T. \textit{s.v.}, after dentals and sibilants); the addition of $<$DAM$>$ is IH's emendation; CM suggests that NIN=ZU could be read DAM=ZU, since these signs were in some cases confused by Hitt. scribes

\item[11 :] \bit{i\v{s}duw\=ati} \p{3s. pret. mp.} `become known; become apparent' (MP); seems to function more like protasis of conditional (`[if] X had become known{\ldots}'), though preceding clause works is introduced by \textit{m\=an} (`X [would] have{\ldots}')

\item[13 :] \textbf{\v{S}E\v{S}{\mpl}}\bit{-\v{S}U} `his brothers'; topicalization, in this case appearing to the left even of clause-initial conjunction; note (i.) that the underlying case is dative; and (ii.) that this left-dislocation triggers obligatory clitic doubling \bit{tagga\v{s}ta} \p{3s. pret. act.} `put together; devise; undertake; allot, assign'; not `build', which function seems to be a post-Hitt. semantic development; the precise semantic range of \textit{tak\v{s}-} in Hitt. is an interesting research question \index{research question} \bit{p\=andu=wa=az a\v{s}andu} \p{3pl. pres. impv. act.} a clear example of the `phraseological' construction \index{quasi-serial verb construction} in OH (NS); though the quotative particle \textit{=wa} is suggests monoclausality, it is confirmed by the local particle \textit{=(a)z} which must belong with \textit{a\v{s}andu-} `sit; settle'; however, \textit{=(a)z} does not really belong here; the semantics here must be the transformative `sit down; settle', which in OH should be MP without reflexive; it may be a late addition, since in NH active + \textit{=z} means `sit down; settle'

\item[14 :] \bit{azzikandu akkuskandu} \p{3pl. pres. impv. act.} `eat (and) drink'; the quotative particle \textit{=wa} is shared by the asyndetically conjoined verbs; interestingly, the verbs `eat and drink' only take the refl. particle when collocated together, never on their own \bit{taga\v{s}\v{s}i} \p{2s. pres. act.} `devise'; IH's emendation \textit{ku}[\textit{-i\v{s}-ki}] is nonsensical with 2\supersc{nd} s. verb; \textit{ku}[\textit{-i\v{t}-ki}] seems more likely

\item[15 :] \bit{id\=alu y\=er} \p{3pl. pret. act.} `treated badly'; though elsewhere we find \textit{id\=alu} treated as direct obj. of verbs, in this collocation it functions adverbially \bit{tar\v{s}ikemi} \p{1s. pres. act.} `speak'; note the archaic form of iterative with \textit{i-}anaptyxis, and inceptive sense of \textit{-ske-} (cf. its use in supine, e.g. \textit{memi\v{s}kewan dai\v{s}} `began to speak')

\item[16 :] \bit{\=e\v{s}{\hith}at} \p{1s. pret. mp.} `seat onself'; recall that in OH, the medio-passive expresses `sit down; seat oneself', while the active means `is sitting, sits'; somewhat confusingly, this latter sense is later continued by the medio-passive, while the active + refl. ptcl. \textit{=za} is used for `sit down; seat oneself' (cf. \citetalias{GrHL} \S28.30)

\item[19 :] \bit{{\hith}ullanzai\v{s}} \p{c. nom. s.; i-st.} `fight; battle'; the morphology here is odd; ultimately, denom. to \textit{{\hith}ulle-} `fight' via an \textit{-nt-}stem derivative; further development into c. \textit{i-}st. is unclear

\item[20 :] \bit{waggariyat} \p{3s. pret. act.} `revolt against', but here seems to mean `cause to revolt' which is problematic; a connection is often assumed with \textit{wakk-} `be lacking'  (e.g. \textit{wakkari \p{3s. pres. mp.}}), but the semantic connection is prety weak; at least synchronically, we have obviously related nominal formations like \textit{waggariyawar} \p{n.} `rebellion'; CM takes the verb as denominative to \textit{*-ro-} adjective, with syncope acc. to Hitt. `\textit{ager}-rule'

\item[22 :] \bit{hantezziya\v{s}\v{s}=a} \textbf{UGULA LU{\mpl}} \bit{LIM} \textbf{\supersc{mD}[}\bit{U-} `foremost chief/overseer of 1000 men, Tarh-'; CM suggests that the list of names in lnn. 22-25 is one massive left-dislocation

\item[23 :] \textbf{{\men}\v{S}\`ATAM} `of the stewards'

\item[24 :] \textbf{{\wood}PA} `of the staff'

\item[26 :] \bit{saq{\hith\hith}un} \p{1s. pret. act.} `know'; this emendation is very insecure, essentially done \textit{ad sensum} \bit{anda} \p{adv.} `therein' (!); this emendation cannot be correct, since \textit{anda} is inexplicable in this context; CM suggests \textit{anku} \p{adv.} `fully, totally', which can occur with \textit{kuen-}; regardless of whether this suggestion is correct, the text should be presented \textit{an}-X

\item[28 :] \bit{uwater} \p{3pl. pret. act.} `bring'; the centripetal deictic value of the verb is strongly in evidence here, i.e. `brought (to me here)' \bit{panku\v{s}} \p{c. nom. s.} `the multitude, the masses; assembly'; it is this text which confirms that \textit{panku-} can also refer to the assembly, i.e. a Hitt. government body with judiciary power \bit{pr\=a{\ldots}{\hith}arkta} `(lit.) hold forth'; regularly in collocation with \textit{{\hith}engani} `doom, death', idiomatically `designate for/sentence to death'; interesting word-order

\item[29 :] \bit{kuwat=war=e} \p{adv.} `why?' \bit{munnanzi} \p{3s. pres. act.} `conceal, hide away'; syntactically, must be `they conceal them\p{acc.} with respect to (their) eyes\p{acc.}', viz. the {\greektext sq\~hma kaj" \Ar{o}lon ka`i m\'eros}; what exactly this means is less clear; significantly, it must be some punishment, which Telepinu opts for instead of death; it is part of his program that he puts an end to the cycle of political murders \bit{kar\v{s}au\v{s}} \p{adj. c. acc. pl.} `plain; bare; unadulterated, pure'

\item[30 :] \textbf{{\men}APIN.LAL} `farmers'; lit. `man of the plow (APIN)'; \textbf{ZAG.UDU}\bit{-az} \p{abl.} `shoulder' \bit{ma\v{s}du\v{s}} \p{c. acc. pl.} `?'; it is possible to read \textit{par-} instead of \textit{ma\v{s}-}, but it remains unclear

\item[31 :] \bit{salla\v{s}=pat} \p{adj. n. gen. s.} `great'; to \textit{salli-} `great; royal' \bit{pangariyattati} \p{3s. pret. mp.} `become numerous/massive'; again, for CM denom. to unattested \textit{*-ro-} adj.; note that the \textit{-i-} is only orthographic, as confirmed by spellings $<$ \textit{-(\v{s})ar-ya-} $>$ of \textit{na{\hith}sariyatt-}

\item[32 :] \bit{wit{\ldots}}\textbf{BA.U\v{S}} \p{3s. pret. act.} `proceeded to die'; another example of the `phraseological' construction \index{quasi-serial verb construction} in OH (NS) \bit{\v{s}iunan antu{\hith}\v{s}e\v{s}\v{s}=a} `men of the gods'; note the placement of the clitic \textit{=a} (probably should be non-geminating) after \textit{antu{\hith}\v{s}a-}, which is a striking example of prosodic domain formation between noun + gen. \index{prosodic domain}

\item[34 :] \bit{tuliyan} \p{c. acc. s.} `meeting, assembly; council' \bit{padalaz} \p{adv.} `(+ \textit{ket}from this point in time; from now on'; rightly, dupl. G II 9{\pr} \textit{ki-it pa-an-ta-la-a[z }, with omission of nasal in main text; the origin of \textit{pandalaz} is obscure; G. Meiser has compared Lat. \textit{pend\=o}; use of \textit{ket} shows that demonstratives did not have a formally distinct ablative in OH (\textit{kez} occurs only post-OH)

\item[35 :] \bit{takke\v{s}zi} \p{3s. pres. act.} `put together; assemble'; here somewhat idiomatically `wield (a knife)', though I could see a metaphorical use of `knife' here, i.e. `form a murder-plot' (\textit{vel sim.}); significantly, note that the negative carries over \index{gapping, NEG} from the previous clause

\item[36 :] \textbf{DUMU\supersc{\textit{RU}}} = Akk. \textit{maru} `son; male child'; here as a partitive genitive; dupl. G II 11{\pr} has DUMU\mpl.NITA `sons; male children'; \textit{=pat} in delimiting function (viz. \textit{only}) \bit{kikki\v{s}tari} \p{3s. pres. mp.} `become'; this passage very clearly shows that reduplication can indicate iterativity (like \textit{*-ske-}); the contrast between the general (habitual) case with reduplicated verb, but then the unreduplicated verbs in each specific circumstance is striking

\item[37 :] \bit{{\hith}antezzi\v{s}} \p{c. nom. s.} `first, foremost'; the main text has the NH form with syncope of the final syllable in \textit{i-}stems, but dupl. G II 12{\pr} has \textit{-z}]\textit{i-ya-a\v{s}}, the expected OH form \bit{t\=an peda\v{s}} \p{n. gen. s.} `in second place'; \textit{peda\v{s}} is a free-standing genitive; later develops (via hypostasis) into a kind of fixed phrase, with inflection of second member, i.e. \textit{d\=an peda-}; this may be reflected already in dupl. G II 12{\pr} \textit{p\=edan}

\item[38 :] \textbf{IBILA} = DUMU.NITA

\item[39 :] \textbf{\man}\bit{antiyantan} \p{c. acc. s.} `son-in-law; fianc\'e'; a univerbation of \textit{anda} + \textit{iya-} `go, march', the technical term for a man becoming a part of the woman's household through marriage; the reverse is the normal Hittite (and IE) practice

\item \bit{ammuk} \textbf{EGIR}\bit{-anda} `after me'; a NH substitute for OH postpositional phrase \textit{amm\=el appan}, which can be easily restored on the basis of dupl. G II 16{\pr}

\item[43 :] \bit{parkunummi} \p{1s. pres. act.} `cleanse (+\textit{ar{\hith}a}) thoroughly'; cf. CHD s.v. \textit{parkunu-} 5 trans. \textit{ad loc.}: ``Do not say: `I will thoroughly clean up,' while however you yourself (\textit{=za}) clean nothing up, but you yourself (\textit{=za}) rather oppress.''; CM strongly rejects this translation of \textit{=za} (`rubbish!'), rather indicates self-interest of speaker

\item[44 :] \bit{{\hith}atke\v{s}nu\v{s}i} \p{2s. pres. act.} `oppress'; beside \textit{{\hith}atku-} \p{adj.} `hard; severe', this has the look of a Caland formation; the the reflexive particle \textit{=za}is unexpected, as is the opposition of \textit{{\hith}atk-} and \textit{park-}; more typically, \textit{anda} + \textit{{\hith}atke\v{s}nu-} means `besiege (a city)'

\item[45 :] \bit{{\hith}a\v{s}\v{s}ana\v{s}=(\v{s})an=za=kan} \p{n. gen. (?) s.} `family; blood relations'; this form is a \textit{monstrum}, clearly created by a late copyist; we might expect instead \textit{{\hith}a\v{s}\v{s}ana\v{s}=\v{s}a\v{s}} \bit{kuenti} \p{2s. pres. act.} `strike; kill'; the introduction of 2s. desinence \textit{-(t)ti} originally proper to the \textit{{\hith}i}-conjugation into the paradigm of athematic \textit{mi-}verbs is very early, already frequent in OH

\item[47 :] \bit{\v{s}ume\v{s}\v{s}=a panku=\v{s}i\v{s}} `You shall be his assembly', i.e. the authoritative legal body; as is typical, the geminating \textit{=a} is incorrect; CM accepts IH's emendation $<$\textit{-i\v{s}}$>$ \bit{kar\v{s}i} \p{adv.} `properly; clearly; frankly' \bit{e\v{s}n\=a\v{s}} \p{n. gen. s.; r/n-st.} `blood'; the archaic genitive singular with laryngeal deletion preserved here; IH has wrongly emended, restoring \textit{-{\hith}-}

\item[48 :] \bit{=wa} \p{quot. ptcl} the particle quotes the tablet \bit{au} \p{pres. impv. act.} `see, behold'

\item[49 :] \bit{=apa} \p{local ptcl.} \textit{=apa} typically denotes movement into physical contact as here, sometimes with telicizing value (cf. \citetalias{GrHL} \S28.100-107) \bit{{\hith}a\v{s}\v{s}annai} \p{n. d-l. s.} `family; kin-group'; the form with ending \textit{-ai} is a \textit{monstrum}, probably an error influenced by \textit{sallai}

\item[50 :] \bit{i\v{s}tarna} \p{postpost.} `within; among'; partitive here, viz. `whoever among the brothers and sisters'

\item[51 :] \bit{suwayezzi} \p{3s. pres. act.} `look (to)'; IH takes from \textit{suwai-} `thrust; shove', \textit{non rect\=e}; it is rather \textit{suwaya-}, also found in the legal formula of the OH Laws: \textit{parna=\v{s}\v{s}e=a suwaizzi} `And he shall look to the house\ldots'; resemblance even stronger if allative \textit{{\hith}ar\v{s}\v{s}ana} `head' is correct, though the sense is unclear; \bit{uttar=\v{s}it} \p{n. nom. s.} `(his) matter', viz. of the guilty party; idiomatic with \textit{paizzi} = `is found guilty'

\item[52 :] \bit{\v{s}arnikdu} \p{3s. impv. act.} `pay, make recompense; \textit{dare poenam}'

\item[53 :] \bit{iwar} \p{postpos.} `like; in the manner'; governs the four preceding names, all genitive; the third (i-stem) is completely unambiguous, the last opaque with syncope \textit{-u\v{s}} $<$ \textit{*-uwas} \bit{kunanzi} \p{3pl. pres. act.} `kill'; impersonal here

\item[54 :] \textbf{\'E}\bit{-ri=\v{s}\v{s}i=\v{s}\v{s}i} \p{n. d-l. s.} `(his) house'; good OH clitic doubling marking intrinsic possession

\item[55 :] \bit{=pat} \p{ptcpl} `only' \textbf{\'E}\bit{-\v{S}U=ma=\v{s}\v{s}i=\v{s}\v{s}an} \p{n. d-l. s.} `his house'

\item[56: ] \bit{kuedani ser} \p{interr.} `on whatever account'; note preceding, topicalized DUMU\men.LUGAL \bit{{\hith}arki\v{s}kantari} \p{3pl. pres. mp.} `die (habitually)'; note switch to mediopassive in \textit{-ske-} form of \textit{{\hith}ark-} `die'

\item[57 :] \bit{\=UL} \p{\sc Neg} IH's `(It is) not (the case for)\ldots' is an attempt to repair a clear case of the scribe forgetting to add the verb at the end of a long list; C. Watkins observed the same in academic German \textbf{{\wood}SAR.GE\v{S}TIN{\hpl}}\bit{-\v{S}UNU} `(their) vineyards'

\item[58 :] \textbf{[AR-]DI{\hpl}}\bit{-\v{S}UNU}{\ldots}SAG.G\'EME.\`IR{\mpl} `male (and) female slaves'

\item[59 :] \bit{m\=an} one dupl. not included by IH preserves OH \textit{takku} here rather than MH/NH \textit{m\=an}

\item[60 :] 
%\textbf{\'E}\bit{-ZU=ma=\v{s}\v{s}i} \p{n. d-l. s.} `to (his) house'; why realized as \textit{-ZU} here? 
\bit{piyani} \p{?} `give; send'; formally, problematic; appears to be dat-loc. of verbal noun \textit{piy\=atar*}; however, functions like infinitive \textit{piyanna} as complement to \textit{natta \=ara}; cf. \textit{CHD} s.v. \textit{pai-}, where it is tentatively suggested that dat-loc. may have been acceptable as substitute for allative  \bit{izzan} \textbf{GI\v{S}}\bit{-ru} `straw (and) wood; \p{idiom.} a (little) bit'; both n. acc. s., and regularly collocated asyndetically, a calque on an Akkadian construction

\item[62 :] \bit{BITUM} `house'

\item[63:] \bit{danna} \p{inf.} `take'; cf. inf. \textit{tiyanna} `place' \bit{i[ - ]laliyanzi} \p{3pl. pres. act.} `?'; one sign cannot be read, and IH's restoration \textit{-da-} makes no sense; the verb must in any case take an infinitive complement here, hence IH trans. `desires to'

\item[64 :] \bit{a\v{s}i=man=wa} \p{pron. c. nom. s.} `this, that'; in NH, recharacterized as \textit{a\v{s}i\v{s}}; the modal particle \textit{=man} in a wish

\item[65 :] \bit{takki\v{s}kizzi} \p{3s. pres. act.} `fashion; do'; the verb concludes the protasis of a conditional whose apodosis is oddly lacking

\item[66 :] \textbf{{\men} I\v{S} GU\v{S}KIN} `man of the golden coach/chariot', contra IH `Goldknappen'; I\v{S} now transliterated RU\v{S}\subsc{7}

\item[67 :] \textbf{\men}\bit{\v{s}ala\v{s}{\hith}iya\v{s}} `(a palace official)'

\item[68 :] \bit{\textsubdot{S}ERI} `open country, land; field' \bit{suma\v{s}} \p{2pl. pron. nom.} `you'; dupl. E II 16{\pr} has expected OH \textit{\v{s}ume\v{s}} \bit{\v{s}ektin} \p{2pl. impv. act.} `know'; the force of preverb \textit{appan} is somewhat unclear, perhaps `recall, remember'; note that it is missing in one dupl.

\item[69 :] \bit{INA PANI=KUNU} `before you' = Hitt. \textit{pera\v{s}=(\v{s})mit} \textbf{GISKIM}\bit{-a\v{s}} \p{c. nom. s.} `sign, symbol'

\item[70 :] \bit{na\v{s}\v{s}u} `either{\ldots} (+\textit{na\v{s}ma}) or'; the latter derived by syncope from \textit{na\v{s}\v{s}u=ma}, cf. \citetalias{GrHL} \S1.77

\item[72 :] \bit{suma\v{s}\v{s}=a} \p{2pl. pron} `you'; incorrect geminating \textit{=a} in main text confirmed by dupl. F 11{\pr} and G III 5{\pr} \bit{panku\v{s}} \p{c. nom. s.} `assembly'; appositional to subject \textit{suma\v{s}} \bit{epten} \p{2pl. impv. act.} `take, seize'; however, the absence of an object makes this interpretation problematic, as does the preverb; Otten therefore emends to \textit{\=erten} `you shall intervene' (\textit{ar-} `arrive' + \textit{anda} = `intervene'), with secondary \textit{e-}vocalism of \textit{ar-} \bit{karipten} \p{2pl. impv. act.} `devour, consume'; IH has emended LA sign in the main text to RE/RI \textit{ad sensum} and with support of dupl. \textbf{\supersc{UZU}KAxUD}\bit{-it} \p{instr.} `teeth'

%spellings of this kind are crucial as they preserve evidence for \textit{*-o/e-} ablaut in the \textit{{\hith}i}-conjugation, viz. /krip-t\'en/ with pre-ictic raising of root vowel \textit{*-e-}

\smallskip

\item[\underline{Rs. III}]

\smallskip

\item[2 :] \textbf{{\men}NIM.GIR} `herald, crier' \bit{ku\=e\v{s}} \p{c. nom. pl.} `who'

\item[3 :] \bit{appizzie$<$\v{s}$>$\v{s}u\v{s}} `last (ones); subordinates'; CM would instead emend \textit{appeziu\v{s}}

\item[4 :] \bit{wedantes} \p{ptcpl. c. nom. pl.} `build'

\item[5 :] \bit{tarnatti} \p{2s. pres. act.} `let; leave (behind)' 

\item[6 :] \bit{{\hith}alkiya=ma=at} \p{c. d-l. (?) s.} `grain'; an example of the `hanging topic' construction, with resumptive clitic pronoun \textit{-at} \index{hangingtopic} \textbf{10}\bit{-i\v{s}} \textbf{20}\bit{-i\v{s}} `10 (or) 20' \bit{nai\v{s}{\hith}ut} \p{2s. impv. mp.} `turn; send; lead'; the restoration here is doubted by CM, though clearly a form of \textit{nai-} (\textit{wid\=ar nai-} = Lat. \textit{aquam ducere})

\item[44 :] \bit{\v{s}\=e\v{s}\v{s}andu} \p{3pl. impv. act.} `sow'; iter. to \textit{\v{s}ai-/\v{s}iya-}; recall that for JJ iteratives in \textit{-\v{s}\v{s}-} are originally \textit{*-{\hi}s-} desideratives; CM regards development from desiderative to iterative implausible, and Pal. \textit{pi-i-\textbf{\v{s}a}-} with single spelling of sibilant is grounds for rejection on formal grounds

\item[45 :] \bit{mar\v{s}atar} \p{n. acc. s.} `falsehood; fraud'; dupl. B III 11{\pr} preserves plene spelling \textit{-\=atar} \bit{\=e\v{s}\v{s}anzi} \p{3pl. pres. act.} `do, make'; iter. \textit{iya-}, with problematic fluctuation of spelling in initial syllable

\item[46 :] \bit{ila\v{s}ni} \p{n. d-l. s.; r/n-st.} `increase; fertility; crop yield, harvest' \bit{par\=a} \p{postpos.} `at' \bit{gipe\v{s}\v{s}ar} `ell, cubit (unit of measure)'

\item[47 :] \bit{{\hith}aminki\v{s}ker} `bind, tie'; perhaps metaphorically here `get control (of)' \bit{utn\=e} \p{n. d-l. s.} `land'; CM inteprets as dative of disadvantage \bit{akku\v{s}kir} \p{3pl. pret. act.} `drink'; if rightly restored, two possible interpretations: (i.) labio-velar is treated as velar + vowel \textit{-u-}; (ii.) the epenthetic vowel is rounded by the adjacent labio-velar

\item[50 :] \bit{\v{S}UM-a=\v{s}mit} \p{n. instr. s.} `name'; likely underlying \textit{lama\v{s}=\v{s}mit}, with /\textit{n=s}/ $>$ [s=s] assimilation \bit{siye\v{s}ki} \p{2s. impv. act.} `press; seal' \textbf{{\men}AGRIG} `administrator, steward' \bit{daliyanzi} \p{3pl. pres. act.} `leave (you) alone', with \textit{=du} the 2\supersc{nd} s. acc.-dat. encl. (beside regular \textit{=(t)ta})

\item[52 :] \bit{nu=wa=at=ma=z} this clitic chain is a \textit{monstrum}; the expeced \textit{-r-} of \textit{-war-} in intervocalical position is absent; meanwhile the postvocalic allomorph of the topic marker \textit{=ma} is unexpected after a consonant; the dupl. shows reflexive \textit{=za} in contrast to main text \textit{=az}, providing more evidence for the underlying form /\textit{z}/ [\texttslig]

\item[53 :] \bit{karpanzi} \p{3pl. pres. act.} `take away'

\item[73 :] \bit{t\=uriyan} \p{sup.} `harness, yoke; constrain'

\item[75 :] \bit{sarninki\v{s}ki} \p{2s. impv. act.} `give compensation for'


\section{The Palace Chronicle}
\item[Publication:] [add]
\item[Edition:] \citet{dardano1997palace}
\item[Background:] [add] \\

\item[Ro. Col. I]

\item[A1 :] \textbf{{\city}}\bit{Ku\v{s}\v{s}ar} `(city of ) Kussar'; the story is framed as Mursili talking about his father Hattusili I; Kussar is the hometown of Hattusili I

\item[A2 :] \bit{pa\v{s}\v{s}ilan} \p{c. acc. s.} `pebble; stone' \bit{I\v{S}BAT} `take, seize; find'; must be `find' here, so possibly \textit{wemiya-} (cf. A12, below), though typically = \textit{\=epta} `take, seize'

\item[A3 :] \bit{parer} \p{3pl. pret. act.} `blow on; kindle, light'; serial construction impossible here given presence of subject clitic \textit{=e} \bit{{\hith}\=upper} \p{3pl. pret. act.} `mistreat; behave badly toward'; strong stem \textit{{\hith}uwapp-} , always with dat. obj. \bit{kuida} \p{conj.} `after; since'

\item[A4 :] \bit{{\hith}attanner} \p{3pl. pret. act.} `hit; beat up'
\bit{saminuer} \p{3pl. pret. act.} `cause to disappear'; PD's ``e lo bruciarono fino a farlo scomparire'' viz. `burned to ashes' is purely \textit{ad sensum}

\item[A6 :] \bit{marnuann=a} \p{c. acc. s.} `\textit{m-}drink; kind of beer' \bit{marakta} \p{3s. pret. act.} `divide; apportion; distribute'; it is possible that there is more than one verbal root of the shape \textit{mark-} 

\item[A7 :] \bit{\v{s}arter} \p{3pl. pret. act.} `rub'; see CM in FsInsler

\item[A8 :] \textbf{MUN}\bit{-an} \p{c. acc. s.} `salt' \bit{eukta} \p{3s. pret. act.} `drink'; viz. under compulsion; note spelling which points to unitary labio-velar

\item[A9 :] \bit{tuwarner} \p{3pl. pret. act.} `break' \bit{wal{\hith}i} \p{n. acc. s.} `\textit{w-}drink'; CM: `the drink that gives you a wallop'

\item[A11 :] \textbf{KUR}\bit{ Arzawi=ya} \p{c. d-l. s.} `and also in Arzawa', with conj. \textit{=ya}

\item[A12 :] \bit{udai} \p{3s. pres. act.} `bring (here)'; viz. to the Hattusa, the deictic center in this episode; it is also described as geographically `up' (= `north') relative to Arzwawa (cf. A 14, below); N- is supposed to be some kind of tax-collector who is then caught embezzling \bit{wemiezzi} \p{3s. pres. act.} `find'; more likely a postposed non-restrictive relative, rather than correlative with \textit{apa\v{s}\v{s}=a} \bit{pittaizzi} \p{3s. pres. act.} `pay for; hand over, transfer'

\item[A13 :] \bit{i\v{s}ia{\hith\hith}i\v{s}} \p{3s. pret. act.} `announce, proclaim; denounce, turn in'

\item[A14 :] \bit{hatrait} \p{3s. pret. act.} `send'; used exceptionally here to `send' a person; almost exclusive with (written) messages; some Akkadian interference may be at work

\item[A15 :] \bit{n\=awi} `(\textit{lit.}) not yet' = `before'

\item[A16 :] \textbf{GU\subsc{4}}\bit{-li} \p{adv.} `ox-like; in the manner of oxen' \bit{\textit{\supersc{m}}Nunnu\v{s}\v{s}=a} \p{c. gen. s.}; \textit{-u\v{s} $<$ *-uwa\v{s}} via syncope; uninterpretable as nom. with 3\supersc{rd} pl. \textit{epper}

\item[A17 :] \bit{\textit{\man}kaina\v{s}=\v{s}an} \p{c. acc. s.} `family; kith, relative by marriage'; probably synchronically via assimilation from /{\it-n=s-}/, though the original construction was likely with free-standing genitive

\item[A18 :] \bit{{\hith}uekta} \p{3s. pret. act.} `slaughter; kill'; a postposition (e.g. \textit{peran}) seems to be missing; the sense is clearly `slaughtered him \textit{before} their eyes'; IH's emendation $<$-\textit{a\v{s}}-$>$ must be correct

\item[A20 :] \bit{\textit{\cloth}i\v{s}{\hith}ial=\v{s}mett=a} \p{n. acc. pl.} `belt' \bit{e\v{s}{\hith}a\v{s}kanta} \p{ptcpl. n. acc. pl.} `bloody'; archaic compound with \textit{-a-}vocalism ($<$ \textit{*-\textsubring{n}}) from oblique of \textit{*r/n}-stem + ptcpl. \textit{i\v{s}ke-} `smear', which (\textit{pace} AK) is not and never was a \textit{*-ye/o-} verb, as confirmed by spelling on new sliver KBo 34.243 \textit{i-i\v{s}-ke-ez-zi} (Ritual of Zarpiya Ai37 [= f3]) \index{compound}

\item[A21 :] \bit{\v{s}eknu\v{s}=(\v{s})met} \p{n. nom. s.; s-st.} `(their) cloak' \bit{anda nean} \p{ptcpl. n. nom. s.} `turned inward', i.e. so the blood is concealed

\item[A22 :] \textbf{\supersc{d}UTU}\bit{=met} `my Majesty'; encl. poss. is properly \textit{=mi} with voc.; this usage exemplifies \textit{=it} as the `all-purpose' encl. poss. ending

\item[A23 :] \bit{\v{s}i\v{s}ta} \p{2s. pres. mp.} `impress, imprint; engrave'; we expect impv. \textit{\v{s}i\v{s}\v{s}a}, in parallel to \textit{\=it} (which constitutes a complete, independent clause); potentially a copyist error, since addition of only a single stroke to make SA to TA; \textit{CHD} gets this wrong, cannot be 2\supersc{nd} sing. pret. to base verb \textit{sai-} (in OH, \supersc{X}\textit{\v{s}aita}) \bit{karda} \p{n. (?) s.} `heart'; the exact restoration is unclear, but the expression is formulaic: `seal it in your heart' (\textit{vel sim.})

\item[A24 :] \bit{{\hith}urla\v{s}\v{s}=a} \p{c. gen./d-l. pl.} `Hurria(n)' \textit{na{\hith\hith}ta} \p{3s. pret. act.} `be afraid (of) \p{(+gen./d-l)}'

\item[A25 :] \bit{e\v{s}{\hith}e} \p{c. d-l. s.} `lord, master'; totally aberrant spelling with {\Hith}E instead of {\Hith}I \bit{penni\v{s}} \p{3s. pret. act.} `drive (to)'; perhaps here in its etymological sense $<$ \textit{*pe + *nai-} {kukkure\v{s}ker} \p{3pl. pret. act.} `cut up; mutilate' ; redup. iter. to \textit{kuer-} `cut'; the copy vowel \textit{-u-} in the reduplicant strongly suggests that the once unitary labiovelar /{\labk}/ has become a sequence /\textit{ku}/

\item[A26 :] \textbf{{\man}ZABAR.DAB} `man of the bronze bowl; dishbearer'

\item[A27 :] \bit{I\v{S}ME} \p{3s. pret. act.} `heard'

\item[A31 :] \bit{A{\Hith}I} `brother'

\item[*A33 :] \bit{takkania\v{s}=\v{s}a\v{s}} \p{c. gen. s.; i-st.} `chest, breast' \bit{par{\hith}u\v{s}\v{s}u\v{s}} \p{c. acc. pl.} `?'; completely obscure; another possible reading is \textit{ma\v{s}{\hith}u\v{s}\v{s}u\v{s}} (MA\v{S} = PAR)

\smallskip
\item[Ro. Col. II]
\smallskip

\item[A1 :] \textbf{\clay}\bit{{\hith}ar{\hith}aran} \p{c. acc. s.} `{\hith}-vessel; pithos'; \textit{suwan} `full' with instr. suppressed

\item[A3 :] \bit{{\hith}inkatta} \p{3s. pres. mp.} `hand over, give; offer, present'

\item[A4 :] \bit{wit{\ldots}tet} `came{\ldots}said'; this looks a lot like QSV \index{quasi-serial verb construction} in OH, but with overt subject; a potential examples of \textit{uwa-} as a raising verb? \bit{apa\v{s}\v{s}=a{\ldots}apa\v{s}\v{s}=} `the one{\ldots}the other'

\item[A5 :] \bit{kuin} \p{c. acc. s.} `which'; a postposed relative modifying \textit{ap\=un} GE\v{S}TIN\textit{-an}

\item[A6 :] \bit{IQBI} `spoke'; again, QSV-like (cf. A4, above)

\item[A7 :] \bit{\=e\v{s}\v{s}ikir} \p{3pl. pret. act.} `do, make; treat'; here, coll. Eng. `work over'; iter. in \textit{-\v{s}ke-} to impfv. in \textit{-\v{s}\v{s}-} to \textit{iya-} `do' with anapytxis, which confirms that sonority plateaus are not tolerated

\item[A9 :] \bit{kuwatta kuwatta} \p{adv.} `in every respect' \textbf{L\'U}\bit{-e\v{s}\subsc{17}} `man'; here implicitly `(good) man'; according to \citet{zucha1988hittite}, the \textit{-e-} spelling preserves an archaism, viz. an original HK \textit{*-n-}stem \textit{*pes-\=en}, secondarily recharacterized (as always in Hitt.) with nom. s. \textit{-s}; note that this derivation gets rid of a putative case of anaptyxis in a cluster \textit{*-sn-}

\item[A10 :] \bit{paknuer} \p{3pl. pret. act.} `signify; report; malign, slandered' \bit{arnut} \p{3s. pret. act.} `cause to move; remove' \bit{IRDI} `transfer, relocate'

\item[A11 :] \textbf{{\man}AGRIG}\bit{-an} `(some kind of) administrator'; this is a huge demotion from EN\textit{-a\v{s}} {\city}\textit{{\Hith}urmi} `lord of {\Hith}urmi', as becomes clear in the following lines \bit{sarku\v{s}} \p{c. nom. s.} `exalted'; the statement anticipates the following \textit{=ma}, i.e. `(previously) he was{\ldots}but'

\item[A13 :] \bit{kaqqapu\v{s}} \p{c. acc. pl.} `(some kind of) critter'

\item[A14 :] \bit{maklante\v{s}} \p{adj. c. nom. pl.} `lean; thin; scrawny'; not only is A- reduced to peddling `critters', but they are scrawny ones to boot

\item[A15 :] \textbf{\man}\bit{{\hith}uprala\v{s}} \p{c. nom. s.} `potter'

\item[A16 :] \textbf{\man}\bit{mania{\hith\hith}atallan} \p{c. acc. s.} `administrator'; agent noun with \textit{-(a)talla-} suffix to factitive stem in \textit{*-e{\hii}-}; dupl. C has same suffix built to iterative stem

\item[A17 :] \bit{man=an=k\'an} seems to express an unrealized wish, viz. `would have liked to (do X), but instead{\ldots}' \textbf{\'E.EN.NU.UN} T. \textit{s.v.} `jail'

\item[A18 :] \bit{udd\=ar arai\v{s}} `\p{idiom.}rumors arose' + dat. of interest; cf. coll. Eng. `word got out about' \bit{p\=ir} \p{3pl. pret. act.} `send'; most likely to \textit{pe/iya-}, though \textit{pai-} `give' s formally and functionally possible as well

\item[A19 :] \bit{ti\=et} \p{3s. pret. act.} `step'; the subj. encl. rules out \textit{t\=et} `spoke', which is otherwise tempting, since the verb introduces direct speech

\item[A21 :] \bit{{\hith}ali{\hith}latti} \p{2s. pres. act./mp.} `prostrate oneself (to)' with acc. complement; to be understood with \textit{mar\v{s}anza zik}, either `you (are) false in that you{\ldots}' or `you, the false one (or: `falsely')\ldots'

\item[A22 :] \textbf{{\men}KU\v{S}\subsc{7}} `wagon-drivers; charioteers' \bit{e\v{s}ir} \p{3pl. pret. act.} `be'; the syntax here is problematic; S- could be nom., but M- cannot; hence either an error, or---perhaps---an inserted nominal sentence `---they were the chiefs of{\ldots}---' following the topicalized acc. objects

\item[A23 :] \textbf{\man}\bit{uralla\v{s}\v{s}aman} \p{c. acc. s.} `(some official rank)'; the significance in the narrative is that I- is made the superior to M- \bit{la{\hith\hith}emu\v{s}} \p{c. acc. pl.} `?'; in this context clearly `late night inspections' on which he discovered various violations (\textit{wa\v{s}tau\v{s}}), but the form is morphologically aberrant and very difficult \bit{wemir} \p{3pl. pret. act.} `find'; both plural and preterite are probably incorrect; better is dupl. C \textit{wemiyazzi} \bit{{\hith}ue\v{s}kizzi} \p{3s. pres. act.} `run'; iter. to \textit{{\hith}uwai-} `run' 

\item[A25 :] \textbf{{\man}\v{S}U.I} `(of the) barber' \bit{ubatiya\v{s}=\v{s}a\v{s}} \p{c. gen. s.} `land grant, estate; allotted share'; usually refers to a `land grant', but here `allotted share (viz. of the troops)'

\item[A27 :] \bit{d\=ai=\v{s}\v{s}an} \p{3s. pret. act.} the syntax of what follows is very difficult; likely is: `after (someone) had placed the neophytes (\textit{ammiyantu\v{s}}) of the drivers ({\men}KU\v{S}\subsc{7}) on the chariotry (AN\v{S}E.KUR.RA\textit{-a\v{s}})', with locatival reading supported by \textit{=\v{s}\v{s}an}

\item[A28 :] \bit{\=ammiyantu\v{s}=(\v{s})mu\v{s}} \p{c. acc. pl.} `small; young'; for CM, the {*}single{*} example of a privative compound (\textit{\#*\textsubring{n}- `NEG' $>$ *\#on- $>$ \#an-*}), though there is substantial spelling fluctuation \index{compound} \bit{maniya{\hith\hith}e\v{s}kizzi} \p{3s. pres. act.} `manage; train, instruct'; the verb introduces a list of skills taught by I-; it is normally acc. theme + dat. beneficiary, but here appear to be double acc.

\item[A29 :] \textbf{GI}\bit{-an} \p{c. acc. s. / gen. pl.} `reed; straw; arrow'; the case is ambiguous, since verbal nouns may take either acc. or gen. complement \textbf{{\wood}UMBIN} \p{c. d-l. s.} `wheel; (legal) trial'; must be interpreted as dat.-loc. sing., but either a phonetic complement or an Akkadographic prep. is expected\bit{{\hith}a\v{s}{\hith}a\v{s}uwar} \p{n. acc. s.} `scraping; sharpening'; deverbative abstract from onomatopoeic verb based on `scraping' sound (\textit{pace} Puhvel et al., unrelated to {\hith}a\v{s}\v{s}- `open') \bit{app\=atar} \p{n. nom./acc. s.} `taking up; seizing' \bit{annanut} \p{3s. pret. act.} `train, instruct'

\item[A30: ] \bit{k\=un} the tablet is unclear, and PD's reading nonsensical; CM would emend \textit{ku-i-u\v{s}}, with indeterminate referent

\item[A31 :] \bit{k\=u\v{s}{\ldots}k\=u\v{s}{\ldots}k\=u\v{s}} `(some of) these{\ldots}(these) others{\ldots}(these) others; resumes \textit{ku-i-u\v{s}} above \textbf{{\men}NIMGIR} \p{c. gen.pl.} `herald'

\item[A32 :] \bit{ulke\v{s}\v{s}ara{\hith\hith}ir} \p{3pl. pret. act.} `cause to be skillfull / expert (\textit{ul/wal-kissara})'

\item[A33 :] \bit{si\v{s}kanzi} \p{3pl. pres. act.} `shoot (a bow), hurl'; iter. to \textit{sai-/siya-} \bit{{\hith}azzizzi} \p{3s. pres. act} `strike; (score a) hit'

\item[A34 :] \bit{e=az} PD errs here; per CM, read instead \textit{e-uk zi-ik} `drink! You are the troop-king!'  \bit{iyal} `full'; connected by CM to \textit{iyattar} `fullness' (whence semantics contra \textit{CHD}) \textbf{GAL}\bit{-ri} \bit{n. acc. s.} `(fired clay) cup' (= \textit{zeri-})

\item[A35 :] \bit{uw\=atar} \p{n. acc. s.} `water'; recall that AK makes a lot out of this rare spelling, but for no justifiable reason; it is purely orthographic

\item[A39 :] \bit{i\v{s}tarniktat} \p{3s. pret. act.} `ail; afflict'; like simple pres., nasal-infix pres. functions impersonally (`it ails X' = `X is sick')
 
\item[A41 :] \bit{markiyaru} \p{3s. impv. mp.} `reject, refuse; disapprove' 

\item[A7{\pr} :] \textbf{KU\subsc{6}}\bit{-un} `fish'

\item[A10{\pr} :] \bit{INA QATI} `by means of; via'; lit. `in the hand'

\item[A12{\pr} :] \bit{{\hith}\=alir} \p{3pl. pret. act.} `kneel; prostrate oneself'

\item[A13{\pr} :] \bit{{\hith}alziyati=wa} \p{3s. pret. mp.} `call out'; already in OH, can function as here as true passive, i.e. `it was called out'

\item[A14{\pr} :] \bit{aru\v{s}=\v{s}mu\v{s}} \p{c. acc. pl.} `friend; companion' \bit{uppir} \p{3pl. pret. act.} `send (here)'; to \textit{uppa-}  \bit{ta{\hith}i\v{s}kir} \p{3pl. pret. act.} `cause; inflict; apportion'

\item[A15{\pr} :] \bit{\=eskanta} \p{3pl. pres. mp.} `sit'; iter. to \textit{es-} `sit'

\item[A16{\pr} :] \bit{\=appann=a} \p{adv.} `and in addition' (?)

\item[A17{\pr} :] \bit{k\I} \p{c. nom. pl.} `these'; better is dupl. G 7{\pr} \textit{ke-e}

\item[A19{\pr} :] \textbf{\wood}\bit{zaluwani\v{s}=(\v{s})ma\v{s}} `(wooden) tray' \bit{{\hith}apa\v{s}u\v{s}} \p{c. acc. pl.} `?'

\item[D3{\pr} :] \bit{per\=an=a} \p{postpos.} `before'; a unique spelling here (vs. regular \textit{p\=eran}), which looks like a second case of an enclitic conditioning rightward accent shift \index{clitic accent shift}

\item[D6{\pr} :] \bit{ariyalli=ma} \p{d-l. s.} `basket; handbag' \bit{ap\=ell=a ap\=ell=a} `of each of those ones'; if rightly restored, a pronominal amredita \index{amredita}

\item[D7{\pr} :] \textbf{{\men}DUGUD} `dignitary'

\smallskip
\item[\underline{Fragments}]
\smallskip

\item[32{\pr} :] \textbf{GU\subsc{4}.APIN.L\'A} `cow-farmer'

\smallskip
\item[\underline{KUB XXXVI 105 Vo}]
\smallskip

\item[12{\pr} :] \bit{anduwa{\hith\hith}a} `man'	= \textit{antu{\hith}\v{s}a-}; here the archaic form is preserved which  per Eichner $<$ AK \textit{*en-{\dh}we{\hii}-\=os} `having spirit/breath within'; formally and semantically, cf. {\greektext \As{e}nj\=umos}; the accent/ablaut is debatable, but the etymology is secure

\smallskip
\item[\underline{VBoT 33}]
\smallskip

\item[16{\pr} :] \bit{{\hith}urzakizzi} \p{3s. pres. act.} `curse'; iter. \textit{{\hith}uwart-}, if rightly restored

\end{description}

\section{Apology of Hattu\v{s}ili}

\begin{description}

\item[Publication:] 
\item[Edition:] \citet{otten1981hattusili}
\item[Background:] Otten's `Apology' of Hattusilim is somewhat misleading; probably closer to the mark is `autobiography' (\textit{vel sim.}). The text is in many ways formulaic, speaking to the development of a well-defined scribal style in New Hittite.

\smallskip

\item[1 :] \textbf{\supersc{m}}\bit{Tabarna} `T-'; originally the name of a king, T- is clearly used here by Hattusili as an official royal title

\item[4 :] \textbf{\v{S}\`A.BAL} `descendant'; H- traces his ancestry to his grandfather, then skips directly to an earlier king with the same name; this deeply formulaic genealogical pattern is also used by Tut{\hith}aliya on the famous bronze tablet; notably, old H- is not LUGAL.GAL `great king', though eventually he did become king of Hattusa

\item[5 :] \bit{pr\=a handantar} `providence'; in this formation, \textit{pra} has its etymological sense `before, in front', hence `that which is arranged (\textit{{\hith}andai-}) in advance \textit{pra}', i.e. `providence' (cf. Puhvel, s.v.); thematically, `providence of I-' is programmatic, as H- will proceed to repeatedly attribute all his good fortune to I-; syntactically, note the absence of clause-initial \textit{nu--}, which shows that even in NH, a clause-initial conjunction is not grammatically required (contrary to the claims of many), but rather that \textit{nu--} has semantic content---namely, a `processual' function

\item[6 :] \bit{ziladuwa} \p{adv.} `in the future; hereafter'; lit. `henceforth', a Luwianism, though it never occurs in Luwian as such; the morphology after the initial deictic particle \textit{zi-} = Hitt. \textit{ki-} is mysterious

\item[7 :] \textbf{NUMUN} `seed; descendant' = Hitt. \textit{warwallan} \supersc{D}UTU\supersc{\textit{\v{S}I}} is redundantly repeated; note the formula is the same as in ln. 4, above; \textbf{DINGIR{\mpl}}\bit{-a\v{s}=kan} \p{c. d-l. pl.} `gods'; a rare use of sentence-internal \textit{=kan}, where it attaches directly to a dat.-loc. nominal; the archaism of this usage is doubtful however; the Hittite-internal evidence points to a NH development, and the external comparison with \textit{-cum} in Italic (e.g. Lat. \textit{vobiscum}) has been challenged by \citet{fortson2010kan}

\item[8 :] \bit{na{\hith\hith}an} \p{ptcpl. c. acc. s.} `feared; awestruck; reverent'; neut. agreement with the nearer NUMUN; often argued to be some kind of \textit{n-}stem, \textit{non rect\=e}

\item[9 :] \bit{ABU=YA=nna\v{s}=za} \p{c. nom. s.} `our father'; there is no overt introduction to direct speech; note, moreover, `asyndeton', i.e. absence of \textit{nu}; CM argues that this text gives strong evidence for the `prosecutive' semantic function of \textit{nu}: it is present when there is a logical connection to the preceding, absent when there is not; the reflexive particle provides good counter-evidence to a very stupid idea, namely, that \textit{=za} occurs with \textit{{\hith}a\v{s}\v{s}-} `beget' only when the subject is the `mother' (due to the `internal' nature of the birthing process); this is obviously refuted by its use with `father' here

\item[10 :] \textbf{\supersc{m}NIR.G\'AL}\bit{-in} \p{c. acc. s.; i-st.} `Muwatalli'; \textbf{DINGIR{\mpl}-IR}\bit{-inn=a} \p{c. acc. s} `Massanuzzi'; a Luw. \textit{Satzname} `asks the gods (for)'

\item[11 :] \bit{nu=za} in NH, the reflexive particle is obligatory in 1\supersc{st}/2\supersc{nd} person nominal sentences

\item[12 :] \textbf{\supersc{KU\v{S}}KA.TAB} `halter; reins'; a free-standing genitive, i.e. `(the man) of the halter of the donkey' \textbf{AN\v{S}E}\bit{=za} \p{c. gen. s.} `(of the) donkey'; note again obligatory \textit{=za}, as in preceding ln. 11

\item[13 :] \textbf{\`U}\bit{-it} \p{c. instr.} `dream'

\item[14 :] \bit{maninkuwante\v{s}} \p{adj. c. nom. pl.} `short; brief'; probably a compound, \index{compound} whose second element has been compared to Lat. \textit{prop-inqu\=o}, Skt. \textit{-a\~nc-}

\item[15 :] \bit{\'UL=war=a\v{s}} \textbf{TI}\bit{-anna\v{s}} lit. `he was not of life' = `he shall not / is not (destined) for (long) life'; note subj. clitic in nominal sentence

\item[16 :] \bit{p\=ai} \p{2s. impv. act.} `give'; the long diphthong in the imperative here is a NH development; in OH it is short (per Cowgill)

\item[18 :] \textbf{\'IR}\bit{-anni} \p{n. d-l. s.} `service; servitude'

\item[19 :] \textbf{\man}\bit{sankunniyanza} \p{c. nom. s.; nt-st.} `priest'; \textit{-nt-}stem beside usual \textit{i-}stem \textit{\v{s}ankunni-} with same meaning; the word was originally a \textit{u-}stem \textit{sangu-}, borrowed from Akkadian into Sumerian and ultimately into Hittite; the \textit{-unni-} suffix is usually attributed to Hurrian  \textbf{BAL}\bit{-a{\hith\hith}un} \p{1s.pret.act.} `libate; offer' = Hitt. \textit{i\v{s}pant-}

\item[20 :] \bit{l\=ulu} \p{n. acc. s.} `prosperity; welfare'; more generally, `a desirable state'; note \textit{Glossenkeil}

\item[21 :] \bit{par\=a {\hith}andante\v{s}ta} \p{3s. pret. act.} `become providential' (for semantics, cf. ln. 4 above); \textit{-e\v{s}\v{s}-} fientive built to ptcpl. \textit{{\hith}andant-} + \textit{par\=a}, which acts as a fixed phrase; a nonce formation here, with a few parallels only in NH; the participle normally has passive meaning---viz. `provided for' rather than `having provided'---but must be active here; for a semantic parallel, cf. \textit{sekkant-} `knowing; known' 

 \item[24 :] \textbf{EN.KARA\v{S}} `lord of the army (camp)'

\item[26 :] \textbf{UGU}\bit{=ya} `up, over; high' = Hitt. \textit{ser}

\item[27 :] \bit{tapar{\hith}a} \p{1s. pret. act.} `rule'; a fully-inflected Luwian verb form, with Luw. only 1\supersc{st} s. pret. ending \textit{-{\hith}a} (vs. Hitt. \textit{-{\hith}un}), integrated into a native Hittite composition; certain evidence for large-scale bilingualism in the NH period \index{Luwianism} 

\bit{peran=ma=at=mu} \p{postpos.} `before'; the postposition is separated from its governed encl. obj \textit{=mu} (viz. `before me') \index{prosodic domain}; this order can be generated only by the interaction of syntactic and phonological operations; I would assume that pronominal enclitics---here, \textit{mu} and \textit{at}---undergo obligatory syntactic movement to the highest possible position on the left-periphery, and get their position in the surface string by phonological movement, cliticizing to a viable prosodic host; in contrast, the surface position of \textit{=ma} is identical to its underlying position, which I would treat as the functional head of (a high) TopP; the cliticization of \textit{=ma} to \textit{peran} thus precedes the prosodic movement and cliticization of \textit{=mu} and \textit{=at} (the order of which with respect to one another is a difficult, but separate question) 

\textbf{\supersc{d}XXX-\supersc{d}U}\bit{-a\v{s}} \p{PN. com. nom. s.} `Arma-tar{\hith}unta'; a compound \index{compound} of two theonyms, the moon-god Arma (logogram identical to `30') and Tar{\hith}un

\item[29 :] \bit{kuit} \p{conj.} `because'; again, the conceit of H- is that he owes all his success to I\v{s}tar's providence \bit{kani\v{s}\v{s}an {\hith}arta} \p{3s. pret. act.} `recognize; acknowledge; esteem'; unclear whether true periphrastic perf. (`had recognized / given recognition') or stative (`held recognized / in esteem')

\item[30 :] \textbf{UK\`U{\mpl}}\bit{-annanz} \p{n. erg.; r/n-st.} `people' : \p{nom. s.} \textit{antu{\hith}satar-}

\item[32 :] \bit{ar\v{s}ani\=er} \p{3pl. pret. act.} `be jealous / envious of'; though marked with the Glossenkeil, the word seems to be native Hittite; on the basis of such examples, CM has argued that the Glossenkeil really functions like `sic', drawing the attention of a scribe/copyist to the fact that there is something odd about the word

\item[33 :] \bit{tam\=au\v{s}} \p{adj. c. nom. pl.} `other'; note the ending \textit{-u\v{s}} proper to the accusative here marks nominative case as elsewhere in NH

\item[34 :] \bit{ti\v{s}kiuwan} \p{sup.} `place'; functionally, it must be a supine built to the iterative stem of \textit{dai-} `place', since it combines idiomatically with \textit{uwai} `woe, distress, hardship'; formally, however, it is problematic, since it matches neither of the two attested iterative stems for this verb, \textit{zikke-} or \textit{zaske-}; the influence of \textit{figura etymologica} with the main verb may somehow be at work 

\bit{tiy\=er} \p{3pl. pret. act.} `place'; a rare form of the 3\supersc{rd} pl. preterite with zero-grade of the root; more frequent are \textit{d\=air} and \textit{dayer} (cf. \citetalias{GrHL} \S13.21) \bit{{\hith}\=uwappir} \p{3pl.pret. act.} `treat badly'

\item[35 :] \bit{arp\v{s}atta} \p{3s. pret. act.} `turn out badly'; a fully-inflected Luwian verb, always used impersonally

\item[36 :] \bit{ANA} \textbf{{\wood}UMBIN} \bit{lamniyat} `(lit.) he named (me) to the wheel'; an idiom of uncertain origin; all that is clear is that it is something \textit{bad} 

\textbf{\`U}\bit{-at} \p{3s. pret. act.} `dream; appear in a dream' = Hitt. \textit{za\v{s}{\hith}aniyat}
`
\item[38 :] \bit{tarna{\hith\hith}i} \p{1s. pres. act.} `release; deliver'; the latter sense frequently with \textit{par\=a}, as repeatedly in the following lines \bit{na{\hith}ti} \p{2s. pres. act.} `fear' ; the `personal' usage here coexists with the impersonal construction 3\supersc{rd} s. + dat., which may be older (though the argument is purely intuitive---essentially, it seems plausible to create a new `personal' usage in accordance with the dominant pattern, but unlikely for an impersonal to simply arise)

\item[39 :] \bit{parkuessun} \p{1s. pret. act.} `become pure'; fientive/inchoative \textit{parkui-} `pure' \textbf{DINGIR}\supersc{\bit{LIM}}\bit{-za} \p{c. abl.} `god'; a clear instrumentival use of the ablative in NH, viz. `by means of; by the agency of; through' \bit{kuit} \p{adv.} `because, since'; if DINGIR\supersc{LUM} GA\v{S}AN=YA is indeed to be taken as a constituent, the position of \textit{kuit} between noun and appositive is unexpected

\item[40 :] \bit{{\hith}uwappi} \p{adj. c. d-l. s.} `evil; bad' (cf. Eng. \textit{evil}); the purpose of the Glossenkeil is unclear here; the word is clearly not foreign, but it is conceivable (CM:) that it is archaic or substandard, hence: \textit{sic!}; contra AK, the spelling with extra \textit{-u-} is a purely orthographic innovation of NH, used to clarify [h\supersc{w}-]; it is not found in OH/MH 

\textbf{DI}\bit{-e\v{s}ni} \p{n. d-l. s.; r/n-st.} `law(-court); (legal) judgment'

\item[41 :] \bit{\=UL} `NEG'; the word order is of syntactic interest here, specifically the relative ordering of preverb, negative, and indefinite with respect to the verb, since all three elements `want' to be immediately prior to it in the unmarked case; for a similar pattern, cf. ln. 52-55

\item[42 :] \bit{\v{s}er wa{\hith}nut} \p{3s. pret. act.} `cause to turn; wave; brandish'; here \textit{ad sensum} 'allow to brandish over'; \textit{ser} in several cases seems to pattern syntactically with preverbs in this text, but can be interpreted here unproblematically as a normal postposition with \textit{=mu}

\item[43 :] \bit{=za} likely marking self-interest, viz. `to herself' \bit{{\hith}umandaz} \p{adj. n. abl.} `all'; here (and below) `in every respect' , another adverbial usage of the ablative

\item[44 :] \bit{i\v{s}tarkzi} \p{3s. pres. act.} `be(come) sick'; impersonal + acc. \bit{kuwapi} \p{adv.} `in any way'; the post-verbal position here strongly supports the idea that indefinite \textit{wh-}words may be weakly accented (cf. Gk. {\greektext t\'is} `who?', {\greektext tis} `someone'), which in Hittite results in clitic-like behavior 

\bit{=za} an important syntactic innovation of NH is that the `reflexive particle' \textit{=za} can fill an argument position---here, of \textit{ser} 

\bit{irmala\v{s}=pat} \p{adj. c. nom. s.} `ill, sick'; the \textit{-l-} in the suffix may be indicative of Luw. influence, as the Glossenkeil suggests (cf. Hitt. \textit{e/irman-} `sickness'); the \textit{=pat} is likely scalar `even'

\item[45 :] \bit{u\v{s}kinun} \p{1s. pret. act.} `see, behold'; iter. \textit{au(s)-} `see'

\item[47 :] \bit{par\=a {\hith}andanza} \p{ptcpl. c. nom. s.} `divinely-provided for; marked by divine providence' \textbf{UKU}\bit{-a\v{s}} \p{c. nom. s.} `man' $\sim$ UN = Hitt. \textit{antu{\hith}sa-}

\item[49 :] \bit{\v{S}A} \textbf{DUMU.NAM.L\'U.UL\`U\supersc{LU}}\bit{-UTTI} \p{n. gen. s.} `humanity'; must be interpreted as an objective genitive (as marked by \textit{\v{S}A}), `to; against' (vel sim.)

\item[51 :] \bit{\=e\v{s}ta} \p{3s. pret. act.} `be'; following O., the only real possibility seems to be a question directly addressed to the divinity \bit{kuwayami} \p{adj. n. d-l. s.} `feared; fearful'; the Glossenkeil marks a real Luwianism, the ptcpl. of a Luw. verb \textit{kwaya-} `fear' (e.g. KARATEPE {\S}XXXIII)

\item[52 :] \bit{=mu{\ldots}ser tiyat} \p{3s. pret. act.} `step{\ldots}over me'; the sense is clearly `(did not) step over and past me' = `(did not) ignore me', in which it is unique: \textit{CHD} s.v. \textit{\v{s}\=er} cites only this single passage with this meaning; CM `strongly feels the need' for \textit{ar{\hith}a}

\item[53 :] \bit{peran katta{\ldots} tarna\v{s}} \p{3s. pret. act.} `deliver; turn over'; though \textit{peran} is formally a postposition, the sense here speaks in favor of a fusion with \textit{katta} so that both behave as preverbs; the same holds for the sequence in ln. 55 immediately below; for a similar fusion, cf. the well-established idiomatic expression \textit{kattan ar{\hith}a zinnai-} in (e.g.) ln. 60, below

\item[54 :] \bit{EN DINI=YA} \p{d-l. s.} `opponent-at-law (lit. `lord of the judgement')'; \textbf{\men}\bit{ar\v{s}anatalla\v{s}} \p{c. d-l. pl.} `begrudger; hater'; agent noun, probably deverbal to attested stative \textit{ar\v{s}an\=e-} `be envious', but a denominal derivation is also possible

\item[55 :] \textbf{INIM} = Hitt. \textit{memiya\v{s}} `word; matter' here (not \textit{uttar}!), which explains common gender agreement with clitic pronoun \textit{=a\v{s}}---hence `(whether it is) a matter of \ldots or \ldots'

\item[57 :] \bit{pal(a){\hith}\v{s}an} \p{c. acc. s.} `(a garment characteristic of I\v{s}tar)'; likely Luwian; note that the word may reflect an inherited \textit{*s-}stem, which has been thematized in Luwian (cf. Luw. \textit{u\v{s}\v{s}a-} `year' (Hitt. \textit{witt-}), on which see \citet{vine2009yearly} \textbf{UGU} = Hitt. \textit{\v{s}er} `above'

\item[60 :] \bit{ar{\hith}a zenna{\hith\hith}un} \p{1s. pret. act.} `completely finish off'; an idiomatic combination

\item[62 :] \bit{\=a\v{s}ta} \p{3s. pret. act.} `remain'; a non-ablauting \textit{a} vowel in the root, esp. of a \textit{mi-}verb is problematic; the origin of this verb is mysterious, for bad ideas see AK and Puhvel

\item[66 :] \bit{par\=a wi\v{s}kit} \p{3s. pret. act.} `send forth'; iter. \textit{wiya-} `send', which does \textit{not} contain the preverb \textit{u-}, and never has the sense `hither'; see Melchert (forthcoming, IXth Hittitology Congress)

%\item[70 :] \bit{kani\v{s}\v{s}uwar=ma=mu \v{S}A{\ldots}} CM jokingly suggests that this should be the title of the text; it is, at the very least, the dominant theme

\item[72 :] \bit{wiyanun} `drive'; cannot be \textit{wiya-} `send', which would imply that H- has control, evidently not the case here with \supersc{L\'U}KUR `enemy'; nor containing \textit{-u-} `hither' deictic particle, on which see Melchert (forthcoming, IXth Hittitology Congress)

\item[73 :] \bit{kuitman=ma=za} \p{conj.} `while'; introduces a clause which (with O.) is subordinate to the \underline{relative} clause that follows

\item[74 :] \bit{{\hith}ant\I} \p{adv.} `separately'; viz. onto another tablet, since H- has filled this one with his complements; the precise syntax is unclear in absence of a case-ending on \textit{TUPPU} \textbf{D\`U}\bit{-mi} = Hitt. \textit{iy\'ami}

\item[75 :] \bit{AMAT} = Hitt. \textit{uttar}

\item[76 :] \bit{\v{S}APLITI} `lower' + KUR = `Lower Land'; a technical designation for the lands to the southwest off the high plateau on which Hattusa is situated

\smallskip
\item[II]
\smallskip

\item[1 :] \textbf{GIDIM\hpl}\bit{=ya} `spirits of the dead; ancestral totems; \textit{Manes}'

\item[2 :] \textbf{DINGIR U}\bit{-tassa} `Tarhuntassa, the city of the god' is now to be restored in the break {\bf EGIR}\bit{-az} = either \textit{apezziyaz} \p{adv.} `afterward' or \textit{appa=z(a)} \p{pp.} `behind his back'

\item[4 :] \textbf{BAL} `outrage; revolt, rebellion' \bit{iy\=at} \p{3s.pret.act.} `do, make'; agreement with the nearer

\item[5 :] \textbf{B\`AD} `wall, fortification'; in conjunction with URU = `fortified cities' \textbf{\supersc{\'ID}}\bit{Mara\v{s}\v{s}andan} \p{c. acc. s.} `river M-'; the classical Halys (modern Kizilirmak), it  separates the Hittites from the land of Gasga along the coast of the Black Sea, a perennial threat to Hattusa

\item[9 :] \bit{pedi} \p{n.loc.s.} `place'; in OH, we find rather \textit{pedi=\v{s}\v{s}i} `in its place' 

\textbf{URU DU\subsc{6}\hpl} `ruined city (lit. city-ruin)'; what exactly this means in context is not clear

\item[10 :] \bit{=ma} \p{ptcl.} `and; but'; a nice example here where the clitic is sense-diagnostic, as it follows the sequence of logograms which form a single DP

\item[11 :] \bit{wal{\hith}anniskiwan} \p{sup.} `strike'; a doubly marked iter./impfv., characterized by both \textit{-anni-} and \textit{-ske-}

\item[12 :] \bit{dannattan} \p{ptcpl. n. nom. s.} `depopulated; deserted'; adj. \textit{dannant-} `empty', denom. vb. \textit{dannattai-} `be empty' 

\item[13 :] \bit{\=ara\v{s}kit} \p{3s. pret. act.} `arrive; come to'; iter. \textit{ar-}

\item[14 :] \bit{i\v{s}parzir} \p{3pl. pret. act.} `escape' $<$ \textit{PIE *sper{\dh}-} `run'; cf. Skt. \textit{spardh-} `contend', Arm. \textit{sprdem} `escape'; the stem originally had a final dental, which resulted in an affricate from the {\sc Double-Dental Rule} and {\sc Hittite Assibilation}; alternations are found in OH/MH, with generalization of the affricate in NH {\bf II}\bit{-\=el} \p{gen.} `2'; the genitive ending here is nonsensical; for discussion, see \citet{eichner1992numerals}

\item[15 :] \bit{=\v{s}ma\v{s}} \p{encl.3pl.dat.} `for them(selves)', with reflexive sense here

\item[16 :] \textbf{NUMUN\ldots}\bit{annie\v{s}kir} \p{3pl. pret. act.} `cultivate{\ldots}seed'; per O, NUMUN used here in its literal agricultural sense rather than more common metaphorical `progeny' 

\bit{par\=a} \p{adv.} `furthermore; moreover'

\item[20 :] \bit{d\=ai\v{s}} \p{3s. pret. act.} `place'; an intransitive use of \textit{d\=ai--} `place' does not work in context; the sense seems to be `laid siege' (vel sim.) with implicit subject `the enemy'

\item[21 :] \bit{tepawaz} \p{adv.} `in small numbers'; a rare adverbial use of the ablative case \citep[cf.][]{melchert1977diss}

\item[22 :] \textbf{\troops} \bit{NAR\=ARU} `auxiliary troops'

\item[24 :] \textbf{GA\v{S}AN GA\v{S}AN}\bit{=YA} \p{c. voc. s.} `lady my lady'; expressive doubling in address to H-'s personal deity I\v{s}tar

\item[25 :] \bit{{\hith}ulliyanun} \p{1s. pret. act.} `fight'; built to the secondary denominative \textit{*--y\'e/\'o--} stem 

\textbf{\v{S}U.AN} `monument'; semantics are \textit{ad sensum}, not a transparent combination of known logograms

 \bit{weda{\hith\hith}un} \p{1s. pret. act.} `build'; originally a \textit{mi-}verb like \textit{pede--}, \textit{wede--}, but becomes hopelessly confused with \textit{widai--} `bring', leading to \textit{{\hith}i}-verb forms

\item[26 :] \bit{p\=e {\hith}arta} \p{3s. pret. act.} `have (with); hold (in possession); bring'; cf. \textit{CHD} s.v. (1)

\item[28 :] \textbf{L\'U{\mpl}}\bit{{\hith}uiyatallu\v{s}=ma} `leaders' (lit. `fore-runner men'); the clitic here after syllabic spelling could indicate some close prosodic unity \index{prosodic domain}---formation of a (small) prosodic domain or perhaps even a real compound; alternatively, read initial L\'U as a determinative rather than a logogram

\item[29 :] \bit{{\hith}inkun} \p{1s. pret. act.} `offer; give; entrust' \textbf{IGI}\bit{-zi} \p{adj. n. nom. s.} `first' = Hitt. \textit{{\hith}antezzi} \textbf{L\'U}\bit{-tar=mit} \p{n. nom. s.} `(my) manly deed; (display of) manliness'; the correct use of the possessive clitic surely indicates a fixed collocation; however, the superfluous use of \textit{=mu} would be unexpected in OH

\item[30 :] \bit{pal\v{s}i} \p{c. d-l. s.} `path; way; (military) campaign; occasion' + IGI-\textit{zi} = \p{adv.} `for the first time'

\item[31 :] \bit{wit=ma{\ldots}anda ara\v{s}} an example of QSV \index{quasi-serial verb construction}, with an interesting use of \textit{=ma}, which could conceivably provide evidence for a topicalization analysis of restructuring; for \citet{koller2013restructuring}, this is an example of raising to subject with \textit{uwa--}; \textit{anda ara\v{s}--} is idiomatic for `show up'

\item[32 :] {\bf \v{S}\`A {\man}K\'UR} `(lit.) within the enemy', which makes no sense without KUR `land' (vel sim.), and is perhaps therefore an error \textbf{ZAG-a\v{s}} \p{c. nom. s.} `border'; the edition of \citet[12]{otten1981hattusili} omits significant text here; lines 16--17 should read: \textit{nu=\v{s}\v{s}i ap\=ez} KUR {\city}\textit{Taqqa\v{s}ta\v{s}} ZAG\textit{-a\v{s} \=e\v{s}ta k\=ez=ma=\v{s}\v{s}i} {\city}\textit{Talmaliya\v{s}} ZAG\textit{-a\v{s} \=e\v{s}ta} `Its border on that side was the city of Taqqasta, and its border on this side was the city of Talmaliya.'

\item[34 :] \bit{\textsubdot{S}\'IMTUM} `bond; yoked pair, team'

\item[35 :] \bit{kappuwawar} \p{n. nom. s.; r/n-st.} `reckoning; (ac)counting' : \textit{kapuwai-} `count' 

\item[37 :] \textbf{L\'U}\bit{=ya} \p{c. nom. s.} `man'; a (fairly rare) concessive use of \textit{=a/ya} (cf. \citetalias{GrHL} \S29.44) \bit{apiya=ya} \p{adv.} `then; at that time'

\item[39 :] \bit{I\v{S}TU} \textbf{N\'I.TE}\bit{=YA} a fixed phrase, often translated `with my might' (which could speak to an underlying \textit{tuekkaz=mit} \p{c. gen. s.} `body; strength'), but the idiomatic sense is perhaps more accurately `personally' per CM

\bit{kui\v{s}} \p{c. nom. s.} `who(ever)'; semantics support `indeterminate' reading, though the \textit{wh}-word is non-initial \index{indeterminate relative clauses} (!); the relative clause is itself subordinate to the subordinate \textit{ma{\hith}{\hith}an} clause  

\bit{peran {\hith}\=uyanza} \p{c. nom. s.} `fore-runner; leader'

\item[40 :] \bit{anda pennu-} \p{1s.(?) pret. act.} `drive'; the \textit{-u-}spelling immediately before the break allows for only infinitive or 1pl. on formal grounds, and neither makes sense in context; moreover, if indeed transitive \textit{penna--}, the absence of an object clitic is very problematic; this entire clause is simply omitted in B

\item[41 :] \bit{piddai\v{s}} \p{3s.pret.act} `flee'; the presence of the reflexive \textit{=za} in this clause is puzzling

\item[46 :] \bit{{\hith}ali\v{s}\v{s}iyanun} \p{1s. pret. act.} `adorn; emblazon; have plated (with silver, gold, etc.)'; denom. \textit{{\hith}ali\v{s}\v{s}a-} `border, edge; ornament'
 
 \item[50 :] \bit{maninkuwann=a=a\v{s}=mu} \p{adv.} `near' : \textit{maninkuwant-} \p{adj.} `short; low; narrow' \textbf{\troops}\bit{=ya=za} the reflexive particle \textit{=za} is the object of the postposition \textit{peran}
 
 \item[52 :] \bit{pidi} \p{n.d-l.s.} `place'; dat-loc. used like abl.: lit. `on the spot' = `from the spot'
 
 \bit{ninikta} \p{3s. pret. act.} `raise up; set in motion; move' 
 
 \item[56 :] \bit{A\v{S}\v{S}UM} \p{prep.} `for the sake of; on account of; regarding' \bit{MUIRTUM} `administration; management'
 
 \item[60 :] \textbf{KUR \'ID.S\'IG} `land (of) the wool (\textit{{\hith}ulana-})-river'; rebus spelling
 
 \item[63 :] \bit{kue k\I} `As for{\ldots}these (lands) which{\ldots}'; {\color{red}is this an embedded relative? ask CM}
 
 \item[65 :] \bit{kui\=e\v{s}} \p{rel. pron. c. acc. pl.} `who; which'; in NH, there is a partial merger of nominative and accusative in pronouns, so that it is possible to use the \textit{-e\v{s}} ending characteristic of the nominative to mark accusative case (note, though, that second indefinite is rightly nominative, and we may speculate that some kind of anticipatory case-attraction may be partially responsible in this case)
 
\item[66 :] \textbf{GAM}\bit{-an} \p{postpos.} `beside; down' = \textit{kattan}; pp. with object \textit{=mu}

\item[68 :] \bit{n=at} \p{encl. pron. n. acc. pl} `them'; double object construction, lit. `made them back into Hattusa' = `made them (part of) Hattusa again'

\item[69 :] \bit{wit{\ldots}pait} \p{3s.pret.act} a possible example of QSV \index{quasi-serial verb construction}, but more likely, a raising to subject construction with \textit{uwa--} \citep[cf.][]{koller2013restructuring}, hence rightly `it came about that{\ldots}he went'

\bit{kuwapi} `at some time/point'; bare \textit{wh}-word used as indefinite

\item[71 :] \bit{k\=el} \p{pron. gen.} `this'; flagged by Otten, the pronoun is formally singular (and agrees with apparently singular KUR\supersc{\textit{TI}}), but clearly semantically plural

\item[76 :] \bit{{\hith}aptat} \p{3s. pret. mp.} `attach, join; \p{(mp.)} succeed, work out'; mediopassive frequently used impersonally 

\bit{\=UL ku\=ezka} `not in any respect whatsoever' 

\bit{kuit} \p{conj.} `because'

\item[77 :] \bit{alwanza{\hith\hith}\=wuanzi} \p{inf.} `ensorcer, bewitch, enchant' + \textit{eppir} = periphrastic `begin to'; the finite verb shows plural agreement \textit{ad sensum}

\item[78 :] \bit{alwanze\v{s}naz} \p{n. abl.} `sorcery'; abl. for instr.

\item[79 :] \bit{s\=er \v{s}unni\v{s}ta} \p{3s. pret. act.} `fill up'

\item[80 :] \bit{iya{\hith}{\hith}a{\hith}at} \p{1s.pret.mp.} `walk; go'; \textit{iya--} \p{mp.} + goal is an innovation of NH

\item[81 :] \textbf{BAL}\bit{-uwanzi} \p{inf.} `libate' = Hitt. \textit{i\v{s}pand-}

\smallskip
\item[III]
\smallskip

\item[1 :] {\bf \supersc{f}}\bit{Puduhepan} \p{PN.c.acc.s.} `P-'; her name and her mother's name (P-) are Hurrian

\item[2 :] \bit{{\hith}and\=awen} \p{1pl. pret. act.} `arrange; \p{intrans.} fit together; be compatible; get along well'; intrans. usage is already common in OH, and recall that the non-passive sense of the MP is recessive in NH

\item[3 :] \textbf{\man}\bit{MUDI} `man; husband' \bit{\=a\v{s}\v{s}iyatar} \p{n. acc. s.} `love, affection'

\item[5 :] \textbf{\`IR}\bit{-a{\hith\hith}ut} \p{2s. impv. mp.} `serve' \bit{QADU} \p{prep.} `(together) with'

\item[7-8 :] \bit{par\=a iyanni\v{s}} \p{3s. pret . act.} `go forth; make progress' \bit{=at} \p{pron.encl.n.acc.s.} `that'; a rather free usage, with the entire preceding clause as antecedent

\item[10 :] \bit{kururiya{\hith}ta} \p{3s. pret. act.} `make into / treat as an enemy'; the expected factitive sense is not found here, though, where it seems to mean `become inimical'

\item[11 :] \bit{uiyanun} \p{1s.pret.act} `drive (off/out)'; here the deixis is conclusively centrifugal, i.e. `off, out, away', hence strong evidence for CM against an etymology with \textit{u--} `hither' \textbf{SIG\subsc{5}}\bit{-a{\hith\hith}un} \p{1s. pret. act.} `make good; restore order'

\item[13 :] \bit{ki\v{s}tat} \p{2s. (?) pret. mp.} `become'; 2s. address to Ishtar

\item[14 :] \bit{kuitki} \p{n.nom./acc.s.} `something'; either indefinite modifier of \textit{{\hith}ane\v{s}\v{s}ar} or adv. `in some way; somehow'

\item[15 :] \bit{{\hith}uittiyattat} \p{3s. pret. mp.} `draw, pull; bring' 

\item[16 :] \bit{tikku\v{s}\v{s}anut} \p{3s. pret. act.} `show, display; reveal' \bit{{\hith}anni\v{s}nanza} \p{n. abl. s.; r/n-st.} `judgement; indictment'

\item[18--19 :] \bit{n=at=\v{s}(\v{s})i=at} CM following Rieken attributes the doubling of the pron. encl. \textit{=at} to Luwian interference; the fact that Luwian shows the inverse order of dat. and acc. pron. encl. is the likely source of the confusion, as a Luwian-speaking scribe recopied the Hitt. text with historically correct ordering; the ``extra'' \textit{YA} sign is likely only [y] as hiatus-breaker 

\bit{piran katta tiyir} \p{3pl.pret.act.} `(lit.) place down before (him)' = `confront (him)'

\textbf{U{\Hith}\subsc{7}}\bit{-za} \p{n. abl. s.} `sorcery' = Hitt. \textit{alwanzannaz}; abl. for instr.

\item[20 :] \bit{katerra{\hith}ta} \p{3s. pret. act.} `make low; humble'; in a legal context, a technical term for `cause to lose the legal judgement/case to'

\item[21 :] \bit{peran nai\v{s}} \p{3s.pret.act.} `(lit.) turn before' = `turn over to'

\item[22 :] \bit{anda} + (ellipsed) copula = `be involved'

\item[23 :] \bit{\v{s}arazziya{\hith}ta} \p{3s. pret. act.} `make above; cause to prevail'; factitive \textit{sarazziya-} `above; upper'; special legal sense, opp. to \textit{katerra{\hith}{\hith}--} `above', i.e. `cause to win the legal'; the text here, however, is problematic: \textit{CHD} (s.v. \textit{\v{s}arazziya{\hith}{\hith}--}) reads: \textit{nu=mu} \v{S}E\v{S}\textit{=YA kuit}[ \textit{ANA} \supersc{md}\textit{S\^IN}]-\supersc{d}IM DINGIR[\textit{-LIM-it}] DI\textit{-e\v{s}naz \v{s}ar\=aziya{\hith}t}[(\textit{a})] `Because my brother [with (the help of)] the deity ruled in my favor over Arma-Tarhunta in a legal case{\ldots}', which maintains the parallelism (in case-frame) with its semantic opposite \textit{katerra{\hith}{\hith}--}

\item[24 :] \bit{mau\v{s}{\hith}a{\hith}at} \p{1s. pret. mp.} `fall'; idiomatically, `fall victim (\p[{[dat.]}to)' with \textit{CHD} (loc. cit.)

\item[26 :] \bit{uwayattat} `inflict harm; arouse pity'; denom. \textit{uwai-} `woe'; in mp. with dat. clit. \textit{=mu} `was a source of pity (to me)'

\item[28 :] \textbf{D\`U}\bit{-nun} \p{1s. pret. act.} `make' = Hitt. \textit{iyanun} \bit{A\v{S}AT} `wife' \textbf{\city}\bit{Ala\v{s}iya} \p{c.d-l.s.} `Alasiya' = Cyprus

\item[29 :] \bit{uppa{\hith\hith}un} \p{1s. pret. act.} `send' \bit{tak\v{s}an} \p{n. loc. s.} `middle'; in endingless form, regularly collocates with \textit{sarra-} \p{c.} `portion; share' = `half; half-share' 

\item[39{\pr} :] \bit{{\hith}anda\v{s}} \p{postpos.} `in accordance with; as per'; the essential point of this periphrasis is that Hattusili did not attempt to usurp the throne

\bit{manqa} \p{adv.} `in some/any way'; indefinite formed to interr. stem. of \textit{m\=an} `if'; for the indefinitizing ptcl., cf. \textit{kuwatka}

\item[40{\pr} :] \bit{{\hith}ui{\hith}u\v{s}\v{s}uwali\v{s}} `legitimate, rightful'; perhaps a Luwianism, but the meaning is any case unclear; the initial $<$\textit{sa-}$>$ ready by \citet[20]{otten1981hattusili} is an error

\item[41{\pr} :] \bit{E\v{S}ERTI} `secondary wife; concubine' 

\item[47{\pr} :] \bit{\=UM} `day(s)' \bit{{\Hith}antili} \p{c.gen.s.} `H-'; a king of the OH period after the reign of Telepinu 

\item[51{\pr} :] \bit{=at} restore instead \textit{=a\v{s}}

\item[53{\pr} :] \bit{dama\v{s}\v{s}an} a passive use of the stative {\hith}ark- construction? (lit. `held subdued' = `remained subdued'); the two city-names are clearly marked nominative; ah, but maybe another case of \textit{=za} as an argument?

\item[55 :] \bit{ar\v{s}aniyat} \p{3s. pret. act.} `become jealous/envious'

\item[56 :] \textit{uw\=ai} \p{n.acc.s.} `woe'; the idiom is usually with \textit{d\=ai--} `place' = `make woe (for)', but the duplicate seems to allow only for a restoration \textit{d\=a\v{s}} with $<$\textit{da-}$>$ in the break; the precise sense is thus unclear

\item[62 :] \bit{da{\hith}u\v{s}iya{\hith\hith}a} \p{1s. pret. act.} `comply; submit'; fully-inflected Luw. verb; etymologically, probably {*}`be quiet', cf. Skt. \textit{t\=u\.{s}ni--} `silence'

%---------------- resume above 9/30

\item[63 :] \bit{{\hith}arkanna} \p{inf.} `perish'; we might expect \textit{=mu{\ldots}san{\hith}-- {\hith}arkninkuwanzi} `sought to destroy me', but the infinitive is built to the intrans. stem \textit{{\hith}ark--} `die'; hence the interpretation `sought (that) I perish' may be the best possibility, but \textit{san{\hith}--} is elsewhere a control verb

\item[66 :] \bit{kururiya{\hith}{\hith}un} \p{1s.pret.act.} again, the expected trans. sense `make hostile' doesn't seem to fit the context; note interclausal head--tail linking, with \textit{=ma} marking contrastive focus; \textit{ap\=at} \p{pron.n.acc.s.} `that'; cataphoric, i.e. `that impropriety, (namely){\ldots}'

\item[67 :] \textbf{{\wood}GIGIR} `chariot'

\item[69 :] \textbf{L\'U}\bit{-nili=\v{s}\v{s}i} `in a manly way' (= \textit{pis(e)nili}); for adv. in \textit{-ili}, see \citetalias{GrHL} \S19.15  

\bit{watarna{\hith\hith}un} \p{1s. pret. act.} `order, command; inform, notify'

\item[70 :] \bit{\v{s}ulliyat} \p{2s. pret. act.} `become arrogant / insolent' < {*}`become swollen'; cf. Eng. \textit{swell} \bit{{\Hith}AL\textsubdot{S}\'I} `fortress, stronghold' 

\item[73 :] \textit{eni\v{s}\v{s}an} \p{adv.} `in that way'; anaphoric \bit{ki\v{s}an} \p{adv.} `thus'; weird spelling? the following indef. \textit{kui\v{s}} looks like cliticization

\item[74 :] \bit{anni\v{s}an} \p{adv.} `previously, formerly' \bit{kuwat} \p{adv.} `why'; a nice example of a direct question where \textit{wh}-word clearly remains low \index{wh-syntax}

\item[76 :] \bit{m\=an} \p{ptcl.} `(irrealis)'; supply here `I would reply' (vel sim.)

\item[77 :] \bit{{\hith}and\=an} \p{adv.} `truly; really'; in a rhetorical question here

\smallskip
\item[IV]
\smallskip

\item[2 :] \bit{par\v{s}ta} \p{3s. pret. act.} `flee; take off; race from'

\item[4 :] \bit{weriyat} \p{3s.pret.act.} `call, summon'; with obj. \textit{=an} and infinitive complement \textit{ni[ninkuwanzi]} `mobilize', the likeliest restoration from $<$\textit{--ni-in-ku-wa-an-zi}$>$ found in a dupl., which \citet[22]{otten1981hattusili} fails to note; another possibility is \textit{{\hith}ar[ninkuwanzi]} `destroy'

\item[6 :] \bit{mar{\hith}ta} \p{3s. pret. mp.} `?'; the verb is unaccusative; IY has now suggested a connection with Pal. \textit{mar{\hith}--}, but this does little to clarify the meaning

\item[7 :] \textbf{LUGAL}\bit{-UTTA} `kingship' %\textit{-UTTA-} as abstract noun suffix?

\item[10 :] \bit{huwiyami} \p{1s. pres. act.} `run; flee'; marked with Glossenkeil, but clearly Hitt. verb with Hitt. verbal morphology; the $<$ -u- $>$ typically marks a lengthened [o\textlengthmark], but leftmost ictus is unexpected here

\item[11 :] \bit{I\v{S}TU \v{S}A} \p{pp.} `to the side of' 

\bit{neyari} \p{3s.pres.mp.} `turn'; intrans. etymological sense here

\bit{sallanun} \p{1s.pret.act.} `raise, bring up; exalt, magnify'; denom. $<$ \textit{salli--} `big, large'

\item[13 :] \bit{\=UL} \p{NEG} redundant `NEG' may be a scribal error

\item[14 :] \bit{kinun(a)=ya=war=an} scribal error for expected geminating \textit{=a} 

\bit{karapmi} \p{1s. pres. act.} `lift; pick up; take away' ($<$ \textit{*kerp--}; cf. Lat. \textit{carp\=o}); the geminate \textit{--pp--} contrasts with singleton \textit{--p--} found in the {\hiverb} \textit{k(a)rap--} `grab; seize'

\textbf{{\city}TUL}\bit{-na} `Arinna'

\item[15 :] \bit{A\v{S}\v{S}UM} \p{prep.} `for the sake of'

\item[16 :] \bit{=mu} \p{pron.encl.1s.acc.s.} `me'; appositional with acc. obj. `Ishtar' \bit{para\v{s}\v{s}in} `?'; meaning unknown, hapax here

\item[17 :] \textbf{GIM}\bit{-an} \p{conj.} `if; when' (= \textit{ma{\hith}{\hith}an}); a unique but apparently functionally equivalent way of conveying the semantics associated with `indeterminate' relative clauses

\bit{=ya} \p{conj.} `and';  phonologically unexpected \textit{=ya} after [--C\#] is very rare, and leads one to wonder about its usage in the context of such an odd syntactic construction

\item[21 :] \bit{dariyante\v{s} innarawa} `(you are) mightily exhausted'; ptcpl. to {\hiverb} \textit{tariya--} `be(come) tired'; the \textit{WA} on adv. \textit{innara} is unexpected

\item[24 :] \bit{kuwapi} \p{loc. s.} `which'; syntactically a subordinate relative clause, but difficult to render in English \bit{tameda} \p{adj. all. s.} `another'; with \textit{tarnae--} `release/allow to go to another place'

\item[26 :] \textbf{GIM}\bit{-an} = \textit{ma{\hith\hith}an} `like, as'; for this `comparative' usage, cf. \citetalias{GrHL} \S15.8 

\bit{{\hith}umma} `(pig-)sty'

\textbf{EGIR}\bit{-pa} \p{prv./adv.} `back; again'; reinforcing with \textit{i\v{s}tapp--} or possibly `in turn' (cf. \textit{appa mem\=ai--} `speak in turn; reply'

\item[28 :] \textbf{GAM}\bit{-an} \p{postp.} `with, beside; against'

\item[31 :] \textbf{\man}\bit{LI} `prisoner, captive; hostage'

\item[32 :] \textbf{B\`AD} `wall, fortification' + URU = `fortified city' \bit{ADDIN} `I gave' (= \textit{pe{\hith\hith}un})

\item[34 :] \bit{kupiyatin kupta} \p{3s. pret. act.} `planned a plan/conspiracy'; figura etymologica, with \textit{--ti--} nominal formed to derive \textit{*--ye/o--} stem; the non-affricated \textit{--ti--} is diagnostic as a real Luwianism

\item[37 :] \textbf{ZAG}\bit{ zainuir} \p{3pl. pret. act.} `cause to cross the border' = `deport'; per \citet{luraghi2012valency}, this verb is the single example of trans.-caus.\textit{--nu--} increasing the valency of an already transitive verb---here, \textit{zai-} `cross (+ X\p{acc.})'---to form a double object construction

\item[39 :] \bit{ilani ilani} \p{c. d-l. s.} `step, degree'; \textit{\=amre\textsubdot{d}ita}-like structure \index{IWJ} = `step by step; gradually'; note obligatory \textit{--ske--} formation

\item[44 :] \textbf{\men}\bit{{\hith}arpanalliu\v{s}} \p{c. acc. pl.} `opponent; foe' < *`betrayer' from \textit{*{\hiii}or{\bh}--} `move from one domain into another; change sides' (cf. Lat. \textit{orbis}) 

\item[46 :] \textbf{UD}\bit{-azza} \p{n. abl.} `day' (= \textit{siwaz}); ellipsis for `day of the mother' = `appointed day of death'---hence in context `naturally' 

\item[47 :] \bit{ar{\hith}a zena{\hith\hith}un} `finish off'

\item[48 :] \bit{=mu} \p{encl. 1s. acc. s.} `me'; appositional/clitic doubling with DUMU.LUGAL \bit{nu} \p{conj.} this usage is unexpected, since there is no prosecutive function; per CM, then, `bad Hittite prose' (and cf. aberrant \textit{anda tarnai--} in ln. 49 immediately below

\item[49 :] \bit{anda tarni\v{s}ta} \p{3s.pret.act} `release/let go into'; a somewhat odd usage of this preverb + verb combination

\item[50 :] \bit{MA{\Hith}RU=YA} `earlier, prior; senior' \bit{a\v{s}\v{s}awa\v{s} memiana\v{s}} `(lit.) of good matter' = `on good terms with'

\item[51 :] \bit{ki\v{s}antat} \p{3pl.pret.mp.} `become'; we expect rather \textit{e\v{s}er} `remained'; cf. \citet{vandenhout1997hattusili}

\item[52 :] \textbf{{\men}TEME} `messengers' \bit{wi\v{s}kiuwan} \bit{supine} `send'; iter. \textit{wiya-} `send' \bit{upe\v{s}\v{s}ar}\textbf{\hpl}\bit{=ma=mu} \p{n. acc. s.; r/n-st.} `sending; gift'; note figura etymologica with  \textit{uppa--} `send'

\item[54 :] \bit{ABBA}\textbf{\hpl} `father; ancestor'

\item[55 :] \bit{n\=a{\hith}uwa\v{s}} \p{c.gen.s.; a-st.} `fear; respect'; free-standing genitive with deontic sense `should be of respect [= respectful]'

\item[57 :] \textbf{ZAG\hpl}\bit{ANA}\textbf{ZAG\hpl} `boundary upon boundary' 

\item[60 :] \bit{ki\v{s}an} \p{adv.} `thus; in this/that way'; a relatively rare anaphoric usage

\item[62 :] \textbf{\supersc{d}KAL}\bit{-an} \p{c. acc. s.} `tutelary deity' ( = \supersc{d}LAMMA = Kuruntiya (PN))

\item[63 :] \bit{A\v{S}RU} `place'; sense and syntax are difficult here; \textit{CHD} trans. `and (in) Tarhunatassa, the place which{\ldots}', treating \textit{A\v{S}RU kuit} as appositive to acc. T-'; CM suggests either `the place which he built up (as) T-' \citep[following[29]{otten1981hattusili} or `the place which he made into the property of the royal house, (namely) T-' (following \textit{CHD}).

\bit{parnawai\v{s}kit} \p{3s. pret. act.} either `make into property of the royal house' or `build up'; evidently denominative to obl. \textit{parn-} `house', though the \textit{--u--}element is formally unclear.

\item[64 :] \bit{ma\v{s}iwan} \p{adv.} `as much; as many'; exclamatory here `O lady I-, how many times{\ldots}!'; cf. CHD \textit{s.v.} (2)

\item[67 :] \bit{appan tarna{\hith\hith}un} \p{1s. pret. act.} `bestow (in perpetuity)'; for this meaning, see CM's review of \citet{otten1988bronze}

\item[68 :] \bit{annallan} \p{adj. n. acc. s./adv.} `previous; former'; as adv. `already'

\item[69 :] \bit{apadd=a=ya} a hypercharacterized syntagm with double \textit{=(y)a}

\item[72 :] \textbf{\supersc{NA\subsc{4}}ZI.KIN} `cult-stone; statue of a deity'

\item[73 :] \bit{i\v{s}{\hith}ui\v{s}kanzi} \p{3pl.pres.act.} `pour; scatter'; iter. to \textit{i\v{s}h(u)wai--} 

\item[74 :] \bit{nu=\v{s}ma\v{s}=an} what is this last \textit{=an}? \bit{\v{s}arlaimmin} `exalted'; Luwian verbal adjective

\item[79 :] \textbf{\'E-ir=ma kuit} appears to be relative DP obj. of \textit{pe{\hith\hith}un} in RC, but if so, difficult to connect syntactically or semantically to the main clause; \citet[28]{otten1981hattusili} takes as locatival, but this is formally problematic

\item[80 :] \bit{karnan marnan} `\textit{k} (and) \textit{m}'; this is utterly opaque

\item[82 :] \textbf{\'IR}\bit{-anni} \p{d-l.s.} `service; servitude'; contra \citet[30--1]{otten1981hattusili}, must be dat./loc. of place from which

\item[83 :] \bit{ezzan} \textbf{GIS}\bit{-ru} \p{n. acc. s.} `straw (and) wood'; idiom. w/ GIS{\it-ru} = `a (little) bit' \bit{ilaliyazi} \p{3s. pres. act.} `desire; covet'

\item[85 :] \bit{sa{\hith\hith}an=(i)ya=(a)\v{s}} `land rent (and) corv\'ee'; regularly collocated together (cf. Puhvel in \textit{HED}); per CM, this line probably ought to follow line 80, otherwise the sense---and esp. the referent of the pronominal clitic---is extremely difficult to explain.

\item[86 :] \textbf{\v{S}\`A.BAL.BAL} `progeny; descendant'

\item[87 :] \bit{ziladuwa} \p{adv.} `in the future'; its repetition here is likely a scribal error (dittography)

\item[88 :] \bit{sar\=a i\v{s}parzazi} \p{3s. pres. act.} `ascend (the throne) (?)'; the simplex verb means `escape', and with prv. \textit{sar\=a} elsewhere seems to mean `show up'; but here the context dictates a basic sense `ascend the throne' (vel sim.)

\end{description}

\section{The Horse-Training Texts}

\begin{description}
\item[Publication:] [add]
\item[Edition:] \citet{kammenhuber1961horse}
\item[Background:] [add] \\

\subsection{The `Kikkuli' Text}

\item[1 :] \textbf{{\man}}\bit{\=a\v{s}\v{s}u\v{s}\v{s}anni} \p{c.nom.s.} `horse-trainer'; perhaps an Indic borrowing reflecting inherited `horse' word; cf. HLuw. \textit{\'asu-} `horse'

\item[3 :] \bit{z\=eni} \p{c. d-l. s.; a-st.} `fall; autumn' \bit{uzu{\hith}rit\=i} `grass'; case?

\item[4 :] \textbf{DANNA} `mile' (?)

\item[5 :] \textbf{IKU{\hpl}} `?'

\item[6 :] \bit{l\=ai} \p{3s.pres.act.} `loose; release'

\item[7 :] \bit{a\v{s}nuzi} \p{3s.pres.act.} `prepare; take care of, attend to'; \bit{sakruanzi} \p{3pl.pres.act.} `water; drink (+\textit{=za})'; shift to plural seems odd, but the encl. pron. must be object

\item[8 :] \textbf{I\v{S}} `charioteer'

\item[9 :] \bit{UPNA} `handful' \bit{kanza} `(a type of grain)\textbf{\v{S}E} ` ' \textbf{{\Hith}AD.DU.A} \p{ppp. c. acc. s.} `dry' = Hitt. \textit{{\hith}atantan}

\item[10 :] \bit{immiyandan} \p{ppp. c. acc. s.} `mix'

\item[11 :] \textbf{\v{S}\`A.GAL} `food; fodder'

\item[12 :] \bit{{\wood}KAK} `picket' \bit{{\hith}uitiezzi} \p{3s.pres.act.} `drag, pull, draw'  

\item[16 :] \bit{la{\hith\hith}ila{\hith\hith}ie\v{s}kinuzi} \p{3s.pres.act.} redup. to?

\item[23 :] \bit{puriyalli} `bridle' (?); maybe a Luwianism?

\item[42 :] \bit{sar\=a tittanuanzi} sense?

\item[46 :] \bit{tak\v{s}an} \p{adv.} `together' (?) or as noun `center; joint'; with UD = `mid-day'?

\item[57 :] \textbf{IN.NU.DA} `straw'

\end{description}


\section{The Ahhiyawa Letters}


\begin{description}

\item[Publication:] [add]
\item[Edition:] \citet{beckman2011ahhiyawa}
\item[Background:] [add] \\

\subsection{Indictment of Madduwatta (CTH 147)}

\item[Background:] \citet[69-70]{beckman2011ahhiyawa}: \\ 
``This document, composed during the reign of Arnuwanda I, constitutes the opening portion of an extensive recounting of the duplicitous activities of a Hittite vassal in western Anatolia during the first decades of the fourteenth century, a time of relative weakness for the Hittite state. The absence of a heading as well as the presence of a substantive erasure (\S36� and alternative formulations of a single sentence (\S22) indicate that this is a preliminary draft. The purpose of the text is uncertain: Is it a warning to be sent to Madduwatta to encourage him to change his behavior, or a summary of evidence to be employed in a legal proceeding at the Hittite court?

Many of the events narrated in this text had taken place under the preceding king, Tudhaliya I/II , and records from his reign are frequently adduced here. In several instances, direct quotations from these documents have been carelessly edited, so that Tudhaliya refers to himself as ``the father of My/His Majesty'' (e.g., \S\S4, 6--7).

This text describes how Tudhaliya had rescued Madduwatta from an attack by Attarissiya of Ahhiya (\S\S1--3), assigned him the land of Mount Zippasla to rule (\S4), and imposed upon him an oath of vassalage (\S\S6--7). Later, against the explicit command of Tudhaliya, Madduwatta attempted to expand his territory at the expense of Kupanta-Kurunta of Arzawa, but the latter ruler soon gained the upper hand (��8�9), only to be repulsed by a Hittite army sent to the aid of Madduwatta (\S\S10--11). Finally, a second offensive by Attarissiya again threatened Madduwatta, necessitating yet another Hittite intervention on his behalf (\S12).

But rather than show his gratitude for Tudhaliya's benevolence, Madduwatta proceeded to engage in intrigues against his Hittite overlord (\S\S13--15)---even plotting with their former common enemy Kupanta-Kurunta (\S\S16--20)---extending his realm (\S\S22--23), seizing Hittite towns (\S24), refusing to extradite fugitives from Hatti (\S\S25, 30--32), and inciting other Hittite vassals to rebellion (\S26).

The present tablet concludes with an account of the mission of Arnuwanda's envoy Mulliyara (\S\S29--36) to the prevaricating Madduwatta, followed by a curious paragraph (\S37�) presenting an animal fable, unfortunately too broken for certain interpretation, but probably somehow reflecting on the conduct of Madduwatta. Undoubtedly the text did not end at this point, but we have not recovered any further tablets. We are also ignorant of the ultimate fate of Madduwatta, for this petty ruler does not appear in any other Hittite records.'' \\

%%resume here; pdf p.70

\item[2 :] \bit{kit\=at} \p{3s. pret. mp.} `lie' + \textit{appan} = `pursue; chase'; I think subj. clitic \textit{=a\v{s}} is correct here, since the verb is unaccusative, and only idiomatically functions transitively (?)

\item[3 :] \bit{piddai\v{s}} \p{2s. pret. act.} `flee; run'; note secondary spread of \textit{-\v{s}} ending proper to 3rd sing. (and 2nd pl.) to 2nd s.

\item[7 :] \bit{i\v{s}hue\v{s}ni} \p{n. d-l. s.; r/n-st.} `heap' 

\item[8 :] \textbf{DIM\subsc{4}} `malt' \textbf{BAPPIR.{\Hith}I.A} `(beer) wort'  \bit{EM\v{S}A} `rennet' \textbf{GA.KIN.AG.{\Hith}I.A} `cheese'

\item[10 :] \textbf{G\'IR} `sword'

\item[11 :] \textbf{UR.GI\subsc{7}.{\Hith}I.A} `dog'

\item[12 :] \bit{k\=astit=man} \p{instr.} `hunger' (?); correctly the instrumental, equivalent to \textit{appan} + dat. above?

\item[13 :] \bit{linkiya\v{s}=\v{s}a\v{s}} \p{c. gen. s.} `(a man) of his oath' = `sworn ally'; a nice example of QSV \index{quasi-serial verb construction}, where the reflexive particle \textit{=z(a)} (nice spelling!) must belong to V\subsc{2}

\item[16 :] \bit{i\v{s}ki\v{s}a} \p{n. all. s.; s-st.} `back' (?);  the overall sense here is unclear to me \bit{\=appa{\ldots}tiyan {\hith}ark} ??; Beckman et al. trans `keep your base of support established' and seem to ignore the `back' + postpos.

\item[18 :] \bit{maninkuwan} \p{adv.} `near' $<$ \textit{adj.} `short; small' 

\item[19 :] \bit{e\v{s}i} \p{2s. impv. act.} ` be'; unusual spelling?

\item[20 :] \bit{{\hith}ap\=atin} \p{c. acc. s.} `river-land' \textbf{ZI\bit{-it}} \p{instr.} `soul, self; will'

\item[24 :] \bit{uk=ak=wa=kan} `or I' (?); weird spelling of disjunctive \textit{=aku}, or perhaps something else entirely?

\item[25 :] \bit{sanna{\hith\hith}i} \p{1s.pres.act.} `hide; conceal' \textbf{KUR}\bit{eanza=ma=tta} \p{n. erg.} `land'; what is the clitic doing so low?

\item[26 :] \bit{QATE}\textbf{\mpl}\bit{=YA} `my hands'

\item[27 :] \bit{SAPAL} \p{pp.} `under' \bit{NI\v{S}} `life; oath'

\item[29 :] \bit{karsi} \p{adv.} `unrestrainedly; straight out' + \textit{za{\hith\hith}iya-} = `to make war without reservation/unconditionally'

\item[32 :] \bit{{\hith}aluki} \p{adj. n. acc. s.} `embassy'; would have expected \textit{{\hith}aluga-} \p{c.}?

\item[34 :] \bit{ku\v{s}(a)ziyadatarr=a=wa} `omen terminus?' I have no idea what this means; B et al. trans. + \textit{iya-} `display enmity' \textbf{\man}\bit{BEL}\textbf{{\wood}TUKUL} `lord of the tool/weapon' = `craftsman'

\item[35 :] \bit{\v{s}annatti=ya l\=e} \p{2s. pres. act.} `hide, conceal'; the clitic seems oddly low, and how is the negative after the verb to be explained? \bit{munn\=asi=ya=war=an} \p{2s. pres. act.} `conceal; hide'

\item[38 :] \bit{zammur\=aizzi} \p{3s. pres. act.} `insult; offend, give offense to'; denom. \textit{zamuri-} `(something bad)', which is likely Luw.

\item[41 :] \bit{sakuwa\v{s}\v{s}ar} \p{adv.} `thoroughly; completely' \textbf{IGI.{\Hith}I.A{\ldots}}\bit{MA{\Hith}AR} `before the eyes'

\item[44 :] \bit{\=e\v{s}} \p{2s. impv. act.} `sit; inhabit, settle'; oppositional transitive semantics with active inflection to normally mp. stem?

\item[45 :] \bit{ninikta} \p{3s. pret. act.} `raise; mobilize (troops)'

\item[46 :] \bit{kattan ar{\hith}a tarna\v{s}} `set loose' (?); with \textit{pait}, a nice example of QSV \index{quasi-serial verb construction}

\item[47 :] \bit{{\hith}aspir} \p{3pl. pret. act.} `annihilate; utterly destroy'; this gloss purely contextual?

\item[48 :] \bit{i\v{s}parter} \p{3pl. pret. act.} `escape; flee'

\item[50 :] \bit{ass\=u[ya]} \p{n. acc. pl.} `goods'; note archaic neut. pl., cf. \textit{a\v{s}\v{s}uwa} in ln. 54 below

\item[52 :] \bit{kappuwante\v{s}=pat} \p{adj. c. nom. pl.} `little; few'; formally past ptcpl. to \textit{kappuwai-} `count', hence lit. `numbered'

\item[53 :] \bit{sardiya\v{s}} \p{c. gen. s.} `helper; ally'; genitive here?

\item[54 :] \bit{s\=er} \p{postp.} `above'; preceding its governed noun is odd

%\item[55 :] how broken is the beginning of the line? if we can really read \textit{=at=\v{s}i}, this looks like an example of clitic right-doubling

\item[56 :] \bit{A{\Hith}ITI=\v{S}U} `his side' + \textit{ara{\hith}za} `away to the side; off to one side'

\item[57 :] \bit{k\=e=kan} \p{n. acc. pl.} `this'; as neut. acc. pl., \textit{k\=e} spelling is odd; weird note in GrHL ``only in LH''---LH? \bit{peda\v{s}\v{s}a{\hith\hith}ir} \p{3pl. pret. act.} `make subordinate; demote' (?); what is the morphology? \textit{tan peda-} `second rank', whence Luw.-looking poss. adj. \textit{tan peda\v{s}\v{s}a-} `of the second rank', then build factitive?; also, the sense `make subordinate' not in B. et al., rather `put in one's place again; re-install', which does seem to fit better in context

\item[59 :] \bit{kuenir} \p{3pl. pret. act.} `kill'; how are B et al. getting a passive reading here?

\item[61 :] [\'ERIN.ME\v{S} AN\v{S}E.KUR.P{\Hith}I.A] might have expected a PP, e.g. \textit{QADU}

\item[62 :] \bit{mazza\v{s}ta} \p{2s. pret. act.} `endure; withstand'

\item[66 :] \bit{wal{\hith}uwanzi} \p{inf.} `strike'; contrast the motion verb + infinitive syntagm with QSV in ln. 62 \index{quasi-serial verb expression}

\item[69 :] \bit{imma} \p{adv.} `in fact; verily'; interesting use, maybe contrastive?

\item[71 :] \bit{nininkir} \p{3pl. pret. act.} `mobilize' (?); seems to mean `defeat, destroy' in context

\item[72 :] \bit{{\hith}a{\hith\hith}ars\v{s}kit} \p{3s. pret. act.} `laugh; ha-ha'

\item[80 :] \textbf{\v{S}\`A}\bit{=\v{S}U} `own' (??) \textbf{\man}\bit{{\Hith}ATTUNI=\v{S}U} `brother-in-law; son-in-law'

\item[81 :] \bit{genzu} \p{n. acc. s.} `womb, lap; love, affection'

\item[82 :] \textbf{\supersc{m}}\bit{Madduwatta} \p{c. voc. s.} `M'; a real \textit{a-}stem vocative? \bit{tu\=ekku\v{s}} \p{c. acc. pl.} `body; limb'; metaphorically here?

\bigskip
\underline{Rev.}
\bigskip

\item[10 :] \bit{{\hith}ali{\hith}lai} \p{3s. pres. act.} `kneel down; genuflect'

\item[12 :] \textbf{\man}\bit{\v{s}ap\=a\v{s}ali\v{s}} \p{c. nom. s.} `scout; spy'; Indic/Mittani borrowing?

\item[13 :] \bit{{\hith}alzi\v{s}\v{s}atti} \p{2s. pres. act.} `call'; function of \textit{-i\v{s}\v{s}a-} suffix here unclear to me

\item[20 :] \bit{\v{s}arrattat} \p{3s. pret. act.} `transgress; overstep'

\item[21 :] \bit{zikke\v{s}} \p{2s. pret. act} `place'; archaic \textit{-ske-} iter. with epenthesis! \index{epenthesis}

\item[22 :] \bit{zikkit} what are we to make of this spelling of 3s. pret. ending \textit{-ta}?

\item[24 :] \bit{=za} `for himself'

\item[25 :] \bit{=tta} B. et al. trans. `through you, i.e. your territory'; rightly? \textbf{1}\bit{e$<$da$>$z} `alone' (?); somehow the reflexive should be here with unaccusative verb 

\item[26 :] \bit{ku-e-ni} \p{2s. impv. act.} `smite'; \textit{-i}-impv., supposedly from \textit{-*si}; \textit{kuen-} + \textit{=kkan} makes QSV very likely \index{quasi-serial verb expression}

\item[28 :] \bit{{\hith}antitiyatalle\v{s}} \p{c. nom. pl.} `informer; tattletale' \textit{{\hith}anti-} `in front' + weak stem \textit{dai-} `place' (?) + agent suffix

\item[30 :] \bit{k\=el} \p{pron. neut. gen. pl. (?)} `this'; as genitive plural?

\item[31 :] \bit{kuit} difficult syntax here:

\item[32 :] \bit{utummanzi} \p{inf} `bring here'; infin. to \textit{uda-}; JL may be interested in \textit{-m-} here \bit{da\v{s}kit} \p{3s. pret. act.} `take'; iter. \textit{d\=a-} `take'; nice contrasts between \textit{zikke-} `place' and \textit{daske-} `take' in this text

\item[33 :] \textbf{{\wood}APIN} `plow'

\item[34 :] \bit{\=e\v{s}tat} \p{3s. pret. mp} `sit; occupy, settle'; + acc. here

\item[35 :] \textbf{\supersc{m}}\bit{Maduwatta\v{s}=ma=a\v{s}=za} PN + conj. + acc. pl. clitic + refl. `for yourself'

\item[36 :] \bit{siedani} \p{num. d-l. s.} `one'; rare orthographic writing \bit{{\hith}atr\=auni} \p{1pl. pres. act.} `send a letter'; what are we to make of this spelling?

\item[39 :] \bit{paranda} \p{adv.} `across; outside; beyond' + \textit{tittanu-} = ``lead astray'' (?); restored + trans. by B

\item[40 :] \bit{\=e\v{s}dumat} \p{2pl. impv. mp.} `sit; settle, occupy'; nice uncommon form here

\item[41 :] \bit{tar{\hith}uilau\v{s} kar\=atau\v{s}} `conquering innards' (?); B trans. `heart' \textit{non rect\=e}; verb is missing; what does CM make of this phrase?

\item[42 :] \bit{munnait} \p{3s. pret. act.} `hide, conceal'; with \textit{\v{s}akuwa}, must be idiomatic; precise meaning?
 
\item[43 :] \bit{parranta} + \textit{tittanu-} = ?

\item[46 :] \bit{i\v{s}iya{\hith\hith}i\v{s}} \p{3s. pret. act.} `announce; report'

\item[48 :] \bit{{\hith}antezzi \textbf{BAL}\bit{\supersc{LIM}}} `on the first occasion; at first, initially' = \textit{{\hith}antezzi palsi} (?)

\item[50 :] \bit{kukupalanni} \p{n. d-l. s.; r/n-st.} `cheating; deception'

\item[51 :] \bit{\textsubdot{S}\'IMDI} `band; group; team' \textbf{{\wood}GIDRU} `staff, scepter'

\item[54 :] \bit{kattan lukk\=er} \p{3rd. pl. pret} `kindle; set on fire'

\item[59 :] \textbf{\man}\bit{\textsubdot{S}\=AIDU} `hunter' \bit{ITTI} ?

\item[85 :] \bit{piddanzi} with \textit{peda-} `bring'

\item[87 :] \bit{w\=atarna{\hith}ta} \p{3s. pret. act.} `order, command; inform'


\end{description}

\section{Midwife Ritual of Papanigri (CTH 476)}

\begin{description}

\item[Publication:] [add]
\item[Edition:] \citet{strauss2006ritual}
\item[Background:] [add]

\bigskip

\item[\underline{VS. I}]

\item[1 :] \textbf{\man}\bit{patili} with CM against \citeauthor{strauss2006ritual}, read $<$\textit{pa-ti-}$>$ for \supersc{X}$<$\textit{pat-ti-}$>$ throughout; anything about this word?

\item[2 :] \textbf{\supersc{DUG}D\'ILIM.GAL} `big vessel; ladle (?)' \bit{{\hith}arn\=awi} \p{c. d.-l. s.} `birthing stool'; plene suffix---ablaut pattern?

\item[3 :] \textbf{{\wood}GAG} `plug' \bit{duwarnattari} \p{3s.pres.mp.} `break'; Alwin's story?

\bigskip
%%room for the story on duwarna

\item[5 :] \textbf{{\wood}AB{\hpl}} `windows'

\item[7 :] {\bit{UNUTE}\textbf{\mpl}\bit{=ya=kan}} `instruments; tools'

\item[13 :] \bit{dammili} \p{adj. d.-l. s.} `inviolate' (?)

\item[14 :] \bit{andan=pat} force of particle?

\item[16 :] \bit{ar{\hith}a ariya} \p{2s.impv.act.} `determine by oracle' \textbf{\supersc{\'E}}\bit{karrimiya} `temple'; this is indirect question!

\item[26 :] \bit{ar{\hith}ayan} \p{adv.} `separately' \bit{i\v{s}{\hith}arnumaizzi} \p{3s. pres. act.} `bloody; cover with blood'

\item[27 :] \bit{uzziya=ya} \p{n. acc. pl.} `flesh'

\item[28 :] \bit{markanzi} \p{3pl. pres. act.} `divide; apportion'

\item[30 :] \bit{kalutezzi} \p{3s. pres. act.} `worship/sacrifice collectively/as a ring/row' (?); denom. \textit{kaluti-} `row; ring'

\item[31 :] \textbf{UZU} `flesh; meat'

\item[39 :] \bit{{\hith}ingazi} \p{3s. pres. act.} `hand over; offer'

\item[41 :] \bit{appezziaz} \p{adv.} `later; afterward'; this makes no sense in context?

\item[43 :] \bit{k\=a} \p{adv.} `here, hither' \bit{par\=a {\hith}andanni} \p{n. d-l. s.} `providence'; sense?

\item[46 :] \bit{\v{s}arnikta} \p{3s.pret.act.} `make recompense'

\item[50--52 :] \bit{MELQIZU} `its contents'; = /melqe/it=su/ \bit{ki\v{s}ri\v{s}} `combed yarn (?)'; \textbf{\supersc{T\'UG}}\bit{kurie\v{s}\v{s}ar} `length of cloth' \bit{tarpala\v{s}} ? \textbf{SA} `red' \textbf{ZA.G\`IN} `blue' \bit{zapzagaya} `glazed/enameled bowls'

\item[53 :] \bit{\v{s}iyan harzi} `keep (it) pressed on top (?)'; I think S. has `has sealed (that) up'

\item[54--56 :] \textbf{\`I.GI\v{S}} `sesame oil' \bit{\=an} \p{ptcpl. n. nom. s.} `warm; hot' \textbf{\supersc{DUG}}\bit{{\hith}\=uppar} `bowl; cup' \textbf{\supersc{DUG}\'UTUL} `pot; jar' \bit{par\v{s}\=ur} `(stew-)pot; kettle' \bit{tu{\hith}alzi} ?

\smallskip
\item[\underline{Vs. II}]
\smallskip

\item[2 :] \bit{{\hith}aratni} \p{n. d.-l. s.} `offense'; very similar sense to conjoined \textit{wa\v{s}tuli}

\item[3 :] \textbf{SILA\subsc{4}} `lamb; sheep' \bit{enuma\v{s}\v{s}i} \p{d-l. (?) s.} `for peace'?; Hurrian

\item[4 :] \bit{\v{s}uniyanzi} \p{3pl.pres.act.} `immerse (in water); dunk'

\item[10 :] \bit{kupa{\hith}in} \p{c. acc. s.} `cap; hat'

\item[11 :] \bit{tarn\=ai} sense here?

\item[12 :] \bit{puriya} \p{n. d.-l. s.} `sight; vision; gaze' (?)

\item[14--15 :] \textbf{\wood}\bit{{\I}rimpi} `(cedar) beam / plank' \bit{\textsubdot{S}ALITTA} `black; dark'

\item[16--19 :] \bit{nu} what is this doing here? syntax of preceding and what follows is confusing; copy down what Craig says \textbf{\bread}\bit{zipinni} `\textit{z-}bread'

\item[24--26 :] \textbf{GA.KIN.AG} `cheese' \textbf{{\wood}P\`E\v{S}} `fig' \bit{tan{\hith}arie\v{s}\v{s}=a} ? \bit{TUDITTUM} `needle; (cloak)pin'

\item[27--30 :] \textbf{\supersc{DUG}ABxA}\bit{-ya} `water basin' \textbf{GADA}\bit{ QATI} `hand-towel' \textbf{\wood}\bit{zuppari} `torch'

\item[31--33 :] \textbf{AD.KID} `reed; mesh' (?) \bit{EM\textsubdot{S}\'U} `sour(dough)' \textbf{\bread}\bit{\v{s}\=ena\v{s}} `\textit{s}-bread' \textbf{\bread}\bit{amp\=ura\v{s}} `\textit{a}-bread' 

\item[34--37 :] \textbf{\wood}\bit{KANNUM} `standing/upright vessel' (?) \textbf{{\wood}MA.S\'A.AB} `basket' \textbf{{\wood}}\bit{INBI}\textbf{\hpl} `fruit; produce' \textbf{\wood}\bit{ariyala} `carrying-basket; pannier' 

\item[38--43 :] \textbf{GIR\subsc{4}} `clay' \textbf{{\wood}P\`E\v{S}} `figs' \textbf{\vessel}\bit{KUKUB} `can; jug' \textbf{\v{S}\`AB\'A} `within; among (which)' \textbf{\`IGI\v{S}} `sesame oil'

\item[44--49 :] \bit{t\=uriyante\v{s}} formally, look like nom./acc., but should be gen. with \textit{\v{S}A} (and cf. \textit{patili\=e\v{s}}, above)---what's happening here?

\item[50--54 :] \textbf{T\'UG.G\'U.\`E.A} `shirt' \textbf{E.\'IB} `girdle; belt' \bit{TA{\Hith}AP\v{S}I} `(leather) belt/strap' \bit{kupa{\hith}iu\v{s}} \p{c. acc. pl.} `\textit{k-}hat; cap'; accusative plural in a list (?) \bit{TAPAL} `set; pair' \textbf{\supersc{KU\v{S}}E.SIR} `shoe; slipper' \textbf{{\Hith}AR.\v{S}U} `bracelet' \textbf{{\Hith}AR.G\`IR} `anklet' \bit{KILILU} `crown; headband' \textbf{SI.GAR} `necklace; chain, string'

\bigskip
\item[\underline{Rs. III}]
\bigskip

\item[1--9 :] \textbf{\supersc{UZU}\'UR} `limb; haunch (of an animal sacrifice)'

\item[10--16 :] \textbf{\supersc{TU\subsc{7}}\v{s}ampukkiya\v{s}} `\textit{\v{s}-}(bread) soup' (?) \bit{UPNI} `handful; fistful' \textbf{IZI} `fire' \bit{{\hith}uma\v{s}=\v{s}an} \p{n. acc. s.} `all'; with /-ns-/ assimilation; CM flags---why? and sense of \textit{appa} here?

\item[17--25 :] \bit{ana{\hith}i} `sample; taste(r)' \bit{kurimpa\v{s}} \p{c. d-l. pl.} `rest; remainder' \bit{dagan} \p{n. acc.(?). s.} `earth'; or endingless loc. `on the earth'? \bit{\=arranzi} \p{3pl. pres. act.} `wash'

\item[26--32 :] \textbf{BA.BA.ZA} `(barley) mash; mush'

\item[33--40 :] \bit{ana{\hith}ita} \p{n. acc. pl.} `taster'; note the stem-final coronal is retained in the pl.

\item[41--47 :] \textbf{{\vessel}\'ABxA} `water basin' \bit{{\hith}ulli\v{s}} `pine cone' (?) or `charcoal' (?) \textbf{{\wood}ERIN} `cedar' \bit{p\=edai} `bring'; against Strauss in main text, with implicit object \textit{=at}

\item[48--55 :] \textbf{\supersc{f}}\bit{katra\v{s}} `\textit{k-}woman' \textbf{{\wood}BALAG} `harp' \bit{esandari} \p{3pl. pres. mp.} `sit'; + \textit{=za} `sit down' \bit{saranzi} \p{3pl. pres. act.} `pluck; pick'; sense with obj. `cloth'?

\bigskip
\item[\underline{Rs. IV}]
\bigskip

\item[1--8 :] \bit{taruppanzi} \p{3pl. pres. act.} `gather; collect; plait/braid together' \bit{\v{s}uritaya} `ball (of wool); wad' \textbf{SILA\subsc{4}} `lamb' \bit{an\v{s}anzi} `wipe off; rinse' \bit{{\hith}amanki} \p{3s. pres. act.} `bind; tie'; pristine 3s. ablauting \textit{{\hith}i}-verb here \bit{{\hith}\=ulaliyanzi} \p{3pl.pres.act.} `wrap around'; denom. \textit{{\hith}ulali-} 

\item[9--14 :] \bit{M\=E} `water' \bit{karapzi} (?) 

\item[15--21 :] \bit{arrumma\v{s}} \p{n. gen. s.} `washing'; vbl. noun in \textit{-war} w/ dissim. \bit{unuwanzi} \p{3pl. pres. act.} `decorate; adorn'; stem \textit{unu-} w/ glide formation 

\item[22--33 :] \bit{san{\hith}anzi} \p{3pl.pres.act.} `wipe off; clean; purify' \textbf{\wood}\bit{pa{\hith\hith}i\v{s}a=ya=\v{s}\v{s}i} `\textit{p-}' (?); `they beat the \textit{p-} above him' (?); syntax unclear to me here

\item[34--36 :] \bit{siptamiya} `sevenfold' (?); or `(a drink made from 7 ingredients)' (?) \bit{teriyalla} `thrice' (?)

\item[37--43 :] \textbf{DUB} `tablet' \bit{AWAT} `word; matter' = Hitt. \textit{uttar / memiya-}; subject of nominal sentence with copula omitted \textbf{{\man}DUB.SAR} `tablet-writer; scribe'; SAR = Hitt. \textit{{\hith}atrae-} \bit{I\v{S}TUR} \p{3s.pret.act.} `write'


\end{description}


\section{The Myth of Illuyanka (CTH 321)}

\begin{description}

\item[Publication:] A = KBo 3.7; B = KUB 17.5; C = KUB 17.6; D = KUB 12.66; E = KUB 36.54; F = KBo 12.83; G = KBo 12.84; H = KBo 22.99; J = KUB 36.53
\item[Edition:] \citet{beckman1982illuyanka}
\item[Translation(s) :] \citet{beckman1982illuyanka}, \citet[12]{hoffner1998myths}
\item[Background:] \citet[10--13]{hoffner1998myths}: 
``These simple tales, taken over from the Hattian people, attribute a poor spring to the defeat and incapacitation of their chief deity, the Storm God, by an evil and powerful reptile. Reptiles are not universally symbols of evil and destruction: In Egypt the uraeus serpent protected the pharaoh from evil. But clearly in Hittite culture, as in Babylonia and ancient Israel, serpents usually represented evil. In both versions of the myth, the Storm God needs the help of a mortal and a trick in order to regain supremacy over the serpent. These stories were probably told or sung during the course of the Purulli Festival, about which we know relatively little outside of these stories. If Pecchioli Daddi (1987) is right in including the Teteshapi cult texts under the rubric of Purulli, we know more. See the discussion by Pecchioli Daddi and Polvani (1990: 39-55) with additional literature. \\

Although the Illuyanka text shows much linguistic archaism, which suggests that the narratives go back at least to the Old Hittite period (c. 1750-1500), the surviving copies date only from the New Kingdom (c. 1500-1190). \\

In the first stoiy the serpent is a land creature who emerges from a hole in the ground. The defeated and disabled Storm God calls for a feast, at which his daughter, the goddess Inara, a goddess of the wild animals of the steppe land, in partnership with a mortal man, Hupasiya, tricks the serpent and renders him powerless. Haas (1982: 45, 111) has characterized Hupasiya as a Jahresk�nig, a king who with his priestess queen guarantees the flourishing of livestock and vegetation. In suggesting that there are allusions in Hupasiya to ritual regicide, Haas seems to assume that Hupasiya is eventually killed. But these interpretations rest upon a superficial use of comparative evidence and lack a proper foundation in solid textual evidence from the Hittite sources. Nothing in the narrative suggests that Hupasiya becomes a king, and it is still uncertain that he is killed. It also goes beyond the present evidence to assert that Hupasiya's sleeping with Inara was a \textit{hieros gamos} (sacred marriage) (Haas 1982), the nature (and even existence!) of which in other ancient Near Eastern cultures is still seriously questioned. In the first version of the story all the characters have names, and the earthly action is set in or near known Anatolian cities, such as Tarukka and Ziggaratta. \\

In the second story the characters have no names but are identified by functional expressions, such as "daughter of a poor man" and "son of the Storm God." No geographical names occur. The two battles of the Storm God and the serpent take place at an unspecified sea. The Storm God's ruse involves a special type of marriage known from the Hittite laws. If a young suitor was too poor to pay a bride-price for a wife, he could offer himself as a ``live-in'' husband to a wealthy father-in-law in exchange for a ``bride-price'' paid to himself. The Storm God's son thus finds himself in a classic situation of divided loyalty: he is son of the Storm God but also live-in son-in-law of the serpent. His agonizing choice costs him his life. The fact that the Storm God's mortal son was also the son of the daughter of a poor man'' helps us to understand why such a marriage arrangement would be necessary for him. Pecchioli Daddi has pointed out that ``the daughter of a poor man'' plays a role in certain cult texts of the deity Teteshapi, whom she identifies with Inara (Pecchioli Daddi 1987).


\end{description}

\subsection{Version 1}

\begin{description}

\item[A i 1-2 :] According to Killa, the \textit{G-}priest of the Storm-god of Nerik, (these) are the words of the \textit{purulli-}festival of [=for] the Storm-god of heaven. When they speak as follows: ``Let the land grow (and) let it prosper! Let the land be protected! When it grows (and) thrives, then they perform the \textit{purulli-}festival.''

\begin{notes}

\bit{m\=au} \p{3s.impv.s.} `grow; thrive, prosper'; impv. to intrans. {\hiverb} \textit{mai--} \bit{\v{s}e\v{s}du} \p{3s.impv.s.} `prosper; proliferate'; etymologically obscure, but regular collocated with \textit{mai--} in contexts where its basic synonymy is assured; per AK, the oldest spellings are all \textit{\v{S}I-I\v{S}}; only in NH does it show orthographic identity with \textit{\v{s}e\v{s}--} `sleep' 

\end{notes}

\item[A i 9--14 :] When the Storm-god and Illuyanka engaged in hand-to-hand combat (?) in the land of Kiskillussa, Iluyanka overcame the Storm-god. And the Storm-god shouted (to) all the gods: ``Step [= come] inside! Inara has made a feast. 

\begin{notes}

\bit{argatiy\=er} \p{3pl.pret.act.} `' \bit{=za} why the reflexive particle present in ln. 11 or 14? \bit{=pa} ?? \bit{y\=et} \p{3s.pret.act.} `do, make'; archaic spelling with initial $<$\textit{\textsubarch{i}}$>$ and plene vowel

\end{notes}

\item[A i 15--26 :] 

She prepared a lot (of) everything: \textit{p}-vessels of wine, \textit{p}-vessels of \textit{m}-drink, (and) \textit{p}-vessels of `punch' and she made abundance in the vessels. Then Inara went to Ziggaratta and found H\=opasiya, a mortal. Inara (spoke) to H\=opasiya as follows: ``I shall do such-and-such thing, so you also join me! And H\=opasiya (spoke) to Inara as follows: ``If I (may) sleep with you, I shall come and do your heart's (desires).'' Then he slept with her. 

\begin{notes}

\textbf{DUG}\bit{pal{\hith}i} \p{n.acc.pl.} `broad (vessel)'; the inflection seems weird: coll. pl. \textit{--i}, but forms below in this text seem to show \textit{a-}stem inflection, e.g. d-l. pl. \textit{-a\v{s}} or acc. s. \textit{-an} \bit{iyada} \p{n.acc.pl.} `fullness; abundance, plenty'; some forms point to an \textit{-r/n-} stem, but this is impossible here \bit{k{\I}=ya k{\I}=ya} `such-and-such' (?) \bit{{\hith}arap{\hith}ut} \bit{2s.impv.mp.} `reassociate, change allegiance; join/combine (with)' 

\end{notes}

\item[B i 3{\pr}--18{\pr} :] And Inara led H\=opasiya and concealed him. Inara adorned herself and called Illuyanka up from (their) holes, (saying): ``I am making a feast. Come to eat (and) drink!'' And Illuyanka together with his children came up and they ate (and) drank. They drank every vessel and they became drunk. Thereupon they were unwilling to go down into their holes, so H\=opasiya came and tied up Illuyanka with bonds. The Storm-god came and slew Illuyanka. The gods were with him.


\begin{notes}

\bit{unuttat} \p{3s.pret.mp.} `adorn, decorate \bit{kalli\v{s}ta} \p{3s.pret.act.} `call; summon' : \textit{kalli\v{s}\v{s}--} (< \textit{*{\palk}\'el{\hi}-s--}; cf. Gk. {\greektext kal'ew}, Lat. \textit{cal\=are} \bit{hante\v{s}naz} \p{n.abl.;r/n-st.} `hole; cave' \bit{nink\=er} \p{3pl.pret.act.} `quench one's thirst; become sated / drunk'; no relation to \textit{n{\I}nink--} `mobilize' \bit{n\=uman} \p{adv.} `be unwilling; do not wish'; the negative ``subject-optative'' marker, on which see \citetalias[\S26.19]{GrHL}; the formal origin of `NEG' element remains unexplained---in particular, its [\=o]-vocalism \bit{kal\=eli\=et} \p{3s.pret.act.} `bind; tie up'; \textit{--y\'e--} denominative to unattested base \textit{*kal\=el--}, with final long vowel (?) (cf. Gk. {\greektext k'alws, k'alos} `rope; line')

\end{notes}

\item[C i 14{\pr}--A ii 14{\pr} :] Then Inara built a house upon a rock (outcropping) in the land of T\=arukka and she settled H\=opasiya within the house. And Inara repeatedly instructed him: ``When I go to the open country, you must not look out from the window. If you look out, you shall see your wife and your children.'' When twenty days had gone (by), that man looked out from the window and saw his wife and his children. When Inara came back from the open country, that man began to cry: ``Let me go back to (my) house!'' Inara (spoke) [to H\=opasiya as foll]ows: ``[{\ldots}] away [{\ldots}''{\ldots}] the Storm-god the meadow [{\ldots}] that man and him [{\ldots}].



\begin{notes}

\bit{wetet} \p{3s.pret.act.} `build'; the verb is clause-internal, either because it has undergone movement left or because the locative PP is postposed/right-dislocated \bit{watarna{\hith}{\hith}i\v{s}kizzi} \p{3s.pres.act.} `command; instruct'; historical present \bit{zigg=a} \p{pron.2s.nom.s.} `you'; geminating \textit{=a} here? \bit{ar{\hith}a=ma} what is \textit{=ma} doing so low? \bit{{\hith}arannet} ?? \textbf{\'U} `meadow' (= \textit{welku--}) 

\end{notes}

\item[A ii 15{\pr}--30{\pr} :] Then Inara came into the city of Kiskussa. Her house and the river of the watery depths (?) she placed [in the] hand of the king. Because we are performing the first \textit{purulli-}festival, the hand of the king also (?) [{\ldots}] the house of Inara and the ri[ver] of the watery depths (?). Mt. Zaliyanu is foremost of all. When he has bestowed rain on the city of Nerik, the man of the staff brings \textit{{\hith}}-bread from Nerik. He demanded rain (from) Mt. Zaliyanu. He brought it, the bread, [to him {\ldots} {\ldots} 

\begin{notes}

\bit{{\hith}un{\hith}wana\v{s}\v{s}=a} \p{gen.} `watery depth' (?); a reduplicative formation, with copy \textit{u}-vocalism \bit{{\hith}eun}; see \citet[21--2]{beckman1982illuyanka} for discussion of its semantics and related forms \p{c. acc. s.} `rain'; a double obj. construction with \textit{wek--}? \bit{NINDA} `bread'; HH seems to suggest clitic doubling with coreferential \textit{=an}?

\end{notes}

\end{description}

\subsection{Version 2}

\begin{description}

\item[D iii 1{\pr}--19{\pr} :] This [is {\ldots} which Kella the \textit{G}-priest] spo[ke:] Illuyanka overcame the Storm-god and took (his) [heart and eyes], and the Storm-god [{\ldots}] him. He took as wife ``the daughter of a poor man'' and begot a son. When he had grown up, he took as wife the daughter of Illuyanka. The Storm-god repeatedly instructed (his) son: ``When you go into your wife's house, demand from them (my) heart and eyes.'' When he went, he demanded from them the heart and eyes, and they gave them to him. And afterward he demanded from them the eyes, and at that time they gave (them) to him. Then he brought them to his father the Storm-god,  and the Storm-god took back his heart and eyes.

\begin{notes}

\bit{=\v{s}ma\v{s}} \p{pron.encl.3pl.dat.} dat. of source \bit{n=a\v{s}=(\v{s})\v{s}i} `them to him'; the clitic appears to be anim. acc. pl. though the referent is singular; GB entertains an irregular assimilation / at = si / $\rightarrow$ \textit{-a\v{s}=\v{s}\v{s}i} \bit{\v{s}akuwa=\v{s}\v{s}\supersc{!}et=a\supersc{!}} \p{neut.acc.pl.} `eyes'; the gemination of \textit{-\v{s}-} is unexpected, as is the non-gemination of final \textit{-t-}, if indeed the sequence includes geminating \textit{=a} as in previous occurrences. These are likely to be explained by the fact that NH scribes no longer command possessive enclitics: this scribe effectively took \textit{sakuwa} + \textit{=a/ya} attested above and tacked on the poss. encl. (??) 

\end{notes}

\item[D iii 20{\pr}--D iii 35 :] When his body was once again whole (and) ???? as before, then he went to the sea for battle. When he gave battle to him and moreover he began to overcome Illuyanka, then Illuyanka was with the son of the Storm-god, and he [= son of StG] called up to his father in heaven: ``Take me also among (them)! You must not show me mercy.'' So the Storm-god slew Illuyanka and his son. And that Storm-god [{\ldots}].

\begin{notes}

\bit{\=e\v{s}\v{s}ri} \p{n.nom.s.} `body, shape; image, statue' \bit{wiliyatta} \p{3s.pres.mp} (??) \textbf{SIG\subsc{5}}\bit{-atta} \p{3s.pres.mp.} `become good; become healthy/whole' \bit{\`U} what underlies this initial usage of \textit{\`U}? \bit{=pa} ?? \bit{genzuwai\v{s}i} \p{2s.pres.act.} `be kind to; show mercy'; denom. \textit{genz\=u} \p{n.} `lap'


\end{notes}

\item[J 1{\pr}--11{\pr} :] Kel[la the \textit{G}-priest of the Storm-god of Nerik] (speaks) as follows: [{\ldots}] and for him to eat [{\ldots}] back to the city of Nerik [{\ldots}] let go [{\ldots}] the god Zashapuna [{\ldots}] the Storm-god of Nerik [{\ldots}] went and to the god(dess?) Zaliy[{\ldots}] gave back [{\ldots} t]o the city of Ne[rik {\ldots}].

\begin{notes}

\end{notes}

\item[D iv 1{\pr}--16{\pr} :] And for the \textit{G}-priest they made the foremost gods last, and the last (ones) they made the foremost gods. There is a lot (of) cult material for the goddess Zalinu. Zalinu, wife of the god Zashapuna, (is) great(er) (?) (than) the Storm-god of Nerik. The gods (speak) as follows to the \textit{G}-priest Tahpurili: ``When we go to the city of Nerik, where should we sit?'' Thus (speaks) the \textit{G}-priest Tahpurili: ``When you sit (upon) a stool of basalt, and the \textit{G}-priests cast the lot, then the \textit{G}-priest who is holding (the image of) Zaliyanun---a stool of basalt is placed above the spring (for him), and he sits down there.

\begin{notes}

\bit{{\hith}alkuy\supersc{?}e\v{s}\v{s}ar} \p{n.nom.s.} `cult material'; weird spelling with medial [y]? \bit{\v{s}alli\v{s}} \p{adj.c.nom.s.} `big; great'; nominal sentence with ellipsis of copula; comparative sense? \bit{nu} error here? \bit{e\v{s}wa\v{s}ta} \p{1pl.ind.mp.} `sit'; transformative mp. is a nice archaism here, repeated also in ln. 12{\pr} \bit{m\=an} \p{conj.} `when, if'; seems nonsensical, and omitted in A and C \textbf{\stone}\bit{\v{S}U.U} `basalt (?)' (= \textit{kunkunuzzi}) \bit{p\=ul} \p{n.acc.s.} `lot; fortune'; + \textit{dai--} = `cast a lot'

\end{notes}  

\item[A iv 14{\pr}-- :] All (of) the gods arrived and they (= the \textit{G}-priests) cast the lot. Zashapanas (shall be) great(est) of all the gods of Kastama. Because she (?) is the wife of Zalinu and Tazzuwassi (is his) (?) concubine, these three persons shall remain in Tanipiya. And afterward in Tanipiya a section of land is given by the king: 6 \textit{k-}measures of land, 1 \textit{k-}measure of wine-garden, a house and a threshing-floor, 3 houses for servants---(that) is on a tablet (?). I (shall) be reverent of the matter. I have spoken these things. One tablet is complete of the word of Kella the \textit{G}-priest. Pihaziti the scribe wrote (it) in the presence of Walwaziti the chief of the scribes.

\begin{notes}

\bit{\v{s}a\v{s}anza} \p{c.nom.s.} `concubine'; old ptcpl. to \textit{\v{s}a\v{s}--} `sleep' \bit{\=e\v{s}zi} \p{3s.pres.act.} `is'; sense here?


\end{notes}

\end{description}

\section{The Pu{\hith}anu-Chronicle}

\begin{description}

\item[Publication:] \begin{enumerate}

\item A = KUB 31.4 + KUB 3.41; B = KBo 12.22 I 1--14 // A. Vs. 1--9; C = KBo 13.78 Vs. 1--15 // A. Vs. 1--14
\item A = KBo 3.40 (3); B = KBo 13.78 Rs. 1{\pr}--14{\pr} // A. Rs. 3{\pr}--14{\pr}
\item KBo 3.43 (+\supersc{?} KUB 31.4 + KBo 3.41) (4)
\item KBo 3.42 (+\supersc{?} KBo 3.40) (5)

\end{enumerate}

\item[Edition:] \citet{soysal1981puhanu}
\item[Translation(s):] 
\item[Background:] 

\end{description}

\subsection{}

\begin{description}

\item[1--9 :] So (speaks) Puhanu, the servant of Sarmassu{\ldots} man for him{\ldots} he wore colorful clothes. A basket is pla[ced] on his head. He holds a bow [in] his [hand]. He calls out for help: ``What have I done? What? I have not take anything from anyone. I have not taken anyone's cow. I have not taken anyone's sheep. I have not taken from anyone. I have not taken the servant (or) slave of anyone. Why have you acted in such a way, (O) Saramu, that you have bound this yoke on me? I shall come (and) bring ice with this basket. I shall do battle and destroy the lands. With this reed you shall touch into your heart (?) once again.''

\begin{notes}

\textbf{T\'UG.G\'U.\`E.A} `clothing'; \textbf{DAR.A} `colorful' \bit{uw\=arra} + \textit{{\hith}alzai--} = `call for help'  \bit{i\v{s}{\hith}ai\v{s}ten} \p{2pl.pret.act.} `bind' \bit{uwami} \p{1s.pres.act.} `come'; relationship to preceding? \bit{natida} \p{c.instr.} `reed' \bit{karda=\v{s}ma} \p{n.all.s.} `(your) heart' \bit{\v{s}alikti} \p{2s.pres.act.} `touch; have contact/intercourse with'

\end{notes}

\item[10--19 :] ``What have you brought to Arinna? Is that (?) my adversary? It is not a donkey (?). I shall sit (?) for him, and you shall take me (there) (???). (As) the one who holds all the land, should I not nail down the rivers, the mountains, and the seas? I shall not nail down the mountain, (so that) it does not turn over. I shall nail down the river, (so that) it does not flow back. {\ldots} became a cow. His horn (was) twisted (?). I ask {\ldots}: Why is his horn twisted in this way? And he (answered) as follows: {\ldots} When I go on campaign, the mountain became burdensome to us. The cow{\ldots} was strong. When he came, that one  took (up) the mountain, and him {\ldots} he turned (?) in that way. And we overcame the sea. And his horn was twisted (?) in that way.


\begin{notes}

\bit{uni} \p{c.acc.s.} `that'; old medial deictic pronoun; syntax? \bit{{\hith}urtallim=man} \p{c.acc.s.} `?' \bit{\=eska{\hith}{\hith}a} \p{1s.pres.mp.} `sit'; iterative-inchoative in \textit{--\v{s}ke--}; mediopassive with \textit{--ske--} to an otherwise active verb; meaning here? \citet{soysal1981puhanu} takes as though \textit{appa es--} `oppose' \bit{tarmai\v{s}kimi} \p{1s.pres.act.} `nail down; fasten'; \textit{--ske--} in distributive function, nicely contrasting with simplex verb in next clause \bit{edi} \p{adv.} `across; beyond; over' \bit{lipsan} \p{ptcpl. n.acc.s.} < (unattested) \textit{lips--} `twist; break' (?) \bit{\v{S}U} \p{pron.c.nom.s.} `he' \bit{wed=a} \p{3s.pret.act.} `come'; with non-geminating =a? \bit{karapta} \p{3s.pret.act.} < \textit{karp--} `take (away/up); lift; pluck' \bit{\v{s}=an=\a{s}ta} rightly? \bit{n\=aes} \p{3s.pret.act.} `turn' (?) \bit{apeda} \p{adv.} `in that way; thus'

\end{notes}

\item[20--26 :] And now his Majesty is {\ldots} And he sends messengers (saying): Go to Halpa! And you shall say to the troops there: Suppiyahsu and Zidi (are) there{\ldots}Suppiyahsu{\ldots} And Innara of the city of Hattusa and Zidia (?) and Hattusa{\ldots}. And go say: Come to Zalpa! You shall come s[trike] their land! {\ldots}they hold behind{\ldots}you shall come to [Zal]pa{\ldots} and we shall break them. Down {\ldots}before{\ldots}we shall com[e th]ere.

\begin{notes}

\bigskip

\bigskip

\end{notes}

\item[1{\pr}--11{\pr} :] {\ldots} they cut. These{\ldots}they cut and he sings{\ldots} And they (say) as follows (?): `Let him not rejoice. Let them not remain {\ldots}I ask the young men{\ldots} these{\ldots} you shall{\ldots}And they (say) as follows: If we establish them a second time (???){\ldots}that mountain which they cut, {\ldots}shall not{\ldots}But now if there is anything (??), he must not step toward it (?). While the Storm-god of Halpa flees from us{\ldots} and he comes for us (?), that one also shall begin to run before (?). And me the manly gods of the Storm-god (??) have sent to the king (saying): `Go find the chiefs!' And let the chiefs say to king: You reverenced me (?) and{\ldots} come{\ldots} (?).

\begin{notes}

\bit{nu=z} \p{refl.} nice archaic spelling of \textit{=z(a)} showing it is [\texttslig] \bit{tu\v{s}ki\v{s}kattaru} \p{3s.impv.mp.} `be happy; rejoice' \textbf{\men}\bit{mayandu\v{s}} \p{c.acc.pl.} `young/vigorous man' \bit{ma(n)=war=u\v{s}} rightly? \bit{kuin} embedded relative, since overt demonstrative \textit{uni}?? \bit{anzitaz} \p{1pl.pron.abl.} `we; us' \bit{wemiya} \p{2s.impv.act.} `find'; probable QSV here \index{quasi-serial verb construction} \bit{na{\hith}ta} \p{2s.pret.act.}

\end{notes}

\item[12{\pr}-- :] The Hurrian has not yet come. Four years have passed by. There are two fighters and they sing:

\begin{verse}
Clothes of Nesa, clothes of Nesa, bind (on) me, bind! \\
Bring me by my mother---bind me, bind! \\
Bring me by my \textit{uwa--} --- bind me, bind! \\
\end{verse}


\begin{notes}

\bit{{\hith}urla\v{s}} \p{adj.c.nom.s.} `Hurrian' \bit{{\hith}ul{\hith}uliyante\v{s}} \p{c.nom.pl.} `fighter'; redup. nominal formation to \textit{{\hith}ulla/i--} `fight' \bit{tiya} \p{2s.impv.act.} `bind'; verb only in impv.; also secondary \textit{r/n}-stem \textit{tiyammanda} \p{instr} `with a cord' (e.g. \textit{tiyammanda i\v{s}{\hith}iyan} `bound with a cord'); root etymology probably from \textit{*de{\hi}--}, e.g. Skt. \textit{d\=a--}, Gk. {\greektext d'idhmi} \bit{uwa\v{s}} \p{c.gen.s.} `nurse' (?) \bit{uku\v{s}} (??) 



\end{notes}



\end{description}

%% the rest of the text is too broken and fragmentary to bother with


\section{Annals of Hattu\v{s}ili I}

\begin{description}

\item[Publication:] 
\item[Edition:] \citet{demartino2003hittite}
\item[Translation(s):] 
\item[Background:] 

\end{description}

\subsection{Vs. I}

\begin{description}

\item[1--10 :] The great-king, the \textit{tabarna} of the land of Hatti, ruled as the son of brother of the \textit{tawananna} (queen). He went into the city of Sanawitta, and destroyed him and destroyed his land. I left behind troops in two places as a garrison. Whatever (cattle) pens there were, I gave them to the troops in the garrison. Afterward I went to Zalpa, and I destroyed it and I took up its gods.

\begin{notes}

\textbf{LUGAL\bit{-\=ezziyat}} \p{3s.pret.act.} `exercise kingship; rule'; denom. \textit{{\hith}a\v{s}\v{s}uwezziya--} $\leftarrow$ \textit{{\hith}a\v{s}\v{s}uwezzi--} \p{n.}`kingdom' \bit{A\v{S}RA} `place, position' \bit{a\v{s}andulanni} \p{dat.s.n.; r/n-st.} `garrison'; \textit{r/n}-stem attested only in oblique; semantically equivalent to thematic \textit{a\v{s}andula--} \bit{a\v{s}auwar} \bit{a\v{s}awar} \p{acc.s.n.; r/n-st.} `(cattle) pen; fold'

\end{notes}

\item[11--21 :] And I brought them to the Sun-goddess of Arinna. I brought one silver cow (statue/vase) and 1 silver fist (statue/vase) into the temple of the Storm-god. And the gods which remained, I brought them into the temple of the god Mezzulla. In the (next?) year, I went into the city of Alalha and I destroyed it. Afterward I went into the city of Warsuwa. From the city of Warsuwa, I went into the city of Ikakala. From the city of Ikakala, I went into the city of Tashiniya. I destroyed these lands, but I took up its [= their] good(s), and I filled up my house with good(s). 

\begin{notes}

\bit{\=a\v{s}\v{s}u} \p{n.acc.coll.} `good'; however, coll. pl. should originally show plene final vowel \textit{--\=u}

\end{notes}

\item[22--32 :] In the (next) year, I went into the city of Arzawa, and I took up their cows (and) sheep. But the Hurrian enemy came into the land behind me, and all the lands became hostile against me. Hattusa alone was the one the city that remained (faithful). The great-king, the \textit{tabarna} is dear to the Sun-goddess of Arinna. And she set me on her knees, and took me by the hand, and ran before me in battle. I went into the city of Nenassa for battle, and when the men of the city of Nenassa came opposite me, they opened up (the gates). 


\begin{notes}

\textbf{GU\subsc{4}\mpl}\bit{-un} \p{acc.s(?).c.} `cow'; Sumerogram indicates plurality, but phonetic complement is clearly singular; similarly, in following `sheep' \textbf{EGIR}\bit{-azziyaz} \p{pp.} `behind' + \textit{=mu} \bit{NARAM} `beloved, dear' (Akk. \textit{ra'\=amu}) \bit{{\hith}ali\v{s}ta} \p{3s.pret.act.} `set in motion'; or possibly to \textit{{\hith}aliya--} `kneel down', but there are too many syntactic arguments 

\end{notes}

\item[33--45 :] Afterward I went into the land of Ulma for battle, and twice the men of the land of Ulma came against me in battle, and twice I fought them. Then I destroyed the land of Ulma, and in its place I (?) weeds. And I brought seven gods into the temple of the Sun-goddess of Arinna. (There was) one silver cow (statue/vase) for the goddess Katiti on Mt. Aranhapilanna. The gods which remained on Mt. Aranhapilanna, I brought them into the temple of the god Mezzulla. When I came back from the city of Ulma, I went into the land of Sallahsuwa, and the land of Sallahsuwa itself under fire [lit. `with fire'] yielded [lit. `let down'] . They turned those ones over to me for servitude, and I came back to my city Hattusa. 
  
\begin{notes}

\textbf{Z\`A.A{\Hith}.LI\supersc{SAR}} `cress; weed' \bit{\v{s}unniyanun} \p{1s.pret.act.} `?' \textbf{IZI}\bit{-it} \p{n. instr.} `fire' \bit{katta tarna\v{s}} \p{3s.pret.act.} `let down; yield (?)' \bit{wa{\hith}nuir} \p{3pl.pret.act.} `turn; change (over to)'; distinction between referent of object \textit{ap\=u\v{s}} and subject of trans. verb is unclear to me

\end{notes}

\item[46--59 :] In the (next) year, I went into the city of Sanahhuitta for battle. I did battle five against the city of Sanahhuitta for 5 months, and in the sixth month I destroyed it. And I appeased the soul of the great king, and within (?) the lands the Sun-goddess took (her) place. The heroic (glory) which{\ldots}I brought to the Sun-goddess of Arinna (?){\ldots}and the wagons of the land of Appaya I fought (?). Then I took up from in front (?) the cattle (and) sheep of the city of Taksannaya. Then I went into the city of Parmanna. Parmanna was the head of (?) those kings, and that one leveled (?) the paths before them. 


\begin{notes}

\bit{war\v{s}iyanun} \p{1s.pret.act.} `appease; satisfy; make content' \textbf{EGIR}\bit{-panda} \p{pp} here, `within' \textbf{L\'U}\bit{-natar=ma} \p{n.acc.s.} \bit{{\hith}uliyanun} \p{1s.pret.act.} `fight'; what is CM's emendation here? \bit{peran sar\=a da{\hith}{\hith}un} \p{1s.pret.act.} `take up from in front (?)'; idiomatic complex preverb combination? \bit{tak\v{s}anni\v{s}kit} \p{3s.pret.act.} `level; flatten'; denom. \textit{tak\v{s}atar} \p{n.} `plane; plateau'; both nominal base and verb show sporadic \textit{--tn--} > \textit{--nn--} assimilation; what is CM's exclamation point?

\end{notes}

\item[60--73 :] And when the came opposite me, they opened up the gates. And in that ??? the Sun-god of heaven took them by the hand. The land of Alha became hostile toward me, and I destroyed the city of Alha. In the (next) year, I went into the land of Zaruna and I destroyed Zaruna. Then I went into the city of Hassuwa. The men of the city of Hassuwa came against me, and their troops had [lit. `were with'] the aid of the land of Aleppo. Then it came to me for battle, and I fought it. In a few days I crossed the river Purana and, like a lion, I trampled the land of Hassuwa with (my) feet. 


\begin{notes}

\bit{memini} \p{?.loc.s.} `?' \bit{anda} \p{adv.?} what is this doing here? \bit{\v{s}ardianni} \p{n.d-l.s.} `help; aid'; cf. \textit{\v{s}ardiya--} \p{c.} `helper' \bit{kappuwandas} \p{adj.d-l.pl.} `few; countable' \bit{\v{s}akkuriyanun} \p{1s.pret.act.} `overpower; lay low; trample\supersc{?}'; meaning per CHD; collocated with G\`IR `foot' as here suggests `trample' \textit{ad sensum}

\end{notes}

\item[74-- 81:] 

\begin{notes} 

\bit{aruzza} ?

\end{notes}

\item[82--90 :] 2 cows of silver (and) 13 statues of silver and gold. 2 sacred precincts (?) and (the part of) the wall which (was) behind the god, I trimmed it (?) out with silver and gold. And its door I fitted out with silver and gold. 1 table trimmed with gold. 2/3 tables trimmed with silver. 2 silver tables. 1 gold throne trimmed with golden (?) {\ldots} of gold. 1 gold ??? {\ldots} 2 {\ldots} of stone fitted out with gold. I brought these gods of the city of Hassuwa to the Sun-goddess of Arinna.


\begin{notes}

\bit{{\hith}amrita} \p{n.nom./acc.pl.} `(Hurrian) sacred precinct/area' \bit{kutta\v{s}} \p{c.gen(?).s.} `wall'; de Martino seems to take as nominative with \textit{kui\v{s}}, but formally looks like genitive; also, no reason not to assume geminating \textit{=a} in original \textbf{EGIR}\bit{-izziyan} \p{pp.} `behind'; nominal sentence with unexpressed copula (?) rather than \textit{kitta} with de Martino (?) \textbf{{\wood}IG}\textit{=ya} \p{acc.s.} `door' \bit{TAML\=U} `border; edge' \textbf{{\wood}GU.ZA} `throne' \bit{NEMEDI} ?? \bit{MAYALTUM} `??'; A has {\wood}\textit{MADNANU} `bedframe' \bit{GAR.RA} `fitted out/equipped with' 

\end{notes}

\item[91--106 :] (For) the daughter of Allati, Hebat, (there are) 3 silver statues. (But) these 2 golden statues I brought into the temple of Mezzulla. 1 silver spear, 1 gold staff, 5 silver weapons, 3 axes of lapis lazuli, 1 axe of gold---(all) this I brought into the temple of the Storm-god. And within one year I overcame the land of Hassuwa. They cast off the spear of Tawannaga, and I, the great king, cut off his head. And I went into the city of Zippasna, and during the night I ``went over'' the city of Zippasna.  And I entered into battle them. And I moved dust up over them (?), and within the lands (?) the Sun-goddess took her place. 


\begin{notes}

\bit{k\I=ma} \p{n.acc.s.} `this'; neut. for pl. in a list; a nice example of contrastive topicalization with \textit{=ma} \bit{IMITTU} `lance, spear (??)' \textbf{GIDRU} `staff, scepter' \bit{{\Hith}\=URPAL\=U} `axe, cudgel (??)' \textbf{{\stone}ZA.G\`IN} `blue stone; lapis lazuli' \bit{m\=arin} \p{c.acc.s.} `lance; spear' \bit{LUGAL.GAL=ma=an=k\'an} schema, with clitic obj. \textit{=an} + acc. of respect SAG.DU=\textit{S\'U} `his head' (?) \textbf{SA{\Hith}AR}\bit{-i\v{s}} \p{n.acc.s.} `dust' \bit{anda} +gen (??)

\end{notes}


\item[107--119 :] I, the great king, the \textit{tabarna}, went into the city of Zippasna. Like a lion, I looked very angry (?) at Hahha (?), and I destroyed the city of Zippasna. I took up its gods, and I brought them to the Sun-goddess of Arinna. Then I went into the city of Hahha and three times I did battle amid the city gates. And I destroyed Hahha, and I took up its goods and I brought them off to Hattusa, my city. 2 sets of wagons were loaded up with silver.


\begin{notes}

\bit{ar{\hith}a tarkuwalli\v{s}kinun} \p{1s.pret.act.} `look very (?) angry' \bit{TAPAL} `pair, set' \bit{t\=ai\v{s}tiyan} \p{ptcpl.n.nom.s.} `load (?)'; singular neut. agreement seems odd with c.pl. subj.

\end{notes}

\end{description}


\section{The Soldier's Oath (CTH 427)}

\begin{description}

\item[Publication:] First tablet: (D) KBo XXI 10; (C) KUB XL 13; (E) unv. Bo 6881. Second tablet: (A) KBo VI 34+ Bo 2523; (B) KUB XL 16+ VIII 59 + 342/u + 797/v + 1087/z; (C) KUB XL 13 Rs
\item[Edition:] \citet{oettinger1976eide}
\item[Translation(s):] 
\item[Background:] 

\end{description}

\subsection{Vs I}

\begin{description}

\item[1--16{\pr} :] He places the cedar in their hand(s) {\ldots} they {\ldots} and ?? {\ldots} he {\ldots} and he says to them {\ldots} does not it {\ldots} The house of the gods {\ldots} like cedar {\ldots} his refreshment {\ldots}

\begin{notes}

\textbf{{\wood}ERIN} `cedar'; restored \bit{warsula\v{s}=\v{s}i\v{s}} \p{c.nom.s.} `(his) refreshment'; thematic derivative to \textit{war\v{s}ul} \p{n.} `id'


\end{notes}

\item[17{\pr}--30{\pr} :] And he says: ``Whe[n] he [wa]s alive, he found the sky above. But now they have blinded him in place of the oath. Whoever transgresses (against) these oath gods and sets up a trap for the king of the land of Hatti and sets his eyes inimically on the land of Hatti, let these oath-gods seize that one and let them blind the army(-camp) of that one also! Furthermore let them make them deaf! The one shall not see the other, and this one shall not hear that one! Let them give an evil doom to them! Let them shackle their feet with shackles below, and also let them bind their hands above!

\begin{notes}

\bit{kuit} temporal? \bit{pedi} \p{n.d-l.s.} `place'; in what sense here? \bit{dasuwa{\hith\hith}er} \p{3pl.pret.act.} `blind' \bit{appali} \p{c.loc.s.} `ambush; trap' + \textit{dai--} = `?' \bit{nu=za=an} refl. + rare loc. \textit{=an}? I don't see how it could be \textit{=\v{s}an}, since I see no source of the necessary \textit{-t-} \bit{tuzzin} \p{c.acc.s.; i-st.} `army(-camp)'; in what sense here? \bit{duddumiya{\hith\hith}andu} \p{3pl.impv.act.} `make deaf'; simplex adjective unattested, but cf. \textit{duddumili} \p{adv.} `quietly; secretly' \textbf{\man}\bit{ara\v{s}} \textbf{\man}\bit{aran} `one{\ldots}the other'; by itself, \textit{ara--} is simply `friend; comrade'; why does it mean this in this expression? \bit{patalliyandu} \p{3pl.impv.act.} `(lit.) foot-bind; shackle'; denom. \textit{patalla--} `foot-binding; shackle'; here in figura etymologica with nominal base \bit{serr=a=a\v{s}} \p{adv} `above'; note contrast between clause-initial `above' and previous `below'; clearly c. acc. pl. pron. clitic \textit{=u\v{s}} only preserved in \textit{n=u\v{s}}, elsewhere there is only \textit{=a\v{s}}; use of geminating \textit{=a} seems to be generally correct so far

\end{notes}

\item[31{\pr}-- :] As the oath-gods bound hand and foot the armies of the land of Arzawa and make them into piles (?), let them also bind the armies of that one and make them into piles (?).


\begin{notes}

\bit{tuziu\v{s}} \p{c.acc.pl.} `army(-camp)'; looks like schema, with body parts as accusative(s) of respect; this is an inner-Hittite innovation \bit{{\hith}arpu\v{s}} \p{c.acc.pl.} `heap, pile'; \citet[79]{oettinger1976eide} renders adverbially ``zuhauf'' \bit{d\=ay\=er} \p{3pl.pret.act.} `place'; the exceptional full-grade contrasts with expect zero-grade in the main clause

\end{notes}


\item[{\ldots}]


\item[5--18 :] Then he takes salt and sinew from their hands [or places into?] and casts it into the fire, and he speaks as follows: ``As this sinew is scorched in the hearth, as salt breaks up in the hearth, whoever transgresses these oaths and entraps the king of the land of Hatti and puts his eyes on the land of Hatti inimically, let these oaths seize him, and let him be scorched like sinew and broken up like salt. And also as there is no seed of salt, so let the name, the seed, the house, the cows, (and) the sheep of that man perish.''


\begin{notes}

\textbf{\supersc{UZU}SA} `sinew; muscle; cord'; \textbf{MUN}\bit{-an} \p{c.acc.s.} `salt' \bit{{\hith}ur\v{s}aknietta} \p{3s.pres.mp.} `char; scorch' (?); translation contextual? \bit{par\v{s}ittari} \p{3s.pres.mp.} `break up; crumble; dissipate' \bit{\v{s}arradda} \p{3s.pres.mp.} `transgress'; archaic trans. mp. preserved here \textbf{NU.G\'AL} = \textit{natta e\v{s}zi} \textbf{UK\`U}\bit{-\v{s}i} \p{c.d-l.s.} `man' = \textit{antu{\hith}\v{s}i} \bit{\v{S}UM=\v{S}U} `(his) name' 

\end{notes}


\item[19--30 :] Then he places malt (and) wort into their hands, and they lick it. And he speaks as follows: ``As they grind this malt (and) wort with a millstone and they mix it with water and they cook it and they pulverize\supersc{?} it, whoever transgresses these oaths and fashions evil for the king, the queen, the children of the king (or) the land of Hatti, let these oaths seize him, and thus let them crush his bones and let him (!?) heat it up and let him (?!) pulverize\supersc{?} it, and let him bear an evil doom. But those ones say: ``So be it.''



\begin{notes}

\textbf{BUL\`UG} `malt' \textbf{BAPPIR} `(beer-)wort' \bit{{\hith}arranu\v{s}kanzi} \p{3pl.pres.act.} : \textit{{\hith}arra--} `crush; pulverize'; \textit{nu}-transitive only in this context with unclear relationship to simplex \bit{inu\v{s}kidu} \p{3s.impv.act.} ? \bit{{\hith}a\v{s}tai=\v{s}i[ti]t} \p{n.acc.pl.} `bone'; expected OH \textit{=\v{s}et}; note insertion of <\textit{-ti-}> as scribal error \bit{inu\v{s}kidu} \p{3s.impv.act.} `heat, make warm; cook' (iter.); 3s. is unexpected for 3pl. here and in subsequent clauses \textbf{\'U\v{S}}\bit{-k\'an} \p{n.acc.s.} `doom' = \textit{{\hith}enkan}

\end{notes}


\item[31--41 :] As there is no reproductive power in this malt and they do not bring it (??) into the field and they make it (??) (into) seed, but they do not make it (??) into bread and they place it into the `house of the seals,' whoever transgresses these oaths and fashions evil against the king, the queen, (or) the children of the king, let these oaths thus destroy his future days. Let his wife not beget son(s) (or) daugher(s). Let the grass not grow (??) for him in the open field, the ploughed field, (or) the meadow. Let his cows and sheep not beget calves (or) lambs.''



\begin{notes}

\bit{{\hith}a\v{s}\v{s}atar=\v{s}et} \p{n.nom.s.} `reproductive power; kinship'; is the poss. cl. really expected here? \textbf{A.\v{S}\`A}\bit{-ni} `(plowed) field' \bit{yenzi} \p{3pl.pres.act.} `do; make'; double object construction (?) \textbf{{\stone}KI\v{S}IB} `seal' = \textit{\v{s}iyatar / parzaki--}; + \'E = ` ``house of the seals'' / treasury' \bit{{\hith}ar\v{s}auna\v{s}} + A.\v{S}\`A = `plowed field' \textbf{L\'IL} `open field' = \textit{gimra--} \bit{welkuwan} \p{n.nom.s.} `grass; herb' \bit{{\hith}uwai} \p{3s.pres.act.} `run, flee; grow (?)'; it seems to mean `grow' in context, but I've never heard of this meaning \textbf{AMAR} `calf' \textbf{SIL\'A} `lamb'

\end{notes}


\item[42--54 :] Then they bring the garment of a woman, a distaff, and a spindle and they break a reed. And you speak as follows to them: ``What is this? Are they not the luxurious garments of a woman? Do we have them (??) for oath (??)? Whoever transgresses these oaths and fashions evil for the king, the queen, (or) children of the king, let these oaths make him, a man, into a woman (??). {\ldots}



\begin{notes}

\textbf{\wood}\bit{{\hith}ue\v{s}ann=a} \p{acc.s.} `spindle' \textbf{GI}\bit{-an} \p{c.acc.s.} `tube; reed' \textbf{\supersc{T\'UG}N\'IG.L\'AM{\mpl}} `fancy/luxurious garment' \bit{yendu} \p{3pl.impv.act.} `make'; a very curious double object construction, with a kind of clitic doubling (??)

\end{notes}

\bigskip
\item[Rs. III]
\bigskip

\item[1--23 :] Then they lead a blind (and) deaf woman before them. And to them you speak as follows: ``This woman is blind (and) deaf. Whoever fashions evil for the king (or) queen, let the oaths seize him and make him, a man, into a woman. Let them blind him like the blind, (and) let them make him deaf like the deaf. Let them destroy him together with his wife and sons within ???.'' Then (someone) places a statue full of water inside in their hands, and speaks as follows: ``Who is this? Has he not sworn (an oath)? He has sworn before the gods, then (subsequently) he has transgressed the oaths. The oaths seized him, so that he swelled up inside and raised up (his) belly before his hand (???). Whoever transgresses these oaths, let the oaths seize him, and let him be swollen within, and the son within his inside (??) let Ishara {\ldots} and let them devour him (???).


\begin{notes}

\textbf{{\man}IGI.NU.G\'AL} `blind' (lit. `see' + `not') \textbf{{\man}\'U.{\Hith}\'UB} `deaf' \bit{pankur=\v{s}it} \p{} ??? \bit{widanda} \p{n.abl.} `water'; archaic instr. \textbf{\v{S}A\textit{-\v{S}U}} \textit{nu{\ldots}namma} is this an unmarked conditional apodosis? \bit{\v{s}arratta} \p{3s.pret.(?)act.} `transgress'; a real preterite despite the original mps. previously? \bit{\v{s}uttati} \p{3s.pret.mp.} `swell' \bit{\v{s}ar{\hith}uwandan} \p{c.acc.s} `stomach, belly; womb; offspring' \bit{peran} \textbf{UGU} \bit{karpan} = \textit{peran \v{s}ara karpan {\hith}arzi} \p{3s.perf.act.} `raise up before' \bit{andurza} \p{adv.} `within; inside'

\end{notes}


\item[24--35 :] The he holds (it) forth, and damages (it with respect to) its eyes. They trample it with (their) feet, and one (?) speaks to them as follows: ``Whoever transgresses these oaths, let the gods of the land of Hatti thus come trample the city of that one with (their) feet, and let them make (their) cities empty. They blow a bubble and they trample it (?) with (their) feet, and it (?!) is gradually released. And he says: (``) As this has become empty, whoever transgresses these oaths, let the house of that one thus become empty of humans, cows, and sheep.'' 


\begin{notes}

\bit{huwapp\=ai} \p{3s.pres.act.} `treat badly; injure, harm; damage' \bit{i\v{s}parranzi} \p{3pl.pres.act.} `trample; kick (with feet)' \bit{uwandu} \p{3pl.impv.act.} `come'; QSV \index{quasi-serial verb construction} is likely here, since there is no subject clitic with \textit{uwa--} (unless it is fronted with overt subject `gods', but the absence of \textit{=ma} speaks against this) \textbf{URU}\bit{-ya\v{s}e\v{s}\v{s}ar} \p{n.acc.coll.} `city'; morphology? \bit{walulan} \p{c.acc.s.} `pupil (of eye); bubble' (?) \bit{pariyanzi} \p{3pl.pres.act.} `blow; inflate' \bit{pa-ra-a-a\v{s}} big controversy here; `air' per Oettinger? discussed by Dunkel, who argues that this is the preverbal \textit{\=amre\textsubdot{d}ita} \textit{*pro pro}, with a weirdly low subject clitic (following Friedrich) \bit{\v{s}annapilie\v{s}ta} \p{3s.pret.act.} `become empty'; fientive to \textit{\v{s}annapili--} `empty; lonely' 

\end{notes}


\item[36--48 :] And you place down before them a stove, and also down before the image you place a plow, a large cart, (and) a wagon. Then they break them into bits, and one speaks as follows: ``Whoever transgresses these (!) oaths, let the Storm-god break his plow into bits. As grass does not come up from a stove, let not barley (or) grain come up from the field of that man. Let scrub brush go up (instead). And you give to them a red hide and he says: ``As they make this red skin bloody, let not the bloody-redness go away from him.


\begin{notes}

\textbf{IM.\v{S}U.N\'IG.NIG\'IN.NA} `stove; oven' \textbf{{\wood}APIN} \textbf{{\wood}M\'AR.G\'ID.DA} `wagon; cart (for hauling)' \bit{{\hith}imma} \p{c.gen.s.} `(ritual) image; substitute' \bit{=ma} whoa, unexpected position! \bit{welku} \p{n.nom.s.} `grass'; note regular nominative case with neuter as subject of unaccusative/(unergative?? ask CM!) verbs \textbf{\v{S}E\supersc{\textit{AM}}} `grain' \textbf{A.\v{S}\`A\textit{-\v{S}U}} `field'; intrinsic possession? \bit{za{\hith\hith}eli} \p{n.nom.s.} `scrub brush'; also written \textit{{\hith}a{\hith\hith}ali}, hence almost certainly a loanword \bit{iyataru} \p{3s.impv.mp.} `go; march' \textbf{KU\v{S}.SA\subsc{5}} `hide/skin' + `red' 

\end{notes}

\bigskip
\item[Rs. IV]
\bigskip

\item[1-- :] Let the oaths thus take you in (?), and let it/them (?) not go away from you (?). 

\begin{notes}


\end{notes}

\end{description}


\section{Ritual of Babilili (CTH 718)}

\begin{description}

\item[Publication:] 
\item[Edition:] \citet{beckman2014babilili}
\item[Translation(s):] 
\item[Background:] -the big important thing is that this *could* be an *original* New Hittite ritual composition

\bigskip
\item[Rs. IV]
\bigskip

\item[\S1 (1--9) :] On the second day, at [dawn, when the sun] has [not yet] risen, the ritual patron [bathes] himself. [And also the priests] who [had been summoned in] for the ritual patron bathe themselves. The priests [ {\ldots} ] Then they wash the temple [ {\ldots} But] they stuff(?) the [ {\ldots} ] \textit{uli{\hith}i} and the linen cloth with which [it had been covered into {\ldots} ] And when [night falls and a star] becomes visible, then [they perform the {\ldots} ]-ritual.


\begin{notes}

\bit{lukkatta} \p{adv.} `in the morning; at dawn'; cannot be the (original!) verb; rather, must be the (secondary) adverb; on the development, see \citet{vijunas2009tstems} and \textit{CHD} (s.v.); CM doubts the restoration, probably more likely would be \textit{lukkatti} or \textit{nu} before \supersc{d}UTU\textit{-u\v{s}}; but the latter would be archaic, and problematic for the ``new composition'' hypothesis \bit{\=upzi} \p{3s.pres.act.} `rise, go up'; contra GB's restoration, (frustratingly) never spelled  with $<$ \'u $>$ \bit{warpzi} \p{3s.pres.act.} `wash; bathe' \bit{ap\=e=ya} \p{pron.c.nom.pl.} `those'; a potential archaism; \textit{ap\=us} as nom. pl. eventually displaces older \textit{ap\=e} \textbf{GAD} `cloth' \bit{anda kariyanza} \p{ptcpl.c.nom.s.} `cover; conceal' \bit{ku\=ez} \p{abl.} `which'; use of abl. for instr. is good new Hittite; likely an embedded relative here, conjoined  \bit{s\=a{\hith}anzi} \p{3pl.pres.act.} `clog; stuff (up/in)' \bit{watkuzzi} \p{3s.pres.act.} `leap; (a)rise' 

\end{notes}

\item[\S2 (10--21) :] Then they take these things: two garments---one [ {\ldots} ]-garment [and one
{\ldots}-garment]---four shekels of silver, a little fine oil, three [ {\ldots} ]-garments, [one tarpala] of red [wool], one tarpala of blue wool, one tarpala [of green wool], three linen \textit{gazzarnul}, three woolen \textit{ki\v{s}ri}, [three \textit{kure\v{s}\v{s}ar}, (a quantity)] of ghee, three narrow-necked jars, three [jugs of wine], three reed trays, one set of [ {\ldots}, (so many) {\ldots}-loaves] of one-half quart (of flour), one G\'UG-loaf of one-half [handful, (so many) {\ldots}-loaves] of one-half handful, three ordinary loaves of one tarna-measure (each), three [ {\ldots} -loaves], thirty unleavened breads, three dry(?) loaves, three [sheep, and {\ldots} ] By the river [they invoke] Pirinkir and into the temple [they {\ldots} ] six substitute sheep. And when they have prepared [the ritual requisites],


\begin{notes}

\bit{k{\i}} \p{n.acc.pl.} `these'; originally \textit{k\=e} plural and \textit{k{\I}} singular, but distribution later obscured; strong sign of NH composition \textbf{S\'IG SA\subsc{5}} `red wool' \bit{{\Hith}AZARTU} `green' \bit{m\=uganzi} `rouse, stir to action'; meaning per \citet{melchert2010mugai} \bit{nakku\v{s}\v{s}i\v{s}} \p{c.nom.s.} `(ritual) substitute'; but case is unexpected, possibly `default' into nom. s. when extraposed but this is conjectural \textbf{\v{S}U.TI} `ritual stuff'

\end{notes}


\item[\S3 (22--32) :] {\ldots}and furthermore, while [the sun] is still high, then a katra-woman takes one narrow-necked jar, two [unleavened] breads, one jug of wine, and a little fine oil and goes to draw the water of purification. When she arrives at the spring, she breaks up the unleavened breads and throws them down into the spring. She pours the wine and dribbles the fine oil down (into the spring) and draws the water. She carries it up to the gate-house and places it on a reed \textit{puriya}. She ties $<$one$>$ linen gazzarnul, [one] tarpala of blue wool, one tarpala of green wool, [one] tarpala of red wool, and one shekel of silver onto the narrow-necked jar.

\begin{notes}

\bit{namma} \bit{n\=uwa} \p{adv.} `still' \bit{\v{s}er} \p{adv.} `up; high' \bit{wetena\v{s}} \p{n.acc.pl.} `water'; almost always occurs in the plural, though formally ambiguous between s./pl. here; cases like this with `go/come' and dat.-loc. objects could be locus of reanalysis between `go to X to V' and the real infinitival reading `go to V X' \bit{{\hith}anuwanzi} \p{inf.} `scoop' : \textit{{\hith}an--} ; \textbf{P\'U} `well; spring' \bit{par\v{s}iyazzi} `break'; against GB, not `crumbles' (which is \textit{par\v{s}ai--}); for CM, this shows renewal of \textit{par\v{s}iya--} as ordinary \textit{--ye/o--} verb \bit{zapnuzzi} \p{3s.pres.act.} `trickle'; likely sound symbolic \bit{{\hith}ani} \p{3s.pres.act.} `draw'; the 3s. in \textit{--i} (not \textit{--ai}) is archaic \bit{{\hith}amanki} \p{3s.pres.act.} `tie (on); bind'

\end{notes}


\item[\S4 (33--44 :] When a star becomes visible, the priest covers a reed table with a linen cloth, ties a \textit{kirinni}-stone to the uli{\hith}i, and places them on the table. Furthermore, she takes a reed tray [and] arranges the following on it: [She takes] a PURSITU-vessel and positions [lue\v{s}\v{s}ar on it. Beneath the PURSITU-vessel] she places one woolen ki\v{s}ri, [while on top] she places [one white kure\v{s}\v{s}ar]. In addition, [she takes] blue wool, [green wool, and red wool] and drops it in(to the PURSITU-vessel(?)). [She puts it on top of the lue\v{s}\v{s}ar {\ldots} ] lue\v{s}\v{s}ar [ ? And] the priest [speaks the(se) words as follows in Akkadian]: \\

(approx. 20 lines lost)

\begin{notes}

\textbf{\man}\bit{\v{s}aknies} \p{c.nom.s.} `priest'; there are a number of nom. s. spelled with $<$--es$>$ as though plural in this text \bit{=at} the referent (or both referents) seem to be animate, so the resolution with neut. pl. would be very unexpected; it is possible that the reading of the sign should be otherwise \bit{lue\v{s}\v{s}ar} `incense; (aromatic) wood shavings'

\end{notes}

\item[\S\S5{\pr}--9{\pr} (1--17) :] {[}When] the priest [has finished] speaking [the(se) words in Akkadian, then he brings] the hand water [out of] the palace. [Then the priest speaks as follows] in Akkadian: ``[Wash] your hands, My Lady, Queen! [Let your fingers] feed [morsels to] your lips!'' [Then the priest] drops the lue\v{s}\v{s}ar (into the fire) [and] with the silver beaker [holds out] the hand water [toward the deity]. And [with the very same] beaker he pours out the hand water for the ritual patron. Then the priest [speaks] as follows in Akkadian: ``Wash your hands, My Lady, Great Queen! Let [your fingers] feed [morsels] to your lips!'' When the priest has finished speaking the(se) words in Akkadian, then the priest takes a silver beaker filled with beer and puts it in the hand of the ritual patron.

\begin{notes}



\end{notes}

\item[\S10{\pr} (18--28) :] But while the lue\v{s}\v{s}ar is burning, the musician sings as follows in Akkadian: ``This that you asked of me: O Lady of the Land, you asked of me; O Lady of the Lands, you asked of me; O Queen of Heaven, you asked of me. The receipt of honors you asked of me. The possessions of your(!) father you asked of me, A lamb you asked of me. Before the ayakku-shirine you answered me. You heard (my plea) for your intercession. You ?, you entered(?) to me, you ? me, ? (Even) by day you are powerful!''

\begin{notes}



\end{notes}

\item[\S\S11{\pr}--14{\pr} (29--49) :] And when the musician has finished speaking the(se) words in Akkadian, then the priest begins to clap. Then the priest [takes an unleavened] bread and crumbles it thoroughly. Then he puts [it] on the lue\v{s}\v{s}ar. [Afterwards] he takes one fish [and] waves [it] over the deity, then puts [it] on the lue\v{s}\v{s}ar. Afterwards he takes another fish and waves it over the ritual patron, then puts it on the lue\v{s}\v{s}ar. He picks up the PURSITU, holding the woolen ki\v{s}ri below, and holds it out to the ritual patron. Afterwards, the priest takes the white kure\v{s}\v{s}ar, and he holds it above the ritual patron, above the silver beaker of beer that the ritual patron holds. And when the lue\v{s}\v{s}ar [ ? ] away, he waves it over the deity. And [while] he is waving the lue\v{s}\v{s}ar, [the musician] sings as follows in Akkadian: ``[ ? ] O Lady [of the Land(?), release my offense(?)!] O Queen of Heaven, release my sin!''

\begin{notes}



\end{notes}

\item[\S\S15{\pr}--16{\pr} (50--57) :] And [when the musician] has finished [speaking] these words in Akkadian, then the priest picks [up] the lue\v{s}\v{s}ar and sets it down on the reed tray. [Then the priest] takes the beakerg away from the ritual patron and libates on the lue\v{s}\v{s}ar, extinguishing it. Then another [priest] picks it up and strews it into the fire.

\begin{notes}



\end{notes}

\item[\S\S17{\pr}--18{\pr} (58--65) :] {[} ? ] The priest prepares [these implements: He takes] two silver beakers. He fills [one] with beer [and one] with water, [and] sets [them] down on the [reed] tray. Further, he takes from the reed tray the unleavened bread upon which the salt was strewn, and he sets the knife on top.i Then one sheep is driven in, and the priest takes the silver beaker of water. He dunks the unleavened bread and the knife (therein) and holds out the hand water to the deity. He sprinkles (it) on the sheep and [the ritual patron]. And [from] this silver beaker [the priest pours] out the hand water for the ritual patron.


\begin{notes}

{\ldots}there is more{\ldots}

\end{notes}

\end{description}

\section{The Expansion of the Cult of the Deity of the Night (CTH 481)}

\begin{description}

\item[Publication:] 
\item[Edition:]  %jared miller, StBoT
\item[Translation(s):] 
\item[Background:] 

\item[\S\S1--2:] Thus (speaks) the Priest of the Deity of the Night: ``When, in which temple of the Deity of the Night a man becomes (a man) of the Deity of the Night, it comes about that he builds another temple of the Deity of the Night on the basis of (=`from') that (original) temple of the Deity of the Night, and in addition, he sets up the deity separately, until he finishes setting up everything, (then) the smiths make a deity of gold. Just as the ritual (paraphernalia) of the deity (was made), so they set (her/it)\supersc{?} to (ritually) make (???). And also just as things (=`beads'\supersc{?}) of \textit{k-}stone, of silver, of gold, of lapiz lazuli, of carnelian, of Babylon-stone, of chalcedony\supersc{?}, of quartz\supersc{?}, (and) of alabaster, solar disks, (a symbol of) life, and a comet\supersc{?} of silver and gold (are) placed behind, so they set them up to worship\supersc{?}. 

\begin{notes}

\bit{a\v{s}nuzi} \p{3s.pres.act.} `take care of; be done with, finish'; takes inf. complement here \bit{{\hith}\=umandazziya} \bit{n.acc.pl.} `all, every'; subst. \textit{*--iyo--} adj. \bit{aniur=\v{s}et} \p{n.acc.s.} `ritual (paraphernalia)'; CM seems to understand both the noun and the verb (generically, `do; treat') in the specialized sense `of ritual paraphernalia/gear' and `to gear up (as such)' \bit{tittanuwanzi} \p{3pl.pres.act.} `install; set up'; note the object clitic \textit{=an}---what is the referent? it cannot be \textit{aniur}, which is neuter; could it be the deity? I wonder if this could read "As goes (= as per) the ritual of the deity, so they set her/it up to perform the ritual.''; the verb is transitive and so should have a real direct object; all this needs to be untangled \textbf{EGIR}\bit{-an i\v{s}garanta=ya=\v{s}\v{s}i} \p{n.nom.pl.} `stuck/place behind'; the position of the clitic \textit{=ya} seems unexpected to me, I would've predicted after the preverb---phrasal domain? \textbf{{\stone}GUG} `carnelian; sard (type of red, semi-precious stone)' \textbf{{\stone}N\'IR} `chalcedony (?)'; \textbf{{\stone}DU{\Hith}.\v{S}\'U.A} `quartz (?)' \textbf{{\stone}A\v{S}.NU\subsc{11}.GAL} `alabaster' \textbf{A\v{S}.ME{\hpl}} `solar disk' \textit{wannuppa\v{s}tali\v{s}\v{s}=a} \p{c.nom.s.} `comet; shooting star (?)' \textit{=a\v{s}} \p{cl.c.acc.pl.} `them'; note animate resumption after list of neuters---nearer agreement with \textit{w--}? but \textit{w--} is also sing.; also, the acc. pl. in \textit{=a\v{s}} after \textit{nu} is a late feature


\end{notes}


\item[\S\S3--4:] (There is) 1 solar disk of gold worth 1 shekel---Pirinkir (is) its name---1 `navel' of gold, 1 set of of gold \textit{p--}---they are set with Babylon-stone. And the priest consigns them to the smiths (as)\supersc{?} ritual paraphernalia. And also (things) of gold, silver, lapiz lazuli, of carnelian, of Babylon-stone, of quartz\supersc{?}, of chalcedony\supersc{?}, of alabaster, (and) 1 `bringing forth' of stone (are) placed behind{\ldots} {[}a bunch of other junk in \S4{\ldots}]

\begin{notes}

\textbf{LI.DUR} `navel' \bit{lamniyazi} \p{3s.pres.act.} `name'; here, in the sense `consign' \bit{aniur=\v{s}et} syntax?? \bit{peran peduna\v{s}} `carrying before; bringing forth'; behaves as a compound (also w/ \textit{peduma\v{s}})  {\ldots}

\end{notes}

\item[\S\S5--6:] [a bunch more ritual junk here{\ldots}]

\begin{notes}


\end{notes}

\item[\S\S7--8:] [a bunch more ritual junk until end A i 50] When they finish `(ritually) gearing up' the goddess, they arrange all this also in (its) place. And the one who sets up the goddess separately, that one (is) the ritual client. During the preceding day the priest and the \textit{k}-women wash (themselves), and that day goes (by). 

\begin{notes}

\bit{lukkatta=ma} cannot mean `when it became light' or `at dawn' anymore; must be adv. `on the next day' \bit{ap\=a\v{s}} \p{nom.s.c.} `that'; subj. of nominal sentence; \textbf{UD}\bit{-an} \p{c.acc.s./adv.} `day' (= \textit{\v{s}iwattan}); with \textit{par\=a}, adv. `on the preceding day'

\end{notes}

\item[\S\S9--10:] On the next day, on the second day, while the sun is still present (lit. `arrives'), they take this from the house of the ritual client: 1 \textit{t--} of red wool, \textit{t--} of blue wool, 1 woolen \textit{k--}, 1 shekel of silver, 1 \textit{g--} cloth, a little bit of fine oil, 3 flatbreads, (and) 1 pitcher of wine. Then they go to draw the waters of purification, and they draw the waters of purification. And them into the temple of the Deity of the Night---the temple(s\supersc{?} which (are/is\supersc{?}) built from the (original\supersc{?}) temple of the Deity of the Night---into those temples they bring them. And the place them on the roof, and they rest beneath the stars. And on the day which they take the waters of purification, on that day the draw the old deity with red wool and with fine oil from 7 paths (and) 7 footpaths, from mountains, rivers, (and) meadows, from heaven (and) earth. Then within the old temple they draw her, and they bind \textit{u--} on the deity.  And the men of the deity take this: [bunch of junk{\ldots}].


\begin{notes}

\bit{KUKUB} `vessel, container; pitcher' \textit{{\hith}anumanzi} \p{inf.} `draw'; perhaps originally `to the waters to draw', with ambiguity between infinitival reading and dat. of purpose \bit{kue} with \'E, which is plurale tantum;  sing. agreement on ppp. is expected in NH, effectively an extension of TZTr rule extends to ppp. agreement in nominal sentences (see van den H) \bit{\v{S}APAL} \p{prep.} `under; beneath' \textbf{KA.G\`IR{\hpl}} `crossroads (?)' \bit{{\hith}uittiyanzi} \p{3pl.pres.act.} `draw; evoke' 

\end{notes}

\item[\S\S11--12:] {[}junk cont. up to A ii 12] They take this for the \textit{a--} ritual. Then they arrange all this in (its) place. The second day is complete.

\begin{notes}

\bit{tarpala\v{s}} \p{c.nom.s.} note lapse into nominative within the list; all should be accusative, but a significant number surface in nominative

\bit{amba\v{s}i} `(type of ritual)'; based on Hurr. root \textit{amb--} `burn'

\end{notes}

\item[\S\S13--14:] On the third day, when it becomes light, the ritual client comes immediately into the temple. The stars are still standing. Then they bring the waters of purification down from the roof. Then the ritual client comes into the presence of the deity, and he bows to the deity, and he occupies himself with the `ritual of drawing upward'. Then the priest draws up the goddess seven times from there, and also the ritual client draws up (the goddess) seven times. Then they come forth out of the temple into the storehouse, and they perform the \textit{d--} ritual. For the \textit{d--} ritual, they take 1 \textit{m--} bread. The \textit{m--} bread which remains, they take it for the praise(-ritual)  after the \textit{d--} ritual. Wherever (is) good for the ritual client, he goes there. But when on that day at twilight a star leaps out (=`appears'), the ritual client comes into the old temple, (but) he does not bow to the deity. Then he occupies himself with the ritual of blood, and they perform the ritual of blood with a fish, then afterwards they sacrifice either a kid or a lamb. The ritual client \textit{a--}'s, and he stands up.


\begin{notes}

\bit{kar\=uwariwar} \p{n.acc(?).s.} `morning; dawn'; usually adv.; morphologically opaque \bit{U\v{S}KEN} \p{3s.pres.act.} `bow' \bit{tianzi} \p{3pl.pres.act.} `stand'; + \textit{appan} = `occupy onself' (idiomatically)\bit{\=apitaz} \p{n.abl.; a-st.} `offering-pit'; a Wanderw\"ort, per Hoffner; a pit from which chthonic deities are invoked; note object pro-drop in second `drawing up' clause\textbf{\supersc{\'E}}\bit{ABUSSI} `storehouse'  \bit{\=a\v{s}zi} \p{3s.pres.act.} `remain' \textbf{EGIR} \p{prep.} here `after', with Sumerian word order \bit{sarlatti} \p{c.dat.s.} `praise'; often with SISKUR `ritual' \bit{a\v{s}\v{s}u} \p{adj.n.nom.s.} `good'; standard idiom for `whatever X would like' \bit{zurkiya\v{s}} \p{gen.s.; i-st.} `blood'; Hurrian borrowing \bit{arnaminti} \p{3s.pres.mp.} `bow\supersc{?}' etymology/meaning obscure; a nice example of nasal perseveration; cf. copy B below \textit{arnamitti} 

\end{notes}


\item[\S\S15--16:] Then he busies himself with the praise-ritual, and they perform the ritual praise with a sheep. Furthermore they treat the ritual client along with the deity with silver and with the \textit{g}-- plant. Then they burn a lamb for a burnt-offering, and the ritual client bows and goes away to his house. While the sun is still standing on that day, the men of the deity take this: [bunch of junk]. They place it on the roof and it rests beneath the stars. [More junk].

\begin{notes}

\bit{=za} the refl. appears with \textit{pai--} only in the idiom `go home'

\end{notes}

\item[\S\S17--19 :] {[}More junk until end of \S17, A ii 69] This they take inside before the deity for (the ritual of) well-being. The ritual client prepares a gift for the deity, either a silver (symbol of) life or a silver comet\supersc{?}. The third day is finished. \\

When on the 4th day a star leaps out, then the ritual client comes into the temple, and he busies himself with (the ritual of) Pirenkir. And they perform the ritual of well-being for Pirenkir. But when they carry it out (=`do well'), they bring down the deity from the roof and they sprinkle bread crumbs and fruit beside her, and they bring her into the temple. And also inside before the goddess they offer for well-being. And the ritual client rewards the god, the priest, and the \textit{k--} women. Then the ritual client bows and he goes away (home). The 4th day is finished. 


\begin{notes}

\bit{keldiya\v{s}} \p{gen.s.} `well-being'; objective gen. \textbf{N\'IG.BA} `gift' \bit{a\v{s}anuanzi} \p{3pl.pres.act.} `make good; perform well; carry out'; trans.-caus. to \textit{a\v{s}\v{s}--} `be good' \textbf{\bread}\bit{p\=urpure\v{s}} `bread balls' \textbf{IN.BI{\hpl}}\bit{=ya} `fruit' \bit{keldiya} \p{dat.s.} `well-being'; possibly dat. here, but gen. elsewhere \bit{piyan\=aizzi} \p{3s.pres.act.} `reward'

\end{notes}

\item[\S\S20--22 :] On the fifth day when it becomes light, they take 5 flatbreads, 1 \textit{m}-bread of 1/2 an \textit{u}-measure, a \textit{g}-stew, cress, and 1 \textit{\hith}-vessel of beer and they perform the \textit{t-}ritual for the deity. But the ritual patron does not come again. The ritual for the old temple is complete. \\

The new temple which was built, when they have turned themselves about therein, they whirl [{\ldots} with] and a lamb{\ldots}. Behind it they wave {\ldots}. Then they bring the new golden deity together with (her) paraphernalia into the new temple. And they set her down on the table with the \textit{t-}container as (she was) before. When he properly performs (lit. ``make good'') those \textit{t-}rituals, then they pour fine oil in a \textit{t-}vessel, and he speaks as follows: ``O revered deity, preserve (``protect'') your soul, but divide your divinity! Come to that new house also, and occupy the revered place! And when you proceed, occupy precisely that place!'' After that they draw the deity away from the walls 7 times with red wool. After that he places \textit{u}-- within the \textit{t-}vessel of fine oil.


\begin{notes}

\textbf{TU\subsc{7}} `pot; kettle (of soup/stew)' \bit{{\hith}ane\v{s}\v{s}an} \p{c.acc.s.} `(type of vessel)' \bit{karuwiliya\v{s}} \p{adj.gen.s.} `previous, former; old'; hyperbaton? \textbf{GIBIL} `new' (= \textit{newa--}) \bit{wedanta} \p{ppp.n.coll.pl.} `build'; the plural agreement is a feature of early Hittite, which in NH is later switched to singular agreement \bit{we{\hith}antat} \p{3pl.pret.mp.} `turn'; the basic meaning does not seem to fit in context; but in ritual contexts, \textit{anda we{\hith}--} is often used of behavior done in front of the altar (`busy about') \bit{tuppaz} \p{abl.} `(type of vessel)'; instrumental use of ablative \bit{ap\=eni\v{s}\v{s}\=uwan}; note double plene spelling of medial [\textipa{o(:)}] sequence before labial consonant \bit{\=ape} \p{n.acc.pl.} `those'; against Miller, read as n.acc.coll. with SISKUR \textit{tu{\hith}alzi} `those \textit{t}-rituals'; it has been fronted away from the NP \bit{a\v{s}nuzi} \p{3s.pres.act.} `make good; do well/properly' \bit{pa{\hith}\v{s}i} \p{2s.impv.act.} `protect'; \textit{(\v{s})i}-imperative \bit{eda\v{s}\v{s}=a} \p{n.acc.s.} `that'; signals far deixis \bit{\v{s}arrii} \p{2s.impv.act.} `divide; distribute'; \textit{(\v{s})i}-imperative \bit{e{\hith}u} \p{2s.impv.act.} `come'; clause-final position, as are subsequent imperatives


\end{notes}

\item[\S\S23--24 :] Then the \textit{t}-vessel (is) stopped up and they carry it into the new temple, and they put it down separately. They do not place it with the deity. If it pleases (``is good to/for'') the ritual client, on the day on which they perform the \textit{t}-ritual in the old temple, on that day they also draw the new deity into the new temple. But if it does not please him, they draw her on the second day. To draw (her) they take this: [bunch of junk until end A iii 41] and they go out to the river.

\begin{notes}

\bit{ar{\hith}ayan} \p{adv.} `aside; separately'; equivalent to \textit{{\hith}anti} \bit{ITTI} \p{pp.} `with; at' \bit{\=UL=ma} `NEG' \textit{=ma} appears after first constituent, not after \textit{m\=an} \textbf{2\supersc{KAM}} `second'; here, in the sense of the ``following day'' (= Day 6); \textbf{GIBIL}\bit{=kan} `new'; note clause-internal \textit{=kan}, which is not archaic

\end{notes}


\item[\S\S25--26 :] Then they draw the deity from the city of Akkad, from Babylon, from Susa, from Elam, from {\Hith}UR.SAG.KALAM.MA in the city you love (?), from the mountain and from the river, from the sea, from the valley, from the meadow, and from the \textit{\v{s}--}, from heaven and from earth, with the 7 paths and the 7 footpaths. The ritual patron proceeds afterward. \\

And when the correctly complete (``do well'') drawing the deity, then a tent (camp) is made before the river, and they bring the \textit{u}-- within the tents, and they place it on a wicker table. Then they take [some junk].

\begin{notes}
\bit{I\v{S}TU} per CM, an instrumental function is likely here; the (ritual) paths are made with dribbled wine and oil
\bit{\v{S}A} ?? \bit{tarammi} \p{2s.pres.act} `love'; Akk. bit{\v{s}aruntaz} ?? \textbf{{\wood}ZA.LAM.GAR{\hpl}} `tent'; plurale tantum; note singular agreement on participle; \textit{iyan} may not be origin \textbf{AD.KID} `wicker'
\bit{na{\hith}zi} `(a unit of measure)'

\end{notes}


\item[\S\S27--28 :] Then they perform the ritual of blood with a kid. Afterward they perform the praise(-ritual) with a lamb. After that a lamb is burnt as a burnt-offering. After that they together with the table-men bring to the deity all the soups [and some other junk], and they give (it) to the deity to eat. Next they bring the \textit{u--} into the house of the ritual client with to the accompaniment of \textit{a--} and a drum. And they sprinkle beside it (i.e. the \textit{u--}) sourdough bread, crumbled cheese, and fruits. Then they wave around \textit{{\hith}--}. Then they settle the deity in the storehouse.  \\

Then as a burn-offering [a bunch of junk] is prepared. They give (it) to the deity as a burnt-offering. They they bring in the \textit{u--} to the deity, and they bind the \textit{u--} on the new deity. There is neither a blood ritual nor a praise ritual. And the ritual client goes away (home).


\begin{notes}

\bit{\v{s}arlattanza} \p{n.acc.s.} `praise(-ritual)'; a Luwianism, with \textit{=sa} particle; also possible is Luw. c. acc. pl.; previously this word has shown regular Hitt. \textit{a}-st. inflection \textbf{\wood}\bit{arkammit} \p{instr.} `harp (?)' \bit{galgalt\=urit} \p{instr.} `drum' \bit{{\hith}amankanzi} \p{3pl.pres.act.} `bind' \bit{=za} only with \textit{pai--} in the idiom `go home'


\end{notes}


\item[\S\S29--30 :] They take those things---[a bunch of junk]---and they go to the waters of purification. Then they place them on the roof and they rest beneath the stars. On that day they do nothing else.  \\

They take those things---[a bunch of junk].


\begin{notes}



\end{notes}



\item[\S\S31--33 :] The \textit{u--} which was brought from the old temple{\ldots}they open that \textit{t}-vessel. The waters with which they wash the walls of the temple, they mix that old fine oil from the \textit{t-}vessel into it. And with that they wash the wall so that the wall is pure. But the ritual client does not come. \\

Then they bind the old \textit{u--} onto the red scarf of the new deity. \\

And when a star leaps out at the time of twilight on the second day, then the ritual client comes into the temple, and he bows to the deity. And the two knives which were made with (?) the new deity, they take those, and they dig a ritual pit for the deity in front of the table. Then they consecrate 1 sheep to the to the deity for \textit{e--}, and they slaughter it (so that the blood flows) down into the pit. There is no drawing from the walls, and a small table is placed (there). And they bloody the golden deity, the wall, and all the paraphernalia of the new deity, and the new deity and the temple become pure. The fat(ty meat) is burnt up; no one eats it. 


\begin{notes}

\bit{kui\v{s}} the relative clause is not properly resumed; the scribe was clearly flummoxed 

\bit{enuma\v{s}\v{s}iya} \p{d-l.s.} `reconciliation (?)' \bit{{\hith}addanzi} \p{3pl.pres.act.} `pierce, stab; slaughter'; \bit{{\hith}atte\v{s}ni} is clearly the derived verbal noun \textbf{SUD}\bit{-a\v{s}} \p{c.nom.s.} `drawing' ( < \textit{{\hith}uitt(iya)--});  

\end{notes}



\item[Colophon:] The first tablet of the word of the priest of the deity of the night. (It is entitled:) ``When someone settles the deity of the night separately, this ritual is for him. (It is) not finished. The hand of Ziti, son of NU.\supersc{GI\v{S}}KIRI\subsc{6} wrote (it) in the presence of Anuwanza, the \supersc{L\'U}SAG


\begin{notes}

\bit{kui\v{s}} bare indefinite under \textit{m\=an} \textbf{\supersc{L\'U}SAG} `eunuch (?)'; in Akk., it certainly means `eunuch', but in Hittite it may simply be an official title


\end{notes}

\bigskip
\item[KUB 32.133, Obv. I]
\bigskip

\item[\S1] Thus (speaks) his majesty Mursili, the great king, the son of Suppliuluma, the great king, hero: When my ancestor Tuthaliya,, the great king, divided the deity of the night from the temple in Kizzuwatna and worshipped (lit. `did') her separately in the temple in Samuha, the rituals (and) regulations which he bound on himself in the temple of the deity of the night, the wooden tablet-scribes and the men of the temple proceeded to begin to alter them; then I Mursili the great king restored them on the basis of the tablets. And in the future when in a temple of the deity of the night in Samuha either a king or a queen or son of a king or daughter of a king comes into the temple of the deity of the night, let them perform these rituals.


\begin{notes}

\bit{n=an=z=an} \p{cl-pron.3s.acc.s.} NH doubling of clitic pronoun; per Yakubovich/Rieken, likely Luwian interference, since in Luwian the pron.cl. follows the reflexive particle \bit{{\hith}azziwita} \p{n.acc.pl.} `ritual'  \bit{kattan} \p{pp} `beside; on'; in NH, with refl. ptcl. \textit{=za} as object \bit{{\hith}amankatta} \p{3s.pres.mp.} `bind' \bit{wahnu\v{s}kewanda} \p{3pl.pres.mp.} `turn; overturn; alter, change'; a nice example of QSV with object clitic \textit{=at} \index{quasi-serial verb construction} \bit{appa aniyanun} \p{1s.pret.act.} `(re-)do; bring about' + \textit{tuppa--} = `set down in writing'; instr. use of abl. \textbf{\v{S}\'U} likely a scribal error in effort to produce Glossenkeil \bit{m\=an{\ldots}na\v{s}ma{\ldots}na\v{s}\v{s}u{\ldots}na\v{s}ma} \p{conj.} `either{\ldots}or{\ldots}or{\ldots}or' 
 
\end{notes}


\item[\S\S2--3] On the first day (is) the \textit{d-}ritual. They take this: [bunch of junk to end of \S2.] When she arranges the ritual paraphernalia, and in addition, while the sun is still up, then the \textit{k}-woman takes a purified silver vessel, and binds [some junk] onto it. Next she takes 3 {[\hith}-vessels, flatbread, wine (and) fine oil] and she goes to draw the waters of purification. When she arrives at the spring, then she breaks the bread and throws it into the spring. She also libates wines down (into it) and also dribbles fine oil down (into it). Then she draws the water. Next {\ldots}brings up{\ldots}waters of purification.


\begin{notes}

\textbf{P\'U} `spring' \bit{zapnuzzi} \p{3s.pres.act.} `trickle; dribble'

\end{notes}

\end{description}


\section{CTH 344: ``Song of Exiting'' (or: ``Theogony'')}

\begin{description}

\item[Publication:] KUB 33.120 + 119 + 36.31 + 48.97 + KBo 52.10; KUB 36.1
\item[Edition:]
\item[Translation(s):] 
\item[Background:] 

\item[1--11 :] (Of) the gods who (are) primordial and [the o]the[r] mighty gods, let them hear. Let Nara, Napsara, Minki, (and) Ammunki hear. Let Ammezadu{\ldots}the mother and the father hear. Enlil (and) Ninlil, those mighty eternal gods who are above and below, and the \textit{k--},  let them hear! (?) Formerly, in primordial times, Alalu was king in heaven. Alalu was seated on the throne, and mighty Anu, the foremost gods, stands up before him. He bows down at (his) feet, he places clay vessels for drinking in his hand.



\begin{notes}

\textbf{\supersc{d}EN.L\'IL} `Enlil' (lit. `lord of the field') \textbf{NIN.L\'IL} `Ninlil' (lit. `sister of the field') \bit{katta \v{s}araya} \p{3s.pres.mp.?} `??' \bit{wakt\=uri\v{s}} \p{adj.c.nom.pl.} `eternal'; rare variant of regular zero-grade \textit{\#ukt--} \bit{kulkulimma\v{s}\v{s}=a} \p{adj.c.nom?.pl.} `(?)'; probably reduplicated, but the meaning is opaque \textbf{MU\hpl} `year' \textbf{{\wood}S\'U.A}\bit{-ki} `chair; throne' \bit{{\hith}inki\v{s}kitta} \p{3s.pret.mp.} `bow';\textbf{NAG}\bit{-na\v{s}=\v{s}i=kan} \p{n.gen.s.} `drink'; obj. gen. of verbal noun \textbf{GAL\hpl}\bit{-u\v{s}} \p{c.acc.pl.} `fired clay vessel' (= \textit{zeriu\v{s}}; \textbf{\v{S}U}\bit{=\v{s}\v{s}i} `(his) hand' \bit{{\hith}antezziya\v{s}=(\v{s})mi\v{s}} \p{adj.c.nom.s.} `foremost'; split genitive construction is unexpected, normally only with inalienable possession; the referent of poss.cl. is `gods'---hence lit. `their$_i$ foremost of the gods$_i$' 


\end{notes}

\item[12--17 :] For nine few years, Alalu was king in heaven. In the ninth year, Anu gave battle against Alalu. And he overcame him, (viz.) Alalu. And he fled before him, and he went down into the dark earth. Down into the dark earth he went, and Anu sat down on the throne. Anu sits on his throne, and mighty Kumarbi gives to him to eat. He bows down at his feet, he places clay vessels for drinking in his hand.

\begin{notes}

\bit{=an} \p{ptcl-loc.} `in'; archaic use of clitic local particle \bit{taru{\hith}ta=an=za=an} \p{3s.pres.act.} `overcome'; translit. now $<$\textit{-u{\hith}-}$>$; clitic right-doubling structure, a characteristically Hurrian feature also found in Ullikummi, etc.; doubling of cl.pron. again \bit{n=an=kan} \p{cl-c.acc(!).s.} expect nominative \textit{=a\v{s}} as subject of unaccusative \textit{pait} `went' \textbf{K\'U}\bit{-na} \p{inf.} `eat' (=\textit{aduna})

\end{notes}

\item[18--36 :] For 9 few years Anu was king in heaven. In the ninth year, Anu gave battle against Kumarbi. Kumarbi, progeny of Alalu, gave battle against Anu, and he could not endure the eyes of Kumarbi,  Anu. From Kumarbi, he ran away from his hands, and he fled, (viz.) Anu. Then he set out for heaven, and behind him Kumarbi advanced. And he seized him by the feet, (viz.) Anu, and dragged him down from heaven. He bit his loins, and his manliness became fused with Kumarbi's insides like bronze. Kumarbi swallowed down the manliness of Anu, and he rejoiced and he laughed. Anu turned back toward him and began to speak to Kumarbi: ``Are you rejoicing to yourself, because you swallowed my manliness? Do not rejoice! I have placed a burden in your innards. I have impregnated you with the mighty Storm-god. A second time I impregnated you with the river Aranzahi, a thing not to be endured. A third time I impregnated you with mighty Tasmi(su). And also I have placed two terrible gods in your innards (as) burdens. And you (shall) proceed to finish striking your head on the rocks of Mt. Tassa.  

\begin{notes}

\bit{manzazzi} \p{3s.pres.act.} `endure; resist' \bit{ki\v{s}\v{s}arazza=\v{s}it=a\v{s}ta} \p{abl.} `hand'; the construction is likely just due to translation, but if real, it would show ``left dislocation'' (or hanging topic) of a marked dative (!!) \bit{{\hith}uellait} \p{3s.pret.act.} `escape; run away' \bit{saligas} \p{3s.pret.act.} `touch; approach; come near'; used for unwanted physical contact \bit{parsinu\v{s}=(\v{s})u\v{s}} `(his) loins' \bit{uli\v{s}ta} \p{3s.pret.act.} `blend' \bit{pa\v{s}ta} \p{3s.pret.act.} `swallow' \bit{aimpan} `burden' (??)---check here \bit{arma{\hith}{\hith}un} \p{1s.pret.act.} `impregnate' \bit{\=a\v{s}ma} near deictic equiv. of \textit{ka\v{s}ma}  \bit{mazzuwa\v{s}} \p{gen.s.} `enduring'; idiomatically with \textit{\=UL}

\end{notes}

\item[37--46 :] When Anu had finished speaking, he went up to heaven, and he concealed himself. He spit forth (from) his mouth, Kumarbi the wise king. He spit out (from) his mouth, and from his mouth he spit out manliness mixed with spittle\supersc{?}. When Kumarbi spit out, Mt. Ganzura gave birth to fearsome Tasmisu. Kumbari{\ldots} went to the city of Nippur and {\ldots} he sat down. Kumarbi did not{\ldots}. For seven months{\ldots}


\begin{notes}

\bit{=za munnaittat} \p{3s.pret.mp.} `hide'; the reflexive particle is probably a late addition, unlikely for OH \bit{par\=a allapahha\v{s}} \p{3s.pret.act.} `spit forth/out' \textbf{KAxU}\bit{[allit]} \p{n.instr.s.} `spittle'; the restoration is uncertain, but there is a marginally attested \textit{i\v{s}\v{s}alli} meaning something like `spittle', which would fit the sense here \bit{{\hith}a\v{s}ta} \p{3s.pret.act.} `give birth'; the mountain gives birth as a consequence of being spat upon by Kumarbi; whether the name of the deity birthed is correct is unclear \bit{{\hith}attanza} \p{adj.m.nom.s.; n-st.} `wise' \bit{iyawaniyawanza} \p{adj.c.nom.s.} `wailing'; likely a pseudo-participial use of \textit{--want--} to a \textit{--ya--} denominative, which have effectively the semantics of active ptcpl. \bit{i\v{s}{\hith}a\v{s}\v{s}wantiyanza} \p{ppp.} how this sequence is to be parsed is unclear; likely contained are forms related to the  r/n-st. \textit{i\v{s}ha\v{s}\v{s}ar} `lordship; dominion' / denom. \textit{i\v{s}ha\v{s}\v{s}arwai--} `rule over' 

\end{notes}

\bigskip
\item[ii]
\bigskip

\item[1--28 :] {\ldots}Kumarbi \p{acc.}{\ldots} ``Come out from his body or come out from {\ldots} or come out from his `good place'!'' A.GILIM began to speak words to Kumarbi within (him): ``Be alive, (o) lord of the source of wisdom. If{\ldots} come out{\ldots} bitten off{\ldots}. Will the earth give me its power. Will heaven give me its bravery. Will Anu give me his manliness. And Kumarbi shall give me his wisdom. Nara shall give me {\ldots}. Will Napsara give me {\ldots}. Will shall give me his power. Will {\ldots} give me his dignity and his wisdom. In all hearts (?) {\ldots} And also life{\ldots} cow Seri{\ldots}wagons{\ldots}let them {\ldots}. Suwaliyat {\ldots} When he gave {\ldots}, he {\ldots} to me. Anu began to {\ldots}``Come{\ldots}I {\ldots} what I gave {\ldots} Come{\ldots} [more junk] Come! If it is good, from the good place {\ldots}.''


\begin{notes}
 
\textbf{N\'I.TE}\bit{-az} `body' (= \textit{tuekka}) \textbf{\supersc{d}GILIM}\bit{-a\v{s}} `water-binder (?)' (= St.G) \textbf{INIM.{\Hith}I.A}\bit{-ar} `word; matter' \textbf{TI}\bit{-anza \=e\v{s}} idomatic greeting in Hittite; also found is similar \textit{hu\=e\v{s}} `live!' \bit{{\hith}a\v{s}umna\v{s}} \p{n.gen.s.; r/n-st.} `headwater; source' \textbf{KI}\bit{-a\v{s}} `earth' (= \textit{taganzipa\v{s}}) \textbf{KALAG}\bit{-tar=\v{s}et} \p{n.acc.s.; r/n-st.} `(its) strength; power' \textbf{UR.SAG}\bit{-liyatar} \p{n.acc.s.; r/n-st.} `heroicism; bravery'


\end{notes}

\item[29--54 :] And in Kumarbi's insides (someone) began to speak to Ea: ``{\ldots} place. If I {\ldots}, he will crush [me] like a reed. If I come out {\ldots}, then it will make me impure within, and it will make me impure in (lit. `with') (my) ear. If I come out from `the good place', a woman shall \v{s}-- me upon my head. When{\ldots}the Storm-God within {\ldots}, he arranges (?) them. Then he split him like a stone, Kumarbi (in) his skull. And he came up from his skull, the Storm-God, the hero, the king. When he set out, he stood before Ea, (did) Kumarbi and he bowed. Then he fell down, (did) Kumarbi. From{\ldots} he struck (?). And afterward he began to seek NAM.HE. Then he began to speak to Ea: ``Give me (my) son so that I can eat him up! Whoever {\ldots} me like a woman{\ldots}[lots of broken context until 51]. The Sun-god of heaven saw him{\ldots}Kumarbi began to eat. The granite {\ldots}-ed the teeth in Kumarbi's mouth. When it had {\ldots}-ed his teeth, he began to wail.


\begin{notes}

\bit{za{\hith\hith}urai\v{s}kezzi} `crush; smash; grind' \bit{apatta} \p{adv.} `there; then' \bit{se\supersc{?}{\hith}ui\v{s}kezzi} \p{3s.pres.act.} `?' \bit{\v{s}e\v{s}{\hith}atta} \p{3s.pres.mp.} `arrange; order; instruct' (?); a Luwianism? \textbf{KA.ZAL} per GB, = the Storm-God \bit{par\v{s}anut} \p{3s.pret.act.} `break; split' (cf. \textit{par\v{s}iya--}) \bit{tarna\v{s}\v{s}an} \p{c.acc.s.; a-st} `head; skull'; note Hoffner's wrong interpretation (from \textit{tarnai--})

%i should double check this analysis of tarna-- `skull'; maybe it is tarnassa-?

\end{notes}

\item[55--87 :] {\ldots}, Kumarbi\p{nom.} and he began to speak words: ``Who have I been afraid of?'' {\ldots}When Kumarbi{\ldots} and he began to speak to Kumarbi. Let them call the rock{\ldots} Let it lie{\ldots}. He threw the granite {\ldots} Let them proceed to call you {\ldots}. Let rich men, heroes, lords, slaughter for you cows (and) sheep. Let the poor men offer to you with meal. [broken until 71] {\ldots} They began to slaughter{\ldots} and they began to offer with meal. [more broken, maybe some of this can be saved] Then the hero, the Storm-god came out [from the good pla]ce. [more broken until 79]{\ldots} he came out. They made her give birth, the{\ldots} like a woman [more broken until 87].

\begin{notes}

\bit{pandu} \p{3pl.impv.act.} `go'; nice example of QSV \index{quasi-serial verb construction}, with clitic climbing of \textit{=tta} \bit{d\=ayer} \p{3rd.pret.act.} `place'; in supine construction; note minimal pair with \textit{tiy\=er} in subsequent clause, which very clearly shows there is no functional contrast

\end{notes}

\bigskip
\item[Col. iii]
\bigskip

\item[1-- 72:] (I missed this) 


\begin{notes}

(notes here)

\end{notes}



\item[1--27 :] When the earth was wailing, [{\ldots}] she bore sons. A messenger went, and on his throne Ea approved the pleasant word. [{\ldots}] ``The earth bore two sons. When august Ea heard the words, the messenger who went down to him with a [word], Ea king of the gods rewarded him with a gift. He [puts] a garment on his body and [he puts] on a fancy shirt on his chest. And he wraps a \textit{I--} of silver around the messenger. (?) (I missed the colophon)


\begin{notes}

\bit{malait} \p{3s.pret.act.} `approve' \bit{\v{sa}nezzi} `pleasant'; \textit{piyanait} \p{3s.pret.act.} `reward' (+ instr.); here, with unmarked instr. N\'IG.BA `gift' \textbf{\supersc{T\'UG}G\'U.\'E.A}\bit{-a\v{s}=\v{s}i} `fancy shirt' \bit{ammuk} CM: ``an unusually egotistic colophon''

\end{notes}

\end{description}

\section{Mursili's Plague Prayer (2nd version) --- CTH 378.2.C}

\begin{description}

\item[Publication:] KUB 14.10 + 16.86
\item[Edition:]
\item[Translation(s):] \citet[57ff.]{singer2002hittite}
\item[Background:] 

\bigskip

\item[1--18 :] O Storm-god of Hatti, my lord, o gods, my lords, Mursili, your servant, has sent me (saying): ``Go speak thus to the Storm-god of Hatti, my lord, (and) the gods, my lords: `What (in the hell) have you done? You have let a plague into the land of Hatti, such that the land of Hatti has been oppressed very much by the plague. In the time of my father, in the time of my brother (people) were dying. Since I have have become priest to the gods, also now in my time there are (people) dying. This is the 20th year since (people) have been (lit. `are') dying in the land of Hatti. And the plague is still not dispersed. I cannot overcome the agitation in (lit. `from') my heart. I cannot overcome the anxiety in my body.

\begin{notes}

\bit{{\I}t} \p{2s.impv.act.} `go'; the fronted imperative without focus particle is rare; the entire opening is quite irregular, since normally the conceit is that Mursili himself is speaking to/pleading with (\textit{arkuwar}) the gods \textbf{\v{S}ABI} `within' (= \textit{\v{S}A}, but fancier) \bit{arumma} `very' \bit{PANI} regularly used to express `in the time of; during the reign of' \bit{kuitt=a=ya} \p{conj.} `since'; redundant use of conjunction \bit{arha{\ldots}tarupt\=ari} \p{3s.pres.mp.} `disperse'; this is a case where \textit{arha} instantiates semantic reversal, since \textit{tarupp--} usually means `gather; collect' \textbf{\v{S}A}\bit{-az} \p{abl.} `heart'; the normal Hittite idiom is +abl. 

\end{notes}

\item[19-- :] Furthermore, when(ever) I performed the state festivals also, I busied myself before all the gods. I did not choose (lit. `place') one temple. On account of the plague I made a plea to all the gods. 

\begin{notes}

\textbf{EZEN\subsc{4}.{\Hith}I.A}\bit{=ya} `state festivals'; the `whenever' reading is here, and \textit{=ya} might mark a `fronted' element --- so this looks like a Held--Garrett violation \textbf{EGIR}\bit{-pa iya{\hith\hith}at} \p{3s.pret.mp.} `go back (and forth); busy oneself' \bit{te{\hith\hith}un} \p{1s.pret.act.} `place'; in this context, must have an idiomatic sense `choose'

\end{notes}


\item[1{\pr}--24 :] {\ldots}so that they attacked Amga, the borderland of Egypt.


\begin{notes}

\bit{mena{\hith\hith}anda} here, `toward; with' \bit{peran wa{\hith}nuer} \p{3pl.pret.act} `do an about-face' \bit{{\hith}\=ud\=ak} \p{adv.} `all at once'; the double plene must be linguistically significant, but what exactly it means is unclear \bit{namma=ya} an `accidental' initial \textit{namma}; it means `again', but has been fronted \bit{anku} \p{ptcl.} rare; some kind of asseverative

\end{notes}

\item[25-- :]  

\begin{notes}

\bit{hannes\v{s}it sarl\=ait} \p{3s.pret.act.} `(lit.) raise up with the law case' = `cause to prevail'

\end{notes}

%--------------------
%------------------------------------------------------------------------------
%--------------------

\printindex

\bibliographystyle{chicagoa}
\bibliography{/Users/adyates/Dropbox/Alexandria-BibTeX}

\end{document}
